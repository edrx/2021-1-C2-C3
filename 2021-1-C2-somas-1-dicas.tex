% (find-LATEX "2021-1-C2-somas-1-dicas.tex")
% (defun c () (interactive) (find-LATEXsh "lualatex -record 2021-1-C2-somas-1-dicas.tex" :end))
% (defun C () (interactive) (find-LATEXsh "lualatex 2021-1-C2-somas-1-dicas.tex" "Success!!!"))
% (defun D () (interactive) (find-pdf-page      "~/LATEX/2021-1-C2-somas-1-dicas.pdf"))
% (defun d () (interactive) (find-pdftools-page "~/LATEX/2021-1-C2-somas-1-dicas.pdf"))
% (defun e () (interactive) (find-LATEX "2021-1-C2-somas-1-dicas.tex"))
% (defun o () (interactive) (find-LATEX "2021-1-C2-somas-1-dicas.tex"))
% (defun u () (interactive) (find-latex-upload-links "2021-1-C2-somas-1-dicas"))
% (defun v () (interactive) (find-2a '(e) '(d)))
% (defun d0 () (interactive) (find-ebuffer "2021-1-C2-somas-1-dicas.pdf"))
% (defun cv () (interactive) (C) (ee-kill-this-buffer) (v) (g))
%          (code-eec-LATEX "2021-1-C2-somas-1-dicas")
% (find-pdf-page   "~/LATEX/2021-1-C2-somas-1-dicas.pdf")
% (find-sh0 "cp -v  ~/LATEX/2021-1-C2-somas-1-dicas.pdf /tmp/")
% (find-sh0 "cp -v  ~/LATEX/2021-1-C2-somas-1-dicas.pdf /tmp/pen/")
%     (find-xournalpp "/tmp/2021-1-C2-somas-1-dicas.pdf")
%   file:///home/edrx/LATEX/2021-1-C2-somas-1-dicas.pdf
%               file:///tmp/2021-1-C2-somas-1-dicas.pdf
%           file:///tmp/pen/2021-1-C2-somas-1-dicas.pdf
% http://angg.twu.net/LATEX/2021-1-C2-somas-1-dicas.pdf
% (find-LATEX "2019.mk")
% (find-CN-aula-links "2021-1-C2-somas-1-dicas" "2" "c2m211somas1d" "c2sod")
%
% «video-1»  (to ".video-1")
% (c2m211somas1da "video-1")
% (find-ssr-links "c2m211somas1d" "2021-1-C2-somas-1-dicas" "pCD1p9FZYdI")
% (code-video     "c2m211somas1dvideo" "$S/http/angg.twu.net/eev-videos/2021-1-C2-somas-1-dicas.mp4")
% (find-c2m211somas1dvideo "0:00")
% (find-c2m211somas1dvideo "0:49" "na aula passada várias pessoas disseram que não tão entendendo nada")
% (find-c2m211somas1dvideo "0:52" "e que os vídeos que eu fiz não tão ajudando nada")
% (find-c2m211somas1dvideo "3:35" "eu nunca iria descobrir ... somatórios")
% (find-c2m211somas1dvideo "3:43" "a gente supõe que os alunos sabem determinadas coisas")
% (find-c2m211somas1dvideo "14:09" "dica 7")
%
% «video-2»  (to ".video-2")
% (c2m211somas1da "video-2")
% (find-ssr-links "c2m211somas1d2" "2021-1-C2-somas-1-dicas-2" "Q0dxniyVZuA")
% (code-video    "c2m211somas1d2video" "$S/http/angg.twu.net/eev-videos/2021-1-C2-somas-1-dicas-2.mp4")
% (find-c2m211somas1d2video "0:00")

% «.video-1»			(to "video-1")
% «.video-2»			(to "video-2")
%
% «.defs»			(to "defs")
% «.title»			(to "title")
% «.telegram-2021jun23»		(to "telegram-2021jun23")
% «.comentario-sobre-a-P1»	(to "comentario-sobre-a-P1")
% «.dica-7»			(to "dica-7")
%
% «.djvuize»			(to "djvuize")

\documentclass[oneside,12pt]{article}
\usepackage[colorlinks,citecolor=DarkRed,urlcolor=DarkRed]{hyperref} % (find-es "tex" "hyperref")
\usepackage{amsmath}
\usepackage{amsfonts}
\usepackage{amssymb}
\usepackage{pict2e}
\usepackage[x11names,svgnames]{xcolor} % (find-es "tex" "xcolor")
\usepackage{colorweb}                  % (find-es "tex" "colorweb")
%\usepackage{tikz}
%
% (find-dn6 "preamble6.lua" "preamble0")
%\usepackage{proof}   % For derivation trees ("%:" lines)
%\input diagxy        % For 2D diagrams ("%D" lines)
%\xyoption{curve}     % For the ".curve=" feature in 2D diagrams
%
\usepackage{edrx15}               % (find-LATEX "edrx15.sty")
\input edrxaccents.tex            % (find-LATEX "edrxaccents.tex")
\input edrxchars.tex              % (find-LATEX "edrxchars.tex")
\input edrxheadfoot.tex           % (find-LATEX "edrxheadfoot.tex")
\input edrxgac2.tex               % (find-LATEX "edrxgac2.tex")
%
%\usepackage[backend=biber,
%   style=alphabetic]{biblatex}            % (find-es "tex" "biber")
%\addbibresource{catsem-slides.bib}        % (find-LATEX "catsem-slides.bib")
%
% (find-es "tex" "geometry")
\usepackage[a6paper, landscape,
            top=1.5cm, bottom=.25cm, left=1cm, right=1cm, includefoot
           ]{geometry}
%
\begin{document}

%\catcode`\^^J=10
%\directlua{dofile "dednat6load.lua"}  % (find-LATEX "dednat6load.lua")

% %L dofile "edrxtikz.lua"  -- (find-LATEX "edrxtikz.lua")
% %L dofile "edrxpict.lua"  -- (find-LATEX "edrxpict.lua")
% \pu

% «defs»  (to ".defs")
% (find-LATEX "edrx15.sty" "colors-2019")
\long\def\ColorRed   #1{{\color{Red1}#1}}
\long\def\ColorViolet#1{{\color{MagentaVioletLight}#1}}
\long\def\ColorViolet#1{{\color{Violet!50!black}#1}}
\long\def\ColorGreen #1{{\color{SpringDarkHard}#1}}
\long\def\ColorGreen #1{{\color{SpringGreenDark}#1}}
\long\def\ColorGreen #1{{\color{SpringGreen4}#1}}
\long\def\ColorGray  #1{{\color{GrayLight}#1}}
\long\def\ColorGray  #1{{\color{black!30!white}#1}}
\long\def\ColorBrown #1{{\color{Brown}#1}}
\long\def\ColorBrown #1{{\color{brown}#1}}
\long\def\ColorOrange#1{{\color{orange}#1}}

\long\def\ColorShort #1{{\color{SpringGreen4}#1}}
\long\def\ColorLong  #1{{\color{Red1}#1}}

\def\frown{\ensuremath{{=}{(}}}
\def\True {\mathbf{V}}
\def\False{\mathbf{F}}
\def\D    {\displaystyle}

\def\drafturl{http://angg.twu.net/LATEX/2021-1-C2.pdf}
\def\drafturl{http://angg.twu.net/2021.1-C2.html}
\def\draftfooter{\tiny \href{\drafturl}{\jobname{}} \ColorBrown{\shorttoday{} \hours}}



%  _____ _ _   _                               
% |_   _(_) |_| | ___   _ __   __ _  __ _  ___ 
%   | | | | __| |/ _ \ | '_ \ / _` |/ _` |/ _ \
%   | | | | |_| |  __/ | |_) | (_| | (_| |  __/
%   |_| |_|\__|_|\___| | .__/ \__,_|\__, |\___|
%                      |_|          |___/      
%
% «title»  (to ".title")
% (c2m211somas1dp 1 "title")
% (c2m211somas1da   "title")

\thispagestyle{empty}

\begin{center}

\vspace*{1.2cm}

{\bf \Large Cálculo 2 - 2021.1}

\bsk

Material extra: dicas de estudo

pro ``integrais como somas de retângulos (1)''

\bsk

Eduardo Ochs - RCN/PURO/UFF

\url{http://angg.twu.net/2021.1-C2.html}

\end{center}


\newpage

%  _____    _                                    _             ____  _____ 
% |_   _|__| | ___  __ _ _ __ __ _ _ __ ___     (_)_   _ _ __ |___ \|___ / 
%   | |/ _ \ |/ _ \/ _` | '__/ _` | '_ ` _ \    | | | | | '_ \  __) | |_ \ 
%   | |  __/ |  __/ (_| | | | (_| | | | | | |   | | |_| | | | |/ __/ ___) |
%   |_|\___|_|\___|\__, |_|  \__,_|_| |_| |_|  _/ |\__,_|_| |_|_____|____/ 
%                  |___/                      |__/                         
%
% «telegram-2021jun23»  (to ".telegram-2021jun23")
% (find-angg ".emacs" "c2-2021-1-telegram")
% (find-angg ".emacs" "c2-2021-1-telegram" "mas que voces discutam aqui")

{\bf Telegram 23/junho/2021}

\msk

{\sl Professor, mas um vídeo explicando a matéria antes dos exercícios
cessaria muitas dúvidas que as vezes são só sobre interpretação...}

\msk

Tá, mas qual matéria? Pra mim a matéria de Cálculo 2 são centenas de
itens diferentes, e os videos e os PDFs que eu já fiz cobrem muitos
deles e deixam alguns como exercícios... eu preciso que vocês comecem
a me dizer coisas tipo ``empaquei nesse pedaco aqui do exercicio tal''
-- e me mandem foto -- ou me digam coisas tipo ``eu não entendi o que
você fez no momento 4:32 do video tal...

Por exemplo, eu nunca iria descobrir que vários de vocês nunca tinham
visto somatório até ver umas coisas que umas pessoas escreveram...

% Estou me referindo a um vídeo explicando o pdf, eu consegui fazer os
% exercícios porque a Isa me ajudou a interpretar muitas das questões
% ja que me atrasei um pouco na primeira semana

\ssk

(...)

\ssk

Cada PDF tem vários exercicios e muitas dezenas de idéias. Se vocês
disserem só ``faz um video explicando o PDF'' eu vou fazer um video de
5 minutos explicando tudo de um PDF por alto, mas se vocês fizerem
perguntas mais específicas aí eu consigo fazer videos bem mais
detalhados sobre aquelas perguntas ou sobre aqueles exercicios.

Gente, vocês não estao discutindo pra descobrir como resolver
os exercicios?

O proximo passo é voces passarem a discutir pra encontrarem
boas perguntas pra fazer...

\ssk

(...)

\ssk

Gente, a matéria de matemática fica cada vez mais difícil à medida que
as matérias ficam mais avançadas e passa a ser comum ter trechos de
uma linha ou de um parágrafo nos livros-texto que vocês vão passar
muitas horas tentando decifrar aquilo. O meu objetivo aqui é fazer
vocês aprenderem a se virar com isso, e a técnica pra isso é vocês
aprenderem a escrever as hipóteses de vocês e aprenderem a fazer
perguntas. A maioria das perguntas voces vão conseguir responder
sozinhos, algumas vocês vão conseguir descobrir a resposta conversando
com amigo (que tambem nao sabia a resposta!) e umas poucas vocês vão
empacar mesmo e não vão conseguir resolver sozinhos.

Mas mandem as dúvidas de voces! Todo mundo aqui tá
tentando aprender não só a matéria como tambem a escrever e a
perguntar...

O pior que pode acontecer é eu ficar meio desesperado pensando
``caramba, essa pessoa devia ter aprendido isso no ensino médio mas
não aprendeu... como é que eu ajudo ela com isso? E deve ter dezenas
de pessoas com a mesma dúvida que ela... como é que eu preparo
material `de revisão' pra todo mundo que tiver duvidas sobre isso?''

Quando a gente explica algo pra alguém a gente vai ajustando
o nível de detalhe das nossas explicações pelas reações da pessoa.
Com video-aulas é mais difícil, mas eu tou tentando fazer um
esquema em que eu gravo os videos, voces me dizem onde tem algo
que vocês não entenderam, e aí eu mando mais detalhes.

Só que C2 tambem é um curso em que vocês vao ter que aprender a
escrever certas coisas na linguagem matemática certa, então quando eu
digo pra voces perguntarem eu tambem tenho um segundo objetivo, que é
fazer vocês escreverem e a gente poder ir discutindo o que é essa
``linguagem matematica certa''.

Dêem uma olhada nisso aqui (o ``Comentário sobre a P1''):

{\footnotesize

% http://angg.twu.net/LATEX/2020-1-C2-P1.pdf#page=10
\url{http://angg.twu.net/LATEX/2020-1-C2-P1.pdf#page=10}

}

\newpage

%   ____                           _     ____  _ 
%  / ___|___  _ __ ___   ___ _ __ | |_  |  _ \/ |
% | |   / _ \| '_ ` _ \ / _ \ '_ \| __| | |_) | |
% | |__| (_) | | | | | |  __/ | | | |_  |  __/| |
%  \____\___/|_| |_| |_|\___|_| |_|\__| |_|   |_|
%                                                
% «comentario-sobre-a-P1»  (to ".comentario-sobre-a-P1")
% (c2m201p1p 10 "comentario-telegram")
% (c2m201p1     "comentario-telegram")

{\bf Comentário sobre a P1}


{\footnotesize

% http://angg.twu.net/LATEX/2020-1-C2-P1.pdf#page=10
\url{http://angg.twu.net/LATEX/2020-1-C2-P1.pdf#page=10}

}

\newpage

%  ____  _             _____ 
% |  _ \(_) ___ __ _  |___  |
% | | | | |/ __/ _` |    / / 
% | |_| | | (_| (_| |   / /  
% |____/|_|\___\__,_|  /_/   
%                            
% «dica-7»  (to ".dica-7")
% (c2m211somas1dp 7 "dica-7")
% (c2m211somas1da   "dica-7")
% (find-LATEXgrep "grep --color=auto -niH --null -e 'dica 7' *.tex")
% (saptp 5 "dica-7")
% (sapta   "dica-7")

\vspace*{-1.0cm}

{\bf Dica 7 (de GA, mas vale pra C2 também)}

Uma solução bem escrita pode incluir, além do resultado final, contas,
definições, representações gráficas, explicações em português, testes,
etc. Uma solução bem escrita é fácil de ler e fácil de verificar. Você
pode testar se uma solução sua está bem escrita submetendo-a às
seguinte pessoas: a) você mesmo logo depois de você escrevê-la ---
releia-a e veja se ela está clara; b) você mesmo, horas depois ou no
dia seguinte, quando você não lembrar mais do que você pensava quando
você a escreveu; c) um colega que seja seu amigo; d) um colega que
seja menos seu amigo que o outro; e) o monitor ou o professor.

Se as outras pessoas acharem que ler a sua solução é um sofrimento,
isso é mau sinal; se as outras pessoas acharem que a sua solução está
claríssima e que elas devem estudar com você, isso é bom sinal. {\sl
  GA é um curso de escrita matemática:} se você estiver estudando e
descobrir que uma solução sua pode ser reescrita de um jeito bem
melhor, não hesite ---\ColorRed{ reescrever é um ótimo exercício}.


\newpage

{\bf Videos}

Video 1:

{\footnotesize

\url{http://angg.twu.net/eev-videos/2021-1-C2-somas-1-dicas.mp4}

\url{https://www.youtube.com/watch?v=pCD1p9FZYdI}

}

\msk

Video 2:

{\footnotesize

\url{http://angg.twu.net/eev-videos/2021-1-C2-somas-1-dicas-2.mp4}

\url{https://www.youtube.com/watch?v=Q0dxniyVZuA}

}





% \newpage




%\printbibliography







\GenericWarning{Success:}{Success!!!}  % Used by `M-x cv'

\end{document}

%  ____  _             _         
% |  _ \(_)_   ___   _(_)_______ 
% | | | | \ \ / / | | | |_  / _ \
% | |_| | |\ V /| |_| | |/ /  __/
% |____// | \_/  \__,_|_/___\___|
%     |__/                       
%
% «djvuize»  (to ".djvuize")
% (find-LATEXgrep "grep --color -nH --null -e djvuize 2020-1*.tex")

 (eepitch-shell)
 (eepitch-kill)
 (eepitch-shell)
# (find-fline "~/2021.1-C2/")
# (find-fline "~/LATEX/2021-1-C2/")
# (find-fline "~/bin/djvuize")

cd /tmp/
for i in *.jpg; do echo f $(basename $i .jpg); done

f () { rm -fv $1.png $1.pdf; djvuize $1.pdf }
f () { rm -fv $1.png $1.pdf; djvuize WHITEBOARDOPTS="-m 1.0" $1.pdf; xpdf $1.pdf }
f () { rm -fv $1.png $1.pdf; djvuize WHITEBOARDOPTS="-m 0.5" $1.pdf; xpdf $1.pdf }
f () { rm -fv $1.png $1.pdf; djvuize WHITEBOARDOPTS="-m 0.25" $1.pdf; xpdf $1.pdf }
f () { cp -fv $1.png $1.pdf       ~/2021.1-C2/
       cp -fv        $1.pdf ~/LATEX/2021-1-C2/
       cat <<%%%
% (find-latexscan-links "C2" "$1")
%%%
}

f 20201213_area_em_funcao_de_theta
f 20201213_area_em_funcao_de_x
f 20201213_area_fatias_pizza



%  __  __       _        
% |  \/  | __ _| | _____ 
% | |\/| |/ _` | |/ / _ \
% | |  | | (_| |   <  __/
% |_|  |_|\__,_|_|\_\___|
%                        
% <make>

 (eepitch-shell)
 (eepitch-kill)
 (eepitch-shell)
# (find-LATEXfile "2019planar-has-1.mk")
make -f 2019.mk STEM=2021-1-C2-somas-1-dicas veryclean
make -f 2019.mk STEM=2021-1-C2-somas-1-dicas pdf

% Local Variables:
% coding: utf-8-unix
% ee-tla: "c2sod"
% ee-tla: "c2m211somas1d"
% End:
