% (find-LATEX "2021-1-C3-planos.tex")
% (defun c () (interactive) (find-LATEXsh "lualatex -record 2021-1-C3-planos.tex" :end))
% (defun C () (interactive) (find-LATEXsh "lualatex 2021-1-C3-planos.tex" "Success!!!"))
% (defun D () (interactive) (find-pdf-page      "~/LATEX/2021-1-C3-planos.pdf"))
% (defun d () (interactive) (find-pdftools-page "~/LATEX/2021-1-C3-planos.pdf"))
% (defun e () (interactive) (find-LATEX "2021-1-C3-planos.tex"))
% (defun o () (interactive) (find-LATEX "2021-1-C3-planos.tex"))
% (defun u () (interactive) (find-latex-upload-links "2021-1-C3-planos"))
% (defun v () (interactive) (find-2a '(e) '(d)))
% (defun d0 () (interactive) (find-ebuffer "2021-1-C3-planos.pdf"))
% (defun cv () (interactive) (C) (ee-kill-this-buffer) (v) (g))
%          (code-eec-LATEX "2021-1-C3-planos")
% (find-pdf-page   "~/LATEX/2021-1-C3-planos.pdf")
% (find-sh0 "cp -v  ~/LATEX/2021-1-C3-planos.pdf /tmp/")
% (find-sh0 "cp -v  ~/LATEX/2021-1-C3-planos.pdf /tmp/pen/")
%     (find-xournalpp "/tmp/2021-1-C3-planos.pdf")
%   file:///home/edrx/LATEX/2021-1-C3-planos.pdf
%               file:///tmp/2021-1-C3-planos.pdf
%           file:///tmp/pen/2021-1-C3-planos.pdf
% http://angg.twu.net/LATEX/2021-1-C3-planos.pdf
% (find-LATEX "2019.mk")
% (find-CN-aula-links "2021-1-C3-planos" "3" "c3m211planos" "c3pl")
%
% Video:
% (find-ssr-links "c3m211planos" "2021-1-C3-planos" "MMpwfW16nxA")
% (code-video     "c3m211planosvideo" "$S/http/angg.twu.net/eev-videos/2021-1-C3-planos.mp4")
% (find-c3m211planosvideo "0:00")

% «.defs»	(to "defs")
% «.title»	(to "title")
%
% «.djvuize»	(to "djvuize")

\documentclass[oneside,12pt]{article}
\usepackage[colorlinks,citecolor=DarkRed,urlcolor=DarkRed]{hyperref} % (find-es "tex" "hyperref")
\usepackage{amsmath}
\usepackage{amsfonts}
\usepackage{amssymb}
\usepackage{pict2e}
\usepackage[x11names,svgnames]{xcolor} % (find-es "tex" "xcolor")
\usepackage{colorweb}                  % (find-es "tex" "colorweb")
%\usepackage{tikz}
%
% (find-dn6 "preamble6.lua" "preamble0")
%\usepackage{proof}   % For derivation trees ("%:" lines)
%\input diagxy        % For 2D diagrams ("%D" lines)
%\xyoption{curve}     % For the ".curve=" feature in 2D diagrams
%
\usepackage{edrx21}               % (find-LATEX "edrx21.sty")
\input edrxaccents.tex            % (find-LATEX "edrxaccents.tex")
\input edrxchars.tex              % (find-LATEX "edrxchars.tex")
\input edrxheadfoot.tex           % (find-LATEX "edrxheadfoot.tex")
\input edrxgac2.tex               % (find-LATEX "edrxgac2.tex")
%
%\usepackage[backend=biber,
%   style=alphabetic]{biblatex}            % (find-es "tex" "biber")
%\addbibresource{catsem-slides.bib}        % (find-LATEX "catsem-slides.bib")
%
% (find-es "tex" "geometry")
\usepackage[a6paper, landscape,
            top=1.5cm, bottom=.25cm, left=1cm, right=1cm, includefoot
           ]{geometry}
%
\begin{document}

%\catcode`\^^J=10
%\directlua{dofile "dednat6load.lua"}  % (find-LATEX "dednat6load.lua")

% %L dofile "edrxtikz.lua"  -- (find-LATEX "edrxtikz.lua")
% %L dofile "edrxpict.lua"  -- (find-LATEX "edrxpict.lua")
% \pu

% «defs»  (to ".defs")
% (find-LATEX "edrx15.sty" "colors-2019")
%\long\def\ColorRed   #1{{\color{Red1}#1}}
%\long\def\ColorViolet#1{{\color{MagentaVioletLight}#1}}
%\long\def\ColorViolet#1{{\color{Violet!50!black}#1}}
%\long\def\ColorGreen #1{{\color{SpringDarkHard}#1}}
%\long\def\ColorGreen #1{{\color{SpringGreenDark}#1}}
%\long\def\ColorGreen #1{{\color{SpringGreen4}#1}}
%\long\def\ColorGray  #1{{\color{GrayLight}#1}}
%\long\def\ColorGray  #1{{\color{black!30!white}#1}}
%\long\def\ColorBrown #1{{\color{Brown}#1}}
%\long\def\ColorBrown #1{{\color{brown}#1}}
%\long\def\ColorOrange#1{{\color{orange}#1}}
%
%\long\def\ColorShort #1{{\color{SpringGreen4}#1}}
%\long\def\ColorLong  #1{{\color{Red1}#1}}
%
%\def\frown{\ensuremath{{=}{(}}}
%\def\True {\mathbf{V}}
%\def\False{\mathbf{F}}
%\def\D    {\displaystyle}

\def\drafturl{http://angg.twu.net/LATEX/2021-1-C3.pdf}
\def\drafturl{http://angg.twu.net/2021.1-C3.html}
\def\draftfooter{\tiny \href{\drafturl}{\jobname{}} \ColorBrown{\shorttoday{} \hours}}



%  _____ _ _   _                               
% |_   _(_) |_| | ___   _ __   __ _  __ _  ___ 
%   | | | | __| |/ _ \ | '_ \ / _` |/ _` |/ _ \
%   | | | | |_| |  __/ | |_) | (_| | (_| |  __/
%   |_| |_|\__|_|\___| | .__/ \__,_|\__, |\___|
%                      |_|          |___/      
%
% «title»  (to ".title")
% (c3m211planosp 1 "title")
% (c3m211planosa   "title")

\thispagestyle{empty}

\begin{center}

\vspace*{1.2cm}

{\bf \Large Cálculo 3 - 2021.1}

\bsk

Aula ??: revisão de planos

\bsk

Eduardo Ochs - RCN/PURO/UFF

\url{http://angg.twu.net/2021.1-C3.html}

\end{center}

\newpage

Antes de voltar pra superfícies vamos rever algumas

coisas sobre planos que às vezes eram vistas em GA...

\msk

O material de hoje é uma adaptação disto aqui:

\footnotesize{

% http://angg.twu.net/LATEX/material-para-GA.pdf#page=45
% http://angg.twu.net/LATEX/material-para-GA.pdf#page=45
\url{http://angg.twu.net/LATEX/material-para-GA.pdf\#page=45}

\url{http://angg.twu.net/LATEX/material-para-GA.pdf\#page=46}

}

\newpage

{\bf Retas parametrizadas em $\R^3$}

% (mpgp 45 "R3-retas-e-planos")
% (mpga    "R3-retas-e-planos")

Sejam:

$r_1 = \setofexprt{(2,2,0)+t\VEC{0,-1,0}}$

$r_2 = \setofexprt{(2,2,1)+t\VEC{0,-1,0}}$

$r_3 = \setofexprt{(2,2,0)+t\VEC{0,1,1}}$

$r_4 = \setofexprt{(0,2,1)+t\VEC{1,0,0}}$

$r_4 = \setofexprt{(1,2,1)+t\VEC{2,0,0}}$

\msk

Quais destas retas se interceptam? Em que pontos? Em que `$t$'s?

Quais destas retas são paralelas? Quais destas retas são coincidentes?

A terminologia para retas que não se interceptam e não são

paralelas é estranha -- ``retas {\sl reversas}''.

\msk

As retas acima são {\sl parametrizadas}.

O que é uma {\sl equação de reta} em $\R^3$?

$\setofxyst{4x+5y=6}$ é uma reta em $\R^2$;

$\setofxyzst{4x+5y+6z=7}$ é um {\sl plano} em $\R^3$...


\newpage

Exercício: encontre

três pontos não colineares de $\setofxyzst{z=0}$,

três pontos não colineares de $\setofxyzst{z=2}$,

três pontos não colineares de $\setofxyzst{x=1}$,

três pontos não colineares de $\setofxyzst{y=3}$,

três pontos não colineares de $\setofxyzst{\frac x2 + \frac y3 + \frac z4 = 1}$,

e visualize cada um destes planos.

\msk

Alguns dos nossos planos preferidos:

$π_{xy} = \setofxyzst{z=0}$ ($x$ e $y$ variam, $z=0$)

$π_{xz} = \setofxyzst{y=0}$ ($x$ e $z$ variam, $y=0$)

$π_{yz} = \setofxyzst{x=0}$ ($y$ e $z$ variam, $x=0$)

\ssk

\newpage

{\bf Uma notação para planos}

\ssk

Notação (temporária):

$[\text{equação}] = \setofxyzst{\text{equação}}$

Obs: $π_{xy} = [z=0]$, $π_{xz} = [y=0]$, $π_{yz} = [x=0]$.

\msk

Exercício: visualize:

$π_1 = [x=1]$,     \qquad $π_8 = [y=x]$,     
                                      
$π_2 = [y=1]$,     \qquad $π_9 = [y=2x]$,    
                                      
$π_3 = [z=1]$,     \qquad $π_{10} = [z=x]$,  
                                      
$π_4 = [z=4]$,     \qquad $π_{11} = [z=x+1]$,

$π_5 = [z=2]$,

Quais deles planos são paralelos?

Quais deles planos se cortam? Onde?

\msk

Escolha dois planos destes que se cortam.

Você consegue dizer dois pontos da interseção deles?

Você consegue parametrizar a reta que é a interseção deles?

\msk

{\sl Faça isto para vários pares de planos que se cortam.}


\newpage

{\bf Planos parametrizados}

% (mpgp 46 "R3-retas-e-planos-2")
% (mpga    "R3-retas-e-planos-2")

\def\und#1#2{\underbrace{#1}_{#2}}

Dá pra parametrizar planos em $\R^3$...

Sejam

$π_6 = \setofst{\und{(2,2,0) + a\VEC{1,0,0} + b\VEC{0,1,0}}
                    {(a,b)_{Σ_6}}
                }{a,b∈\R}$,

$π_7 = \setofst{\und{(3,2,1) + a\VEC{1,0,0} + b\VEC{0,1,0}}
                    {(a,b)_{Σ_7}}
                }{a,b∈\R}$.

Calcule e visualize:

$(0,0)_{Σ_6}$, $(1,0)_{Σ_6}$, $(0,1)_{Σ_6}$, $(1,1)_{Σ_6}$,

$(0,0)_{Σ_7}$, $(1,0)_{Σ_7}$, $(0,1)_{Σ_7}$, $(1,1)_{Σ_7}$,

e resolva:

$(a,b)_{Σ_6} = (0,3,0)$,

$(a,b)_{Σ_7} = (2,4,1)$,

$(a,b)_{Σ_7} = (2,4,0)$.

\newpage

Nossos três modos preferidos de descrever planos em $\R^3$ (por equações) são:

$[z = ax+by+c]$ (``$z$ em função de $x$ e $y$''),

$[y = ax+bz+c]$ (``$y$ em função de $x$ e $z$''),

$[x = ay+bz+c]$ (``$x$ em função de $y$ e $z$'').



%\printbibliography

\GenericWarning{Success:}{Success!!!}  % Used by `M-x cv'

\end{document}

%  ____  _             _         
% |  _ \(_)_   ___   _(_)_______ 
% | | | | \ \ / / | | | |_  / _ \
% | |_| | |\ V /| |_| | |/ /  __/
% |____// | \_/  \__,_|_/___\___|
%     |__/                       
%
% «djvuize»  (to ".djvuize")
% (find-LATEXgrep "grep --color -nH --null -e djvuize 2020-1*.tex")

 (eepitch-shell)
 (eepitch-kill)
 (eepitch-shell)
# (find-fline "~/2021.1-C3/")
# (find-fline "~/LATEX/2021-1-C3/")
# (find-fline "~/bin/djvuize")

cd /tmp/
for i in *.jpg; do echo f $(basename $i .jpg); done

f () { rm -fv $1.png $1.pdf; djvuize $1.pdf }
f () { rm -fv $1.png $1.pdf; djvuize WHITEBOARDOPTS="-m 1.0" $1.pdf; xpdf $1.pdf }
f () { rm -fv $1.png $1.pdf; djvuize WHITEBOARDOPTS="-m 0.5" $1.pdf; xpdf $1.pdf }
f () { rm -fv $1.png $1.pdf; djvuize WHITEBOARDOPTS="-m 0.25" $1.pdf; xpdf $1.pdf }
f () { cp -fv $1.png $1.pdf       ~/2021.1-C3/
       cp -fv        $1.pdf ~/LATEX/2021-1-C3/
       cat <<%%%
% (find-latexscan-links "C3" "$1")
%%%
}

f 20201213_area_em_funcao_de_theta
f 20201213_area_em_funcao_de_x
f 20201213_area_fatias_pizza



%  __  __       _        
% |  \/  | __ _| | _____ 
% | |\/| |/ _` | |/ / _ \
% | |  | | (_| |   <  __/
% |_|  |_|\__,_|_|\_\___|
%                        
% <make>

 (eepitch-shell)
 (eepitch-kill)
 (eepitch-shell)
# (find-LATEXfile "2019planar-has-1.mk")
make -f 2019.mk STEM=2021-1-C3-planos veryclean
make -f 2019.mk STEM=2021-1-C3-planos pdf

% Local Variables:
% coding: utf-8-unix
% ee-tla: "c3pl"
% ee-tla: "c3m211planos"
% End:
