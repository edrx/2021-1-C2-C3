% (find-LATEX "2021-1-C2-MT2.tex")
% (defun c () (interactive) (find-LATEXsh "lualatex -record 2021-1-C2-MT2.tex" :end))
% (defun C () (interactive) (find-LATEXsh "lualatex 2021-1-C2-MT2.tex" "Success!!!"))
% (defun D () (interactive) (find-pdf-page      "~/LATEX/2021-1-C2-MT2.pdf"))
% (defun d () (interactive) (find-pdftools-page "~/LATEX/2021-1-C2-MT2.pdf"))
% (defun e () (interactive) (find-LATEX "2021-1-C2-MT2.tex"))
% (defun o () (interactive) (find-LATEX "2021-1-C2-MT1.tex"))
% (defun l () (interactive) (find-LATEX "2021-1-C2-critical-points.lua"))
% (defun u () (interactive) (find-latex-upload-links "2021-1-C2-MT2"))
% (defun v () (interactive) (find-2a '(e) '(d)))
% (defun d0 () (interactive) (find-ebuffer "2021-1-C2-MT2.pdf"))
% (defun cv () (interactive) (C) (ee-kill-this-buffer) (v) (g))
%          (code-eec-LATEX "2021-1-C2-MT2")
% (find-pdf-page   "~/LATEX/2021-1-C2-MT2.pdf")
% (find-sh0 "cp -v  ~/LATEX/2021-1-C2-MT2.pdf /tmp/")
% (find-sh0 "cp -v  ~/LATEX/2021-1-C2-MT2.pdf /tmp/pen/")
%     (find-xournalpp "/tmp/2021-1-C2-MT2.pdf")
%   file:///home/edrx/LATEX/2021-1-C2-MT2.pdf
%               file:///tmp/2021-1-C2-MT2.pdf
%           file:///tmp/pen/2021-1-C2-MT2.pdf
% http://angg.twu.net/LATEX/2021-1-C2-MT2.pdf
% (find-LATEX "2019.mk")
% (find-CN-aula-links "2021-1-C2-MT2" "2" "c2m211mt2" "c2mt2")
%
% Video (not yet):
% (find-ssr-links "c2m211mt2" "2021-1-C2-MT2" "{naoexiste}")
% (code-video     "c2m211mt2video" "$S/http/angg.twu.net/eev-videos/2021-1-C2-MT2.mp4")
% (find-c2m211mt2video "0:00")

% «.defs»		(to "defs")
% «.title»		(to "title")
% «.regras»		(to "regras")
% «.defs-funcao-f»	(to "defs-funcao-f")
% «.questoes-1-e-2»	(to "questoes-1-e-2")
% «.questao-3»		(to "questao-3")
% «.gabarito»		(to "gabarito")
%
% «.djvuize»		(to "djvuize")

\documentclass[oneside,12pt]{article}
\usepackage[colorlinks,citecolor=DarkRed,urlcolor=DarkRed]{hyperref} % (find-es "tex" "hyperref")
\usepackage{amsmath}
\usepackage{amsfonts}
\usepackage{amssymb}
\usepackage{pict2e}
\usepackage[x11names,svgnames]{xcolor} % (find-es "tex" "xcolor")
\usepackage{colorweb}                  % (find-es "tex" "colorweb")
%\usepackage{tikz}
%
% (find-dn6 "preamble6.lua" "preamble0")
%\usepackage{proof}   % For derivation trees ("%:" lines)
%\input diagxy        % For 2D diagrams ("%D" lines)
%\xyoption{curve}     % For the ".curve=" feature in 2D diagrams
%
\usepackage{edrx21}               % (find-LATEX "edrx21.sty")
\input edrxaccents.tex            % (find-LATEX "edrxaccents.tex")
\input edrx21chars.tex            % (find-LATEX "edrx21chars.tex")
\input edrxheadfoot.tex           % (find-LATEX "edrxheadfoot.tex")
\input edrxgac2.tex               % (find-LATEX "edrxgac2.tex")
%
%\usepackage[backend=biber,
%   style=alphabetic]{biblatex}            % (find-es "tex" "biber")
%\addbibresource{catsem-slides.bib}        % (find-LATEX "catsem-slides.bib")
%
% (find-es "tex" "geometry")
\usepackage[a6paper, landscape,
            top=1.5cm, bottom=.25cm, left=1cm, right=1cm, includefoot
           ]{geometry}
%
\begin{document}

\catcode`\^^J=10
\directlua{dofile "dednat6load.lua"}  % (find-LATEX "dednat6load.lua")

%L dofile "edrxtikz.lua"  -- (find-LATEX "edrxtikz.lua")
%L dofile "edrxpict.lua"  -- (find-LATEX "edrxpict.lua")
%L -- (find-LATEX "2021-1-C2-critical-points.lua" "Piecewisify-tests")
%L dofile         "2021-1-C2-critical-points.lua"
\pu

% «defs»  (to ".defs")
% (find-LATEX "edrx15.sty" "colors-2019")

\def\drafturl{http://angg.twu.net/LATEX/2021-1-C2.pdf}
\def\drafturl{http://angg.twu.net/2021.1-C2.html}
\def\draftfooter{\tiny \href{\drafturl}{\jobname{}} \ColorBrown{\shorttoday{} \hours}}



%  _____ _ _   _                               
% |_   _(_) |_| | ___   _ __   __ _  __ _  ___ 
%   | | | | __| |/ _ \ | '_ \ / _` |/ _` |/ _ \
%   | | | | |_| |  __/ | |_) | (_| | (_| |  __/
%   |_| |_|\__|_|\___| | .__/ \__,_|\__, |\___|
%                      |_|          |___/      
%
% «title»  (to ".title")
% (c2m211mt2p 1 "title")
% (c2m211mt2a   "title")

\thispagestyle{empty}

\begin{center}

\vspace*{1.2cm}

{\bf \Large Cálculo 2 - 2021.1}

\bsk

Mini-teste 2

\bsk

Eduardo Ochs - RCN/PURO/UFF

\url{http://angg.twu.net/2021.1-C2.html}

\end{center}

\newpage

% «regras»  (to ".regras")
% (c2m211mt2p 2 "regras")
% (c2m211mt2a   "regras")
% (c2m211mt1p 2 "regras")
% (c2m211mt1a   "regras")
% (c2m201mt1p 7 "miniteste-regras")
% (c2m201mt1    "miniteste-regras")

{\bf Regras para o mini-teste}

As questões do mini-teste serão disponibilizadas às 19:00 da
sexta-feira 13/agosto/2021 e você deverá entregar as respostas
\ColorRed{escritas à mão} até as 19:00 do sábado 14/agosto/2021 na
plataforma Classroom; dese\-nhos feitos no computador serão
\ColorRed{ignorados}.

Se o Classroom der algum problema mande também para este endereço de
e-mail:

\ssk

\ColorRed{eduardoochs@gmail.com}

\ssk

Mini-testes entregues após este horário não serão considerados.

Durante as 24 horas do mini-teste nem o professor nem o monitor
responderão perguntas sobre os assuntos do mini-teste mas você pode
discutir com os seus colegas --- inclusive no grupo da turma.

Este mini-teste vale 0.5 pontos extras na P1.

\newpage

{\bf Regras para o mini-teste (2)}

\msk

Pra entender o que eu espero das respostas

de vocês releia a ``Dica 7'' daqui,

{\footnotesize

% (c2m211somas1dp 7 "dica-7")
% (c2m211somas1da   "dica-7")
\url{http://angg.twu.net/LATEX/2021-1-C2-somas-1-dicas.pdf#page=7}
}

\msk

o comentário sobre adivinhar o contexto daqui,

{\footnotesize

% (c2m211somas24p 12 "contexto")
% (c2m211somas24a    "contexto")
\url{http://angg.twu.net/LATEX/2021-1-C2-somas-2-4.pdf#page=12}

}

\msk

e o comentário sobre reler e revisar muitas vezes daqui:

{\footnotesize

% (c2m201p1p 10 "comentario-telegram")
% (c2m201p1a    "comentario-telegram")
\url{http://angg.twu.net/LATEX/2020-1-C2-P1.pdf#page=10}

}


\newpage


% «defs-funcao-f»  (to ".defs-funcao-f")
% (c2m211mt2p 4 "defs-funcao-f")
% (c2m211mt2a   "defs-funcao-f")
%
%L f_questao_1 = function (x)
%L     if x < 1 then return 1 end
%L     if x < 3 then return 2 end
%L     if x < 4 then return 1 end
%L     if x < 5 then return 0 end
%L     if x < 6 then return -1 end
%L     if x < 8 then return -2 end
%L     if x < 9 then return -1 end
%L     return 0
%L   end
%L
%L pwi = Piecewisify {f = f_questao_1}
%L pwi:setpoints(2, 8, seq(0, 10, 1))
%L pwirects = function (n, method)
%L     return pwi:rects(Partition.new(1, 9):splitn(n), method)
%L   end
\pu

\def\xonetoten{%
    \celllower=2.5pt%
    \def\cellfont{\scriptsize}%
    \put(1,-0.5){\cell{1}}%
    \put(2,-0.5){\cell{2}}%
    \put(3,-0.5){\cell{3}}%
    \put(4,-0.5){\cell{4}}%
    \put(5,-0.5){\cell{5}}%
    \put(6,-0.5){\cell{6}}%
    \put(7,-0.5){\cell{7}}%
    \put(8,-0.5){\cell{8}}%
    \put(9,-0.5){\cell{9}}%
    \put(10,-0.5){\cell{10}}%
  }

\def\fwithapprs#1{%
  \vcenter{\hbox{%
    \beginpicture(0,-3)(10,3)
    \pictgrid%
    #1%
    \pictaxes%
    \expr{pwi:pw(0, 10)}
    \xonetoten
    \end{picture}%
  }}}
\def\fwithapprscc#1#2{
  \fwithapprs{%
    \ColorUpper{\expr{#1}}%
    \ColorLower{\expr{#2}}%
  }}
\def\fwithapprsccc#1#2#3{%
  \fwithapprscc{pwirects(#1, "#2")}{pwirects(#1, "#3")}%
  }

\def\Fwithapprs#1{%
  \vcenter{\hbox{%
    \beginpicture(0,-3)(10,3)
    \pictgrid%
    #1%
    \pictaxes%
    \expr{pwiF:pw(0, 10)}
    \xonetoten
    \end{picture}%
  }}}

\long\def\ColorUpper #1{{\color{Gold}#1}}
\long\def\ColorLower #1{{\color{Orange!75!red}#1}}

\def\pictgrid{{\color{GrayPale}\linethickness{0.3pt}\expr{pictgrid()}}}
\def\pictgrid{{\color{GrayPale!75!white}\linethickness{0.2pt}\expr{pictgrid()}}}

\msk

\unitlength=15pt


% «questoes-1-e-2»  (to ".questoes-1-e-2")
% (c2m211mt2p 4 "questoes-1-e-2")
% (c2m211mt2a   "questoes-1-e-2")

Sejam:
%
$$f(x) \;\;=\;\; \fwithapprs{}
$$

e $F(x) \;\;=\;\; \D \Intt{2}{x}{f(t)}$.

\bsk

1) (0.1 pts) Represente graficamente
$F(3.5)$,
$F(6.5)$
e $F(6.5) - F(3.5)$.

2) (0.1 pts) Represente $\Intx{3.5}{6.5}{f(x)}$ como uma soma de retângulos

e calcule o resultado desta soma. Dica: veja o slide 29 daqui...

\ssk

{\scriptsize

% (c2m211prp 29 "integrando-escadas")
% (c2m211pra    "integrando-escadas")
%    http://angg.twu.net/LATEX/2021-1-C2-propriedades-da-integral.pdf#page=29
\url{http://angg.twu.net/LATEX/2021-1-C2-propriedades-da-integral.pdf#page=29}

}


\newpage

% «questao-3»  (to ".questao-3")
% (c2m211mt2p 5 "questao-3")
% (c2m211mt2a   "questao-3")

3) (0.3 pts) Desenhe o gráfico da função $F(x)$ entre $x=0$ e $x=10$.

Dica: faça primeiro o trecho entre $x=2$ e $x=10$ e depois descubra

como fazer o trecho entre $x=0$ e $x=2$.


\newpage

% «gabarito»  (to ".gabarito")
% (c2m211mt2p 6 "gabarito")
% (c2m211mt2a   "gabarito")

{\bf Gabarito: questão 1}

% (find-LATEXfile "2021-1-C2-propriedades-da-integral.tex" "pwi:rects")

\def\GabOne#1#2{%
   \scalebox{0.7}{$
     \fwithapprs{%
       \ColorOrange{\expr{pwi:pol(#1, #2, "*")}}%
     }
   $}}

1)
$\begin{array}[t]{rcl}
            F(3.5) &=& \GabOne{2.0}{3.5} \\
            F(6.5) &=& \GabOne{2.0}{6.5} \\
   F(6.5) - F(3.5) &=& \GabOne{3.5}{6.5} \\
 \end{array}
$

\newpage

{\bf Gabarito: questões 2 e 3}

\bsk

3) $F(6.5) - F(3.5) =
    \pmat{
          1·(4-3.5)    \\
      +\; 0·(5-4)      \\
      +\; (-1)·(6-5)   \\
      +\; (-2)·(6.5-6) \\
    }
    =
    \pmat{ 0.5 \\ +\; 0 \\ +\; (-1) \\ +\; (-1) }
    = -1.5
   $

\bsk

%L F_questao_1 = function (x)
%L     local g = function (x0,y0, x1,y1, x)
%L         return y0 + (x-x0)*((y1-y0)/(x1-x0))
%L       end
%L     if x < 1 then return x-3 end
%L     if x < 3 then return g(1,-2, 3,2, x) end
%L     if x < 4 then return g(3,2,  4,3, x) end
%L     if x < 5 then return g(4,3,  5,3, x) end
%L     if x < 6 then return g(5,3,  6,2, x) end
%L     if x < 8 then return g(6,2,  8,-2, x) end
%L     if x < 9 then return g(8,-2, 9,-3, x) end
%L     return -3
%L   end
%L
%L pwiF = Piecewisify {f = F_questao_1}
%L pwiF:setpoints(2, 8, seq(0, 10, 1))
\pu

3)
$F(x) \;\;=\;\; \Fwithapprs{}
$



%\printbibliography

\GenericWarning{Success:}{Success!!!}  % Used by `M-x cv'

\end{document}

%  ____  _             _         
% |  _ \(_)_   ___   _(_)_______ 
% | | | | \ \ / / | | | |_  / _ \
% | |_| | |\ V /| |_| | |/ /  __/
% |____// | \_/  \__,_|_/___\___|
%     |__/                       
%
% «djvuize»  (to ".djvuize")
% (find-LATEXgrep "grep --color -nH --null -e djvuize 2020-1*.tex")

 (eepitch-shell)
 (eepitch-kill)
 (eepitch-shell)
# (find-fline "~/2021.1-C2/")
# (find-fline "~/LATEX/2021-1-C2/")
# (find-fline "~/bin/djvuize")

cd /tmp/
for i in *.jpg; do echo f $(basename $i .jpg); done

f () { rm -fv $1.png $1.pdf; djvuize $1.pdf }
f () { rm -fv $1.png $1.pdf; djvuize WHITEBOARDOPTS="-m 1.0" $1.pdf; xpdf $1.pdf }
f () { rm -fv $1.png $1.pdf; djvuize WHITEBOARDOPTS="-m 0.5" $1.pdf; xpdf $1.pdf }
f () { rm -fv $1.png $1.pdf; djvuize WHITEBOARDOPTS="-m 0.25" $1.pdf; xpdf $1.pdf }
f () { cp -fv $1.png $1.pdf       ~/2021.1-C2/
       cp -fv        $1.pdf ~/LATEX/2021-1-C2/
       cat <<%%%
% (find-latexscan-links "C2" "$1")
%%%
}

f 20201213_area_em_funcao_de_theta
f 20201213_area_em_funcao_de_x
f 20201213_area_fatias_pizza



%  __  __       _        
% |  \/  | __ _| | _____ 
% | |\/| |/ _` | |/ / _ \
% | |  | | (_| |   <  __/
% |_|  |_|\__,_|_|\_\___|
%                        
% <make>

 (eepitch-shell)
 (eepitch-kill)
 (eepitch-shell)
# (find-LATEXfile "2019planar-has-1.mk")
make -f 2019.mk STEM=2021-1-C2-MT2 veryclean
make -f 2019.mk STEM=2021-1-C2-MT2 pdf

% Local Variables:
% coding: utf-8-unix
% ee-tla: "c2mt2"
% ee-tla: "c2m211mt2"
% End:
