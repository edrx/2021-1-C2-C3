% (find-LATEX "2021-1-C2-MT1.tex")
% (defun c () (interactive) (find-LATEXsh "lualatex -record 2021-1-C2-MT1.tex" :end))
% (defun C () (interactive) (find-LATEXsh "lualatex 2021-1-C2-MT1.tex" "Success!!!"))
% (defun D () (interactive) (find-pdf-page      "~/LATEX/2021-1-C2-MT1.pdf"))
% (defun d () (interactive) (find-pdftools-page "~/LATEX/2021-1-C2-MT1.pdf"))
% (defun e () (interactive) (find-LATEX "2021-1-C2-MT1.tex"))
% (defun o () (interactive) (find-LATEX "2021-1-C2-propriedades-da-integral.tex"))
% (defun l () (interactive) (find-LATEX "2021-1-C2-critical-points.lua"))
% (defun u () (interactive) (find-latex-upload-links "2021-1-C2-MT1"))
% (defun v () (interactive) (find-2a '(e) '(d)))
% (defun d0 () (interactive) (find-ebuffer "2021-1-C2-MT1.pdf"))
% (defun cv () (interactive) (C) (ee-kill-this-buffer) (v) (g))
%          (code-eec-LATEX "2021-1-C2-MT1")
% (find-pdf-page   "~/LATEX/2021-1-C2-MT1.pdf")
% (find-sh0 "cp -v  ~/LATEX/2021-1-C2-MT1.pdf /tmp/")
% (find-sh0 "cp -v  ~/LATEX/2021-1-C2-MT1.pdf /tmp/pen/")
%     (find-xournalpp "/tmp/2021-1-C2-MT1.pdf")
%   file:///home/edrx/LATEX/2021-1-C2-MT1.pdf
%               file:///tmp/2021-1-C2-MT1.pdf
%           file:///tmp/pen/2021-1-C2-MT1.pdf
% http://angg.twu.net/LATEX/2021-1-C2-MT1.pdf
% (find-LATEX "2019.mk")
% (find-CN-aula-links "2021-1-C2-MT1" "2" "c2m211mt1" "c2mt1")
%
% Video (not yet):
% (find-ssr-links "c2m211mt1" "2021-1-C2-MT1" "{naoexiste}")
% (code-video     "c2m211mt1video" "$S/http/angg.twu.net/eev-videos/2021-1-C2-MT1.mp4")
% (find-c2m211mt1video "0:00")

% «.defs»	(to "defs")
% «.title»	(to "title")
% «.regras»	(to "regras")
% «.questao-1»	(to "questao-1")
% «.questao-2»	(to "questao-2")
% «.gabarito»	(to "gabarito")
%
% «.djvuize»	(to "djvuize")

\documentclass[oneside,12pt]{article}
\usepackage[colorlinks,citecolor=DarkRed,urlcolor=DarkRed]{hyperref} % (find-es "tex" "hyperref")
\usepackage{amsmath}
\usepackage{amsfonts}
\usepackage{amssymb}
\usepackage{pict2e}
\usepackage[x11names,svgnames]{xcolor} % (find-es "tex" "xcolor")
\usepackage{colorweb}                  % (find-es "tex" "colorweb")
%\usepackage{tikz}
%
% (find-dn6 "preamble6.lua" "preamble0")
%\usepackage{proof}   % For derivation trees ("%:" lines)
%\input diagxy        % For 2D diagrams ("%D" lines)
%\xyoption{curve}     % For the ".curve=" feature in 2D diagrams
%
\usepackage{edrx21}               % (find-LATEX "edrx21.sty")
\input edrxaccents.tex            % (find-LATEX "edrxaccents.tex")
\input edrx21chars.tex            % (find-LATEX "edrx21chars.tex")
\input edrxheadfoot.tex           % (find-LATEX "edrxheadfoot.tex")
\input edrxgac2.tex               % (find-LATEX "edrxgac2.tex")
%
%\usepackage[backend=biber,
%   style=alphabetic]{biblatex}            % (find-es "tex" "biber")
%\addbibresource{catsem-slides.bib}        % (find-LATEX "catsem-slides.bib")
%
% (find-es "tex" "geometry")
\usepackage[a6paper, landscape,
            top=1.5cm, bottom=.25cm, left=1cm, right=1cm, includefoot
           ]{geometry}
%
\begin{document}

\catcode`\^^J=10
\directlua{dofile "dednat6load.lua"}  % (find-LATEX "dednat6load.lua")

%L dofile "edrxtikz.lua"  -- (find-LATEX "edrxtikz.lua")
%L dofile "edrxpict.lua"  -- (find-LATEX "edrxpict.lua")
%L dofile "2021-1-C2-critical-points.lua"
%L -- (find-LATEX "2021-1-C2-critical-points.lua" "f_parabola_preferida")
%L -- (find-LATEX "2021-1-C2-critical-points.lua" "Piecewisify-tests")
\pu

% «defs»  (to ".defs")
% (find-LATEX "edrx15.sty" "colors-2019")

\def\mname#1{\text{[#1]}}

\def\drafturl{http://angg.twu.net/LATEX/2021-1-C2.pdf}
\def\drafturl{http://angg.twu.net/2021.1-C2.html}
\def\draftfooter{\tiny \href{\drafturl}{\jobname{}} \ColorBrown{\shorttoday{} \hours}}



%  _____ _ _   _                               
% |_   _(_) |_| | ___   _ __   __ _  __ _  ___ 
%   | | | | __| |/ _ \ | '_ \ / _` |/ _` |/ _ \
%   | | | | |_| |  __/ | |_) | (_| | (_| |  __/
%   |_| |_|\__|_|\___| | .__/ \__,_|\__, |\___|
%                      |_|          |___/      
%
% «title»  (to ".title")
% (c2m211mt1p 1 "title")
% (c2m211mt1a   "title")

\thispagestyle{empty}

\begin{center}

\vspace*{1.2cm}

{\bf \Large Cálculo 2 - 2021.1}

\bsk

Mini-teste 1

\bsk

Eduardo Ochs - RCN/PURO/UFF

\url{http://angg.twu.net/2021.1-C2.html}

\end{center}

\newpage

% «regras»  (to ".regras")
% (c2m211mt1p 2 "regras")
% (c2m211mt1a   "regras")
% (c2m201mt1p 7 "miniteste-regras")
% (c2m201mt1    "miniteste-regras")

{\bf Regras para o mini-teste}

As questões do mini-teste serão disponibilizadas às 18:00 da
sexta-feira 23/julho/2021 e você deverá entregar as respostas
\ColorRed{escritas à mão} até as 18:00 do sábado 14/julho/2021 na
plataforma Classroom; dese\-nhos feitos no computador serão
\ColorRed{ignorados}.

Se o Classroom der algum problema mande também para este endereço de
e-mail:

\ssk

\ColorRed{eduardoochs@gmail.com}

\ssk

Mini-testes entregues após este horário não serão considerados.

Durante as 24 horas do mini-teste nem o professor nem o monitor
responderão perguntas sobre os assuntos do mini-teste mas você pode
discutir com os seus colegas --- inclusive no grupo da turma.

Este mini-teste vale 0.5 pontos extras na P1.

\ssk

(Aliás, vale 0.7...)

\newpage

{\bf Regras para o mini-teste (2)}

\msk

Pra entender o que eu espero das respostas

de vocês releia a ``Dica 7'' daqui,

{\footnotesize

% (c2m211somas1dp 7 "dica-7")
% (c2m211somas1da   "dica-7")
\url{http://angg.twu.net/LATEX/2021-1-C2-somas-1-dicas.pdf#page=7}
}

\msk

o comentário sobre adivinhar o contexto daqui,

{\footnotesize

% (c2m211somas24p 12 "contexto")
% (c2m211somas24a    "contexto")
\url{http://angg.twu.net/LATEX/2021-1-C2-somas-2-4.pdf#page=12}

}

\msk

e o comentário sobre reler e revisar muitas vezes daqui:

{\footnotesize

% (c2m201p1p 10 "comentario-telegram")
% (c2m201p1a    "comentario-telegram")
\url{http://angg.twu.net/LATEX/2020-1-C2-P1.pdf#page=10}

}


\newpage

% «questao-1»  (to ".questao-1")
% (c2m211mt1p 4 "questao-1")
% (c2m211mt1a   "questao-1")

{\bf Questão 1}

\unitlength=12pt

Seja:
%
$$f(x) =
  \vcenter{\hbox{%
    \beginpicture(0,0)(10,5)
    \pictgrid%
    %
    \pictpiecewise{(0,3)--(2,1)--(8,4)--(10,0)}%
    \pictaxes%
    \end{picture}%
  }}
$$

\msk

Em cada um dos casos abaixo represente num gráfico só

a função $f$ e os dois somatórios pedidos.

\msk

a) (0.1 pts) $\mname{sup}_{[1,9]_{2^1}}$, $\mname{inf}_{[1,9]_{2^1}}$

b) (0.1 pts) $\mname{sup}_{[1,9]_{2^2}}$, $\mname{inf}_{[1,9]_{2^2}}$

c) (0.1 pts) $\mname{sup}_{[1,9]_{2^3}}$, $\mname{inf}_{[1,9]_{2^3}}$

d) (0.1 pts) $\mname{max}_{[1,9]_{2^1}}$, $\mname{min}_{[1,9]_{2^1}}$

e) (0.1 pts) $\mname{max}_{[1,9]_{2^2}}$, $\mname{min}_{[1,9]_{2^2}}$



\newpage

% «questao-2»  (to ".questao-2")
% (c2m211mt1p 5 "questao-2")
% (c2m211mt1a   "questao-2")

{\bf Questão 2 (bonus)}

(0.2 pts) Dê uma definição por casos

pra função $f$ da página anterior.



\bsk
\bsk
\bsk
\bsk

Dica: lembre que toda a notação que estamos

usando vem daqui, das páginas 27 em diante...

\ssk

{\footnotesize

% (c2m211somas2p 27 "metodos-nomes")
% (c2m211somas2a    "metodos-nomes")
\url{http://angg.twu.net/LATEX/2021-1-C2-somas-2.pdf#page=27}

}


\newpage

% «gabarito»  (to ".gabarito")
% (c2m211mt1p 6 "gabarito")
% (c2m211mt1a   "gabarito")
% (find-LATEX "2021-1-C2-critical-points.lua" "Piecewisify-tests")
% (c2m211somas2p 37 "exercicio-16-fig1")
% (c2m211somas2a    "exercicio-16-fig1")

% (c2m211somas2p 37 "exercicio-16-defs")
% (c2m211somas2a    "exercicio-16-defs")
% (c2m211somas2a    "exercicio-16-defs" "ColorUpperA")
\long\def\ColorUpperA#1{{\color{red!20!white}#1}}
\long\def\ColorUpperB#1{{\color{Gold1!20!white}#1}}
\long\def\ColorUpperC#1{{\color{Green1!20!white}#1}}
\long\def\ColorUpperD#1{{\color{Blue1!20!white}#1}}
\long\def\ColorLowerA#1{{\color{red!80!white}#1}}
\long\def\ColorLowerB#1{{\color{Gold1!80!white}#1}}
\long\def\ColorLowerC#1{{\color{Green1!80!white}#1}}
\long\def\ColorLowerD#1{{\color{Blue1!80!white}#1}}
\long\def\ColorUpper #1{{\color{Gold}#1}}
\long\def\ColorLower #1{{\color{Orange}#1}}
\long\def\ColorLower #1{{\color{Orange!75!red}#1}}

%L f_questao_1 = function (x)
%L     if x < 2 then return 3-x end
%L     if x < 8 then return x/2 end
%L     return 20 - 2*x
%L   end
%L
%L pwi = Piecewisify {f = f_questao_1}
%L pwi:setpoints(2, 8, seq(0, 10, 1))
%L pwirects = function (n, method)
%L     return pwi:rects(Partition.new(1, 9):splitn(n), method)
%L   end
\pu

\def\fwithapprs#1{%
  \vcenter{\hbox{%
    \beginpicture(0,0)(10,5)
    \pictgrid%
    #1%
    \pictaxes%
    \expr{pwi:pw(0, 10)}
    \end{picture}%
  }}}
\def\fwithapprscc#1#2{
  \fwithapprs{%
    \ColorUpper{\expr{#1}}%
    \ColorLower{\expr{#2}}%
  }}
\def\fwithapprsccc#1#2#3{%
  \fwithapprscc{pwirects(#1, "#2")}{pwirects(#1, "#3")}%
  }


{\bf Gabarito}

\msk

\unitlength=8pt

$\text{1a) } \fwithapprsccc{2}{sup}{inf}
 \quad
 \text{1b) } \fwithapprsccc{4}{sup}{inf}
 \quad
 \text{1c) } \fwithapprsccc{8}{sup}{inf}
$

\msk

$\text{1d) } \fwithapprsccc{2}{max}{min}
 \quad
 \text{1e) } \fwithapprsccc{4}{max}{min}
$

\bsk

$\text{2) }
  f(x)
  \;\; = \;\;
  \begin{cases}
    3-x & \text{quando $x≤2$}, \\
    x/2 & \text{quando $2<x≤8$}, \\
    20-2x & \text{quando $8<x$} \\
  \end{cases}
$

\newpage

{\bf Alguns erros que muitas pessoas cometeram}

\msk

``$\{1,5,9\}=[1,5]∪[5,9]$'': Isto é falso!!! 

$\{1,5,9\}$ é um conjunto de três elementos,

e $[1,5]∪[5,9]=[1,9]$ é um conjunto infinito!...

\newpage

{\bf Alguns erros que muitas pessoas cometeram (2)}

\bsk

\def\eqfrown{\overset{\ColorRed{\text{(não!!!)}}}{=}}
\def\NEQ{\ColorRed{≠}}

$f(x)
  \;\; = \;\;
  \vcenter{\hbox{%
    \beginpicture(0,0)(10,5)
    \pictgrid%
    %
    \pictpiecewise{(0,3)--(2,1)--(8,4)--(10,0)}%
    \pictaxes%
    \end{picture}%
  }}
  \;\; \eqfrown \;\;
  \begin{cases}
    3-x & \text{quando $x≤2$}, \\
    6-x & \text{quando $2<x≤8$}, \\
    -2x-20 & \text{quando $8<x$} \\
  \end{cases}
$

\bsk

Vocês deveriam ser capazes de testar esse ``$6-x$''

e esse ``$-2x-20$'' de cabeça em poucos segundos,

fazendo algo como isto aqui:
%
$$\begin{array}{rcccccccc}
  (6-x)[x:=2]     &=& 6-2      &=&   4 &\NEQ& 1 &=& f(2) \\
  (6-x)[x:=8]     &=& 6-8      &=&  -2 &\NEQ& 4 &=& f(8)  \\
  (-2x-20)[x:=8]  &=& -2·8-20  &=& -36 &\NEQ& 4 &=& f(8)  \\
  (-2x-20)[x:=10] &=& -2·10-20 &=& -40 &\NEQ& 0 &=& f(10) \\
  \end{array}
$$



%\printbibliography

\GenericWarning{Success:}{Success!!!}  % Used by `M-x cv'

\end{document}

%  ____  _             _         
% |  _ \(_)_   ___   _(_)_______ 
% | | | | \ \ / / | | | |_  / _ \
% | |_| | |\ V /| |_| | |/ /  __/
% |____// | \_/  \__,_|_/___\___|
%     |__/                       
%
% «djvuize»  (to ".djvuize")
% (find-LATEXgrep "grep --color -nH --null -e djvuize 2020-1*.tex")

 (eepitch-shell)
 (eepitch-kill)
 (eepitch-shell)
# (find-fline "~/2021.1-C2/")
# (find-fline "~/LATEX/2021-1-C2/")
# (find-fline "~/bin/djvuize")

cd /tmp/
for i in *.jpg; do echo f $(basename $i .jpg); done

f () { rm -fv $1.png $1.pdf; djvuize $1.pdf }
f () { rm -fv $1.png $1.pdf; djvuize WHITEBOARDOPTS="-m 1.0" $1.pdf; xpdf $1.pdf }
f () { rm -fv $1.png $1.pdf; djvuize WHITEBOARDOPTS="-m 0.5" $1.pdf; xpdf $1.pdf }
f () { rm -fv $1.png $1.pdf; djvuize WHITEBOARDOPTS="-m 0.25" $1.pdf; xpdf $1.pdf }
f () { cp -fv $1.png $1.pdf       ~/2021.1-C2/
       cp -fv        $1.pdf ~/LATEX/2021-1-C2/
       cat <<%%%
% (find-latexscan-links "C2" "$1")
%%%
}

f 20201213_area_em_funcao_de_theta
f 20201213_area_em_funcao_de_x
f 20201213_area_fatias_pizza



%  __  __       _        
% |  \/  | __ _| | _____ 
% | |\/| |/ _` | |/ / _ \
% | |  | | (_| |   <  __/
% |_|  |_|\__,_|_|\_\___|
%                        
% <make>

 (eepitch-shell)
 (eepitch-kill)
 (eepitch-shell)
# (find-LATEXfile "2019planar-has-1.mk")
make -f 2019.mk STEM=2021-1-C2-MT1 veryclean
make -f 2019.mk STEM=2021-1-C2-MT1 pdf

% Local Variables:
% coding: utf-8-unix
% ee-tla: "c2mt1"
% ee-tla: "c2m211mt1"
% End:
