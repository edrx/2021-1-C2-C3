% (find-LATEX "2021-1-C2-subst.tex")
% (defun c () (interactive) (find-LATEXsh "lualatex -record 2021-1-C2-subst.tex" :end))
% (defun C () (interactive) (find-LATEXsh "lualatex 2021-1-C2-subst.tex" "Success!!!"))
% (defun D () (interactive) (find-pdf-page      "~/LATEX/2021-1-C2-subst.pdf"))
% (defun d () (interactive) (find-pdftools-page "~/LATEX/2021-1-C2-subst.pdf"))
% (defun e () (interactive) (find-LATEX "2021-1-C2-subst.tex"))
% (defun o () (interactive) (find-LATEX "2021-1-C2-subst.tex"))
% (defun u () (interactive) (find-latex-upload-links "2021-1-C2-subst"))
% (defun v () (interactive) (find-2a '(e) '(d)))
% (defun d0 () (interactive) (find-ebuffer "2021-1-C2-subst.pdf"))
% (defun cv () (interactive) (C) (ee-kill-this-buffer) (v) (g))
%          (code-eec-LATEX "2021-1-C2-subst")
% (find-pdf-page   "~/LATEX/2021-1-C2-subst.pdf")
% (find-sh0 "cp -v  ~/LATEX/2021-1-C2-subst.pdf /tmp/")
% (find-sh0 "cp -v  ~/LATEX/2021-1-C2-subst.pdf /tmp/pen/")
%     (find-xournalpp "/tmp/2021-1-C2-subst.pdf")
%   file:///home/edrx/LATEX/2021-1-C2-subst.pdf
%               file:///tmp/2021-1-C2-subst.pdf
%           file:///tmp/pen/2021-1-C2-subst.pdf
% http://angg.twu.net/LATEX/2021-1-C2-subst.pdf
% (find-LATEX "2019.mk")
% (find-CN-aula-links "2021-1-C2-subst" "2" "c2m211subst" "c2s")
%
% Video:
% (find-ssr-links "c2m211subst" "2021-1-C2-subst")
% (code-video     "c2m211substvideo" "$S/http/angg.twu.net/eev-videos/2021-1-C2-subst.mp4")
% (find-c2m211substvideo "0:00")

% «.defs»			(to "defs")
% «.title»			(to "title")
% «.corrigir-igual»		(to "corrigir-igual")
% «.subst-box»			(to "subst-box")
% «.subst-zoomed»		(to "subst-zoomed")
% «.testar-hipoteses»		(to "testar-hipoteses")
% «.regra-da-cadeia»		(to "regra-da-cadeia")
% «.regra-da-cadeia-2»		(to "regra-da-cadeia-2")
% «.acrescentamos»		(to "acrescentamos")
% «.dicas-subst»		(to "dicas-subst")
% «.igual-depois-de-subst»	(to "igual-depois-de-subst")
% «.dicas-subst-2»		(to "dicas-subst-2")
% «.dicas-subst-3»		(to "dicas-subst-3")
% «.dicas-subst-5»		(to "dicas-subst-5")
% «.dicas-subst-6»		(to "dicas-subst-6")
% «.exercicios-1-e-2»		(to "exercicios-1-e-2")
% «.somatorios»			(to "somatorios")
% «.soma-PG»			(to "soma-PG")
% «.somatorio-expansao»		(to "somatorio-expansao")
% «.somatorios-exercs»		(to "somatorios-exercs")
% «.para-todo-e-existe»		(to "para-todo-e-existe")
% «.visualizando-fas-e-exs»	(to "visualizando-fas-e-exs")
% «.visualizando-fas-e-exs-2»	(to "visualizando-fas-e-exs-2")
% «.visualizando-fas-e-exs-3»	(to "visualizando-fas-e-exs-3")
%
% «.djvuize»			(to "djvuize")

\documentclass[oneside,12pt]{article}
\usepackage[colorlinks,citecolor=DarkRed,urlcolor=DarkRed]{hyperref} % (find-es "tex" "hyperref")
\usepackage{amsmath}
\usepackage{amsfonts}
\usepackage{amssymb}
\usepackage{pict2e}
\usepackage[x11names,svgnames]{xcolor} % (find-es "tex" "xcolor")
\usepackage{colorweb}                  % (find-es "tex" "colorweb")
%\usepackage{tikz}
%
% (find-dn6 "preamble6.lua" "preamble0")
%\usepackage{proof}   % For derivation trees ("%:" lines)
%\input diagxy        % For 2D diagrams ("%D" lines)
%\xyoption{curve}     % For the ".curve=" feature in 2D diagrams
%
\usepackage{edrx15}               % (find-LATEX "edrx15.sty")
\input edrxaccents.tex            % (find-LATEX "edrxaccents.tex")
\input edrxchars.tex              % (find-LATEX "edrxchars.tex")
\input edrxheadfoot.tex           % (find-LATEX "edrxheadfoot.tex")
\input edrxgac2.tex               % (find-LATEX "edrxgac2.tex")
%
%\usepackage[backend=biber,
%   style=alphabetic]{biblatex}            % (find-es "tex" "biber")
%\addbibresource{catsem-slides.bib}        % (find-LATEX "catsem-slides.bib")
%
% (find-es "tex" "geometry")
\usepackage[a6paper, landscape,
            top=1.5cm, bottom=.25cm, left=1cm, right=1cm, includefoot
           ]{geometry}
%
\begin{document}

%\catcode`\^^J=10
%\directlua{dofile "dednat6load.lua"}  % (find-LATEX "dednat6load.lua")

% %L dofile "edrxtikz.lua"  -- (find-LATEX "edrxtikz.lua")
% %L dofile "edrxpict.lua"  -- (find-LATEX "edrxpict.lua")
% \pu

% «defs»  (to ".defs")
% (find-LATEX "edrx15.sty" "colors-2019")
\long\def\ColorRed   #1{{\color{Red1}#1}}
\long\def\ColorViolet#1{{\color{MagentaVioletLight}#1}}
\long\def\ColorViolet#1{{\color{Violet!50!black}#1}}
\long\def\ColorGreen #1{{\color{SpringDarkHard}#1}}
\long\def\ColorGreen #1{{\color{SpringGreenDark}#1}}
\long\def\ColorGreen #1{{\color{SpringGreen4}#1}}
\long\def\ColorGray  #1{{\color{GrayLight}#1}}
\long\def\ColorGray  #1{{\color{black!30!white}#1}}
\long\def\ColorBrown #1{{\color{Brown}#1}}
\long\def\ColorBrown #1{{\color{brown}#1}}
\long\def\ColorOrange#1{{\color{orange}#1}}

\long\def\ColorShort #1{{\color{SpringGreen4}#1}}
\long\def\ColorLong  #1{{\color{Red1}#1}}

\def\frown{\ensuremath{{=}{(}}}
\def\True {\mathbf{V}}
\def\False{\mathbf{F}}
\def\D    {\displaystyle}

\def\V    {\mathbf{V}}
\def\F    {\mathbf{F}}
\def\ph#1 {\phantom{#1}}
\def\ph   {\phantom}

\def\drafturl{http://angg.twu.net/LATEX/2021-1-C2.pdf}
\def\drafturl{http://angg.twu.net/2021.1-C2.html}
\def\draftfooter{\tiny \href{\drafturl}{\jobname{}} \ColorBrown{\shorttoday{} \hours}}



% «subst-defs»  (to ".subst-defs")
% (find-LATEX "2020-1-C2-TFC2-2.tex" "subst-defs")

\def\pfo#1{\ensuremath{\mathsf{[#1]}}}
\def\veq{\rotatebox{90}{$=$}}
\def\Rd{\ColorRed}
\def\D{\displaystyle}

% Difference with mathstrut
\def\Difms #1#2#3{\left. \mathstrut #3 \right|_{s=#1}^{s=#2}}
\def\Difmu #1#2#3{\left. \mathstrut #3 \right|_{u=#1}^{u=#2}}
\def\Difmx #1#2#3{\left. \mathstrut #3 \right|_{x=#1}^{x=#2}}
\def\Difmth#1#2#3{\left. \mathstrut #3 \right|_{θ=#1}^{θ=#2}}

\def\iequationbox#1#2{
    \left(
    \begin{array}{rcl}
    \D{ #1 } &=& \D{ #2 } \\
    \end{array}
    \right)
  }
\def\isubstbox#1#2#3#4#5{{
    \def\veq{\rotatebox{90}{$=$}}
    \def\ph{\phantom}
    \left(
    \begin{array}{rcl}
    \D{ #1 } &=& \D{ #2 } \\
    {\veq#3} \\
    \D{ #4 } &=& \D{ #5 } \\
    \end{array}
    \right)
  }}
\def\isubstboxT#1#2#3#4#5#6{{
    \def\veq{\rotatebox{90}{$=$}}
    \def\ph{\phantom}
    \left(
    \begin{array}{rcl}
    \multicolumn{3}{l}{\text{#6}} \\%[5pt]
    \D{ #1 } &=& \D{ #2 } \\
    {\veq#3} \\
    \D{ #4 } &=& \D{ #5 } \\
    \end{array}
    \right)
  }}
\def\isubstboxTT#1#2#3#4#5#6#7{{
    \def\veq{\rotatebox{90}{$=$}}
    \def\ph{\phantom}
    \left(
    \begin{array}{rcl}
    \multicolumn{3}{l}{\text{#6}} \\%[5pt]
    \D{ #1 } &=& \D{ #2 } \\
    {\veq#3} \\
    \D{ #4 } &=& \D{ #5 } \\
    \multicolumn{3}{l}{\text{#7}} \\%[5pt]
    \end{array}
    \right)
  }}

% Definição das fórmulas para integração por substituição.
% Algumas são pmatrizes 3x3 usando isubstbox.

\def\TFCtwo{
  \iequationbox {\Intx{a}{b}{F'(x)}}
                {\Difmx{a}{b}{F(x)}}
}
\def\TFCtwoI{
  \iequationbox {\intx{F'(x)}}
                {F(x)}
}

\def\Sone{
  \isubstbox
    {\Difmx{a}{b}{f(g(x))}}  {\Intx{a}{b}{f'(g(x))g'(x)}}
    {\ph{mmm}}
    {\Difmu{g(a)}{g(b)}{f(u)}} {\Intu{g(a)}{g(b)}{f'(u)}}
}
\def\SoneI{
  \isubstbox
    {f(g(x))} {\intx{f'(g(x))g'(x)}}
    {\ph{m}}
    {f(u)}    {\intu{f'(u)}}
}

\def\Stwo{
  \isubstboxT
    {\Difmx{a}{b}{F(g(x))}}   {\Intx{a}{b}{f(g(x))g'(x)}}
    {\ph{mmm}}
    {\Difmu{g(a)}{g(b)}{F(u)}}  {\Intu{g(a)}{g(b)}{f(u)}}
    {Se $F'(u)=f(u)$ então:}
}
\def\StwoI{
  \isubstboxT
    {F(g(x))}  {\intx{f(g(x))g'(x)}}
    {\ph{m}}
    {F(u)}     {\intu{f(u)}}
    {Se $F'(u)=f(u)$ então:}
}
\def\StwoI{
  \isubstboxTT
    {F(g(x))}  {\intx{f(g(x))g'(x)}}
    {\ph{m}}
    {F(u)}     {\intu{f(u)}}
    {Se $F'(u)=f(u)$ então:}
    {Obs: $u=g(x)$.}
}

\def\Sthree{
  \iequationbox {\Intx{a}{b}{f(g(x))g'(x)}}
                {\Intu{g(a)}{g(b)}{f(u)}}
}
\def\SthreeI{
  \iequationbox {\intx{f(g(x))g'(x)}}
                {\intu{f(u)}
                 \qquad [u=g(x)]
                }
  % [u=g(x)]
}

\def\Sthree{
  \pmat{
    \D \Intx{a}{b}{f(g(x))g'(x)} \\
    \veq \\
    \D \Intu{g(a)}{g(b)}{f(u)}
  }}

\def\SthreeI{
  \pmat{
    \D \intx{f(g(x))g'(x)} \\
       \veq \\
    \D \intu{f(u)} \\
    \text{Obs: $u=g(x)$.} \\
  }}



\def\Subst#1{\bmat{#1}}




%  _____ _ _   _                               
% |_   _(_) |_| | ___   _ __   __ _  __ _  ___ 
%   | | | | __| |/ _ \ | '_ \ / _` |/ _` |/ _ \
%   | | | | |_| |  __/ | |_) | (_| | (_| |  __/
%   |_| |_|\__|_|\___| | .__/ \__,_|\__, |\___|
%                      |_|          |___/      
%
% «title»  (to ".title")
% (c2m211substp 1 "title")
% (c2m211subst    "title")

\thispagestyle{empty}

\begin{center}

\vspace*{1.2cm}

{\bf \Large Cálculo 2 - 2021.1}

\bsk

% Aula nn: ponha o título aqui

Aula 1: O operador de substituição `$[:=]$'

\bsk

Eduardo Ochs - RCN/PURO/UFF

\url{http://angg.twu.net/2021.1-C2.html}

\end{center}

\newpage

% «corrigir-igual»  (to ".corrigir-igual")
% (c2m211substp 2 "corrigir-igual")
% (c2m211substa   "corrigir-igual")

{\bf ``Eu só vou corrigir os sinais de igual''}

Uma dos slogans que eu mais vou repetir quando estiver tirando dúvidas
ou corrigindo exercícios de vocês é ``\ColorRed{Eu só vou corrigir os
  sinais de igual}''.

Em Cálculo 1 muita gente se enrola com a fórmula da regra da cadeia --
porque se enrola na hora de substituir os `$f$'s, `$g$'s, `$f'$'s e
`$g'$'s nela... uma das fórmulas mais importantes, e mais difíceis de
acreditar, de Cálculo 2 é a da \ColorRed{Integração por Substituição},
que é BEEEEM pior do que a Regra da Cadeia. O \ColorRed{operador de
  substituição}, `$[:=]$', que não tem \ColorRed{nada a ver} com a
Integração por Substituição, vai nos ajudar bastante a aplicar essas
fórmulas passo a passo sem a gente se perder.

\newpage

% «subst-box»  (to ".subst-box")
% (c2m211substp 3 "subst-box")
% (c2m211substa   "subst-box")

Nós vamos reescrever isto:

\msk

\begin{center}
\fbox{\begin{minipage}{7cm}
Se substituirmos $x$ por $10a+b$

e $y$ por $3c+4d$ em:
%
$$x^y + 2x$$
%
obtemos:
%
$$(10a+b)^{3c+4d} + 2(10a+b)$$
\end{minipage}}
\end{center}

\msk

deste jeito:

$$(x^y+2x) \bmat{x:=10a+b \\ y:=3c+4d} = (10a+b)^{3c+4d} + 2(10a+b)$$

\newpage

% «subst-zoomed»  (to ".subst-zoomed")
% (c2m211substp 4 "subst-zoomed")
% (c2m211substa   "subst-zoomed")

Repare: em
%
$$\scalebox{2.0}{$
  \begin{array}{l}
  (x^y+2x) \bmat{x:=10a+b \\ y:=3c+4d} \\[12pt]
  = (10a+b)^{3c+4d} + 2(10a+b)
  \end{array}
  $}
$$

a notação é
%
$$\text{(expressão original)[substituições] = (expressão nova)}$$

e cada uma das substituições é da forma:
%
$$\text{variável} := \text{expressão}$$


\newpage

% «testar-hipoteses»  (to ".testar-hipoteses")
% (c2m211substp 5 "testar-hipoteses")
% (c2m211substa   "testar-hipoteses")

A notação `$[:=]$' vai ser bem prática pra gente fazer hipóteses e
testá-las. Por exemplo, digamos que queremos testar se 2 e 3 são
soluções da equação $x+2=5$...
%
$$\begin{array}{rcl}
  (x+2=5)[x:=2] &=& (2+2=5) \\
                &=& (4=5) \\
                &=& \False \\[5pt]
  (x+2=5)[x:=3] &=& (3+2=5) \\
                &=& (5=5) \\
                &=& \True \\[5pt]
  \end{array}
$$

Note que os `$=$'s das expressões entre parênteses são
\ColorRed{comparações} -- como a operação `\texttt{==}' do \texttt{C}
-- e retornam ou $\True$ (``Verdadeiro'') ou $\False$ (``Falso'').

\newpage

% «regra-da-cadeia»  (to ".regra-da-cadeia")
% (c2m211substp 6 "regra-da-cadeia")
% (c2m211substa   "regra-da-cadeia")

{\bf Exemplo: regra da cadeia}

Primeiro vou inventar uma abreviação para a regra da cadeia.

\ColorRed{Obs: vários dos truques que vamos usar agora são inspirados
  em notações de Teoria da Computação e não são padrão!!! Não use eles
  em outros cursos!!! {\bf Os professores podem não entender e podem
    ficar putos!!!}}

\msk

O `$:=$' abaixo é uma \ColorRed{atribuição}, como o `\texttt{=}' do
\texttt{C}. A linha abaixo quer dizer: ``\ColorRed{a partir de agora}
o valor de $[RC]$ vai ser a \ColorRed{expressão} entre os parênteses
grandes.
%
$$[RC] \;\; := \;\; \left( \ddx f(g(x)) = f'(g(x))g'(x) \right)$$

\newpage

% «regra-da-cadeia-2»  (to ".regra-da-cadeia-2")
% (c2m211substp 7 "regra-da-cadeia-2")
% (c2m211substa   "regra-da-cadeia-2")

{\bf Exemplo: regra da cadeia (2)}

Continuando...
%
$$[RC] \;\; := \;\; \left( \ddx f(g(x)) = f'(g(x))g'(x) \right)$$

Então:
%
$$\begin{array}{rcl}
  [RC] \bmat{f := \sen} &=&
     \left( \ddx \sen(g(x)) = \sen'(g(x))g'(x) \right) \\[5pt]
  [RC] \bmat{f(u) := \sen u} &=&
     \left( \ddx \sen(g(x)) = \sen'(g(x))g'(x) \right) \\[5pt]
  [RC] \bmat{f(u) := u^4 \\ f'(u) := 4u^3 } &=&
     \left( \ddx (g(x))^4 = 4(g(x))^3g'(x) \right) \\
  \end{array}
$$

Repare que agora estamos substituindo o `$f$' \ColorRed{como se ele
  fosse uma variável} -- mas precisamos de gambiarras novas. No caso
do meio escrevemos $f(u) := \sen u$ ao invés de $f := \sen$, e...


\newpage

% «acrescentamos»  (to ".acrescentamos")
% (c2m211substp 8 "acrescentamos")
% (c2m211substa   "acrescentamos")

$$[RC] \;\; := \;\; \left( \ddx f(g(x)) = f'(g(x))g'(x) \right)$$

$$\begin{array}{rcl}
  [RC] \bmat{f := \sen} &=&
     \left( \ddx \sen(g(x)) = \sen'(g(x))g'(x) \right) \\[5pt]
  [RC] \bmat{f(u) := \sen u} &=&
     \left( \ddx \sen(g(x)) = \sen'(g(x))g'(x) \right) \\[5pt]
  [RC] \bmat{f(u) := u^4 \\ f'(u) := 4u^3 } &=&
     \left( \ddx (g(x))^4 = 4(g(x))^3g'(x) \right) \\
  \end{array}
$$
%
...e no caso de baixo acrescentamos uma linha ``$f'(u) := 4u^3$'' na
lista de substituições. Essa linha é uma \ColorRed{consequencia} da
linha ``$f(u) := u^4$'', e ela está lá só pra ajudar a gente a se
enrolar menos.


\newpage

% «dicas-subst»  (to ".dicas-subst")
% (c2m211substp 9 "dicas-subst")
% (c2m211substa   "dicas-subst")
% (c2m202isp 6 "dicas-subst")
% (c2m202isa   "dicas-subst")

% «igual-depois-de-subst»  (to ".igual-depois-de-subst")
% (c2m211substp 9 "igual-depois-de-subst")
% (c2m211substa   "igual-depois-de-subst")

{\bf Uma regra estranha: o `$=$' depois da operação `$[:=]$'}

\ssk

Nas duas substituições abaixo a primeira está certa

e a segunda está errada:
%
$$\begin{array}{rll}
  (x + 2 = 5) \, [x:=4] &=& (4 + 2 = 5) \\
  (x + 2 = 5) \, [x:=4] &=& (6 = 5) \\
  \end{array}
$$

O `$=$' depois de uma substituição tem um significado especial: a
pronúncia dele é ``o resultado da substituição à esquerda é a
expressão à direita'', e na segunda linha a gente fez mais coisas além
de só substituir todos os `$x$'s por `4's.

Note que isto aqui está certo:
%
$$\begin{array}{rll}
  (x + 2 = 5) \, [x:=4] &=& (4 + 2 = 5) \\
                        &=& (6 = 5) \\
  \end{array}
$$

\newpage

{\bf A explicação pra regra estranha}

Vocês já devem ter visto em Prog que vocês podem definir as funções de
vocês, e nas matérias de Matemática vocês também vão aprender a fazer
definições de vários tipos. A operação `$[:=]$' que nós estamos usando
é uma operação \ColorRed{que eu defini} baseada em operações parecidas
com ela que aparecem em muitos lugares, mas que geralmente ficam meio
disfarçadas --- e que ficam disfarçadas de ``óbvias''.

Então, esse nosso `$[:=]$' é uma operação nova, e a gente tem que
definir todas as regras de como ela vai funcionar. Tem vários detalhes
em que a gente poderia definir se ela iria funcionar de um jeito ou de
outro, e eu \ColorRed{escolhi} que ela vai funcionar do jeito que vai
nos ajudar mais a fazer contas fáceis de entender...

\ColorRed{...e eu vi que a restricão do slide anterior faz com que as
  pessoas (incluindo eu!) se enrolem muito menos nas contas.}

\newpage

{\bf A explicação pra regra estranha (2)}

No primeiro vídeo deste semestre eu mostrei que o SymPy, que e' um
programa de computação simbólica, tem uma espécie de `$[:=]$', que ele
chama de `{\tt subs}'. A definição do {\tt subs} no SymPy é MUITO mais
difícil do que a gente vai precisar aqui em Cálculo 2 --- ela envolve
recursão, ela tem um truque pra lidar do jeito ``certo'' com variáveis
livres, e ela tem um truque complicadíssimo --- que o Bruno Macedo,
que foi monitor no semestre passado, descobriu e me mostrou --- pra
substituir coisas que não são só variáveis.


\newpage

% «dicas-subst-2»  (to ".dicas-subst-2")
% (c2m211substp 10 "dicas-subst-2")
% (c2m211substa    "dicas-subst-2")

{\bf Outro exemplo de uso errado do `$[:=]$'}

Aqui a primeira substituição está certa e a segunda está errada...

Na segunda um `$u$' foi substituido por `$e^{2x}$'!!!!!!!! $\;\;\;=\!($
%
$$\scalebox{0.9}{$
  \begin{array}{rcl}
  \SthreeI [g(x):=e^{2x}] & = &
     \pmat{ \D \intx{f(2^{2x})(2e^{2x})} \\
            \veq \\
            \D \intu{f(u)} \\
            \text{Obs: $u=e^{2x}$.} \\
          }
  \\
  \\
  \SthreeI [g(x):=e^{2x}] & = &
     \pmat{ \D \intx{f(2^{2x})(2e^{2x})} \\
            \veq \\
            \D \intu{f(\ColorRed{e^{2x}})} \\
            \text{Obs: $u=e^{2x}$.} \\
          }
  \end{array}
  $}
$$



\newpage

% «dicas-subst-3»  (to ".dicas-subst-3")
% (c2m211substp 11 "dicas-subst-3")
% (c2m211substa    "dicas-subst-3")

{\bf Mais dicas sobre a operação `$[:=]$' (3)}

No primeiro PDF do curso nós usamos a operação `$[:=]$' para testar
EDOs como $f'(x)=x^4$ em vários ``valores'' de $f$, pra tentar
resolver EDOs por chutar-e-testar... Em
%
$$(f'(x)=x^4)\, [f(x):=x^2] = (2x = x^4)$$
%
na expressão original, $(f'(x)=x^4)$, o símbolo $f$ faz o papel de uma
função qualquer, ou de uma variável cujo valor é uma função; a
substiuição ``$[f(x):=x^2]$'' diz como substituir a $f$ original,
genérica, pela $f$ que tem esta {\sl definição} aqui: $f(x)=x^2$... e
nós já temos bastante prática com obter consequências de uma definição
como $f(x)=x^2$. Por exemplo:
%
$$\begin{array}{rclcrcl}
  f(200)  &=& 200^2            && f'(x) &=& 2x \\
  f(3u+4) &=& (3u+4)^2         && f'(3u+4) &=& 2(3u+4) \\
  f(42x^3+99) &=& (42x^3+99)^2 && f'(42x^3+99) &=& 2(42x^3+99) \\
  \end{array}
$$


\newpage

{\bf Mais dicas sobre a operação `$[:=]$' (4)}

No slide anterior eu disse que
%
$$f(x)=x^2$$

tem estas consequências, entre muitas outras:
%
$$\begin{array}{rclcrcl}
  f(200)  &=& 200^2            && f'(x) &=& 2x \\
  f(3u+4) &=& (3u+4)^2         && f'(3u+4) &=& 2(3u+4) \\
  f(42x^3+99) &=& (42x^3+99)^2 && f'(42x^3+99) &=& 2(42x^3+99) \\
  \end{array}
$$

Vamos entender isso em português.

Se $f(x)=x^2$ é verdade para todo $x$

então $f'(x)=2x$ para todo $x$.

Obs: aqui você também pode pensar graficamente!

A curva $y=f(x)$ é uma parábola, e $f'(x)$

é o coeficiente angular dela.

\newpage

% «dicas-subst-5»  (to ".dicas-subst-5")
% (c2m211substp 13 "dicas-subst-5")
% (c2m211substa    "dicas-subst-5")

{\bf Mais dicas sobre a operação `$[:=]$' (5)}

Continuando: temos
%
$$\begin{array}{rcl}
   f(x) &=& x^2 \quad \text{e} \\
  f'(x) &=& 2x, \\
  \end{array}
$$

então no ponto $x=200$ temos $f(x)=200^2$ e $f'(x)=2·200$,

e em $x=3u+4$ temos
%
$$\begin{array}{rcl}
   f(3u+4) &=& (3u+4)^2 \quad \text{e}\\
  f'(3u+4) &=& 2(3u+4). \\
  \end{array}
$$

para todo $u∈\R$.

Muitos livros fingem que isso é óbvio ---

eles dizem só ``podemos substituir $x$ por $3u+4$'' ---

mas eu acho que não é óbvio não... quando eu estava na

graduação eu tive que pensar vários dias pra entender isso.

\newpage

% «dicas-subst-6»  (to ".dicas-subst-6")
% (c2m211substp 14 "dicas-subst-6")
% (c2m211substa    "dicas-subst-6")

{\bf Mais dicas sobre a operação `$[:=]$' (6)}

A operação `$[:=]$' nos permite fazer a substituição

de $x$ por $3u+4$ ``mecanicamente'' ---

ou melhor: ``sintaticamente'' ---

sem a gente ter que pensar muito em {\sl porque} essa

substituição faz sentido. Por exemplo:


\def\prcl#1{ \left( \begin{array}{rcl} #1 \end{array} \right) }

$$\begin{array}{c}
  \prcl{
    f(x) &=& x^2 \quad \text{e} \\
    f'(x) &=& 2x
    }
  \;
  [x:=3u+4]
  \\[10pt]
  = \prcl{
       f(3u+4) &=& (3u+4)^2 \quad \text{e}\\
       f'(3u+4) &=& 2(3u+4) \\
    }
  \end{array}
$$

O fato é que \ColorRed{variáveis são feitas para serem substituídas}.

Um modo da gente se acostumar com como isso funciona é

testando \ColorRed{muitos} casos particulares --- como no exercício

do próximo slide.

\newpage

% «exercicios-1-e-2»  (to ".exercicios-1-e-2")
% (c2m211substp 15 "exercicios-1-e-2")
% (c2m211substa    "exercicios-1-e-2")

{\bf Exercício 1}

Digamos que $f(x)=x^2$...

a) e que $u=0$. Neste caso é verdade que $f(3u+4) = (3u+4)^2$?

b) e que $u=1$. Neste caso é verdade que $f(3u+4) = (3u+4)^2$?

c) e que $u=10$. Neste caso é verdade que $f(3u+4) = (3u+4)^2$?

\msk

{\bf Exercício 2}

Digamos que $f(x)=42$...

a) e que $u=0$. Neste caso é verdade que $f(3u+4) = (3u+4)^2$?

b) e que $u=1$. Neste caso é verdade que $f(3u+4) = (3u+4)^2$?

c) e que $u=10$. Neste caso é verdade que $f(3u+4) = (3u+4)^2$?

\bsk

Note que nós não testamos todos os valores possíveis de $u$,

nem todos as funções `$f$' possíveis, e nem usamos

a notação `$[:=]$'...

\newpage

A operação `$[:=]$' nos permite fazer substituições como

`$[x:=10x+2]$', que parecem bem estranhas à primeira vista.

\bsk

{\bf Exercício 3.}

Calcule o resultado das substituições abaixo --- ou seja,

calcule o que você deve pôr no lugar do `$\ColorRed{?}$' em cada item.

\bsk

(Ooops -- ainda não terminei de escrever esse exercício)

\newpage

%  ____                        _             _           
% / ___|  ___  _ __ ___   __ _| |_ ___  _ __(_) ___  ___ 
% \___ \ / _ \| '_ ` _ \ / _` | __/ _ \| '__| |/ _ \/ __|
%  ___) | (_) | | | | | | (_| | || (_) | |  | | (_) \__ \
% |____/ \___/|_| |_| |_|\__,_|\__\___/|_|  |_|\___/|___/
%                                                        
% «somatorios»  (to ".somatorios")
% (c2m211substp 19 "somatorios")
% (c2m211substa    "somatorios")

{\bf Somatórios}

Antigamente somatórios eram matéria de ensino médio,

mas hoje em dia muita gente chega em Cálculo 2 sem

nunca ter visto somatórios...

\ssk

As fórmulas para somas de progressões aritméticas (PAs) e para

somas de progressões geométricas (PGs) usam `$\sum$'s. Veja:

\bsk

{\footnotesize

% https://pt.wikipedia.org/wiki/Progress%C3%A3o_aritm%C3%A9tica
\url{https://pt.wikipedia.org/wiki/Progress\%C3\%A3o_aritm\%C3\%A9tica}

% https://pt.wikipedia.org/wiki/Progress%C3%A3o_geom%C3%A9trica
\url{https://pt.wikipedia.org/wiki/Progress\%C3\%A3o_geom\%C3\%A9trica}

% https://pt.wikipedia.org/wiki/Somat%C3%B3rio
\url{https://pt.wikipedia.org/wiki/Somat\%C3\%B3rio}

}

\newpage

% «soma-PG»  (to ".soma-PG")
% (c2m211substp 13 "soma-PG")
% (c2m211substa    "soma-PG")

Relembre:
%
$$\scalebox{0.9}{$
  \begin{array}{rcl}
         \sum_{k=2}^{5} 10^k &=& 10^2 + 10^3 + 10^4 + 10^5 \\
                             &=& 100 + 1000 + 10000 + 100000 \\
                             &=& 111100 \\
  (1-10) \sum_{k=2}^{5} 10^k &=& (1-10)(100 + 1000 + 10000 + 100000) \\
                             &=& (100 + 1000 + 10000 + 100000) \\
                              && - (1000 + 10000 + 100000 + 1000000) \\
                             &=& 100 - 1000000 \\
                             &=& 10^2 - 10^{5+1} \\
         \sum_{k=2}^{5} 10^k &=& \D \frac{10^2 - 10^{5+1}}{1-10} \\
  \end{array}
  $}
$$

A fórmula geral é: \quad
%
$\D \sum_{k=a}^{b} x^k
 \; = \;
 \D \frac{x^a - x^{b+1}}{1 - x}
 \; = \;
 \D \frac{x^{b+1}- x^a}{x - 1}
$
\;.

\newpage

% «somatorio-expansao»  (to ".somatorio-expansao")

Repare que dá pra calcular o somatório do início

do slide anterior em mais passos usando o `$[:=]$'...

$$\scalebox{0.9}{$
  \begin{array}{rcl}
         \sum_{k=2}^{5} 10^k &=& 10^2 + 10^3 + 10^4 + 10^5 \\[5pt]
         \sum_{k=2}^{5} 10^k &=& (10^k) [k:=2] \\
                             &+& (10^k) [k:=3] \\
                             &+& (10^k) [k:=4] \\
                             &+& (10^k) [k:=5] \\[2.5pt]
                             &=& 10^2 + 10^3 + 10^4 + 10^5 \\
  \end{array}
  $}
$$

Às vezes a gente vai usar esse passo intermediário com `$[:=]$'s

pra não se enrolar em somatórios de expressões complicadas...

Por exemplo aqui, e nas páginas seguintes:

\ssk

% (c2m211somas1p 12 "partition-sum")
% (c2m211somas1a    "partition-sum")
% http://angg.twu.net/LATEX/2021-1-C2-somas-1.pdf#page=12
{\footnotesize

\url{http://angg.twu.net/LATEX/2021-1-C2-somas-1.pdf\#page=12}

}


\newpage

% «somatorios-exercs»  (to ".somatorios-exercs")
% (c2m211substp 22 "somatorios-exercs")
% (c2m211substa    "somatorios-exercs")

{\bf Exercícios básicos de somatórios}

\msk

Expanda e calcule:

a) $\sum_{n=1}^5 (2n-1)$

\ssk

b) $\sum_{n=0}^4 (2n+1)$

\msk

c) $\sum_{k=0}^2 (k+1)$

\msk

d) $\sum_{k=0}^2 k + 1$

\msk

e) $\left( \sum_{k=0}^2 k \right) +1$

\msk

Expanda e calcule/simplifique até onde der:

f) $\sum_{n=1}^5 (2k-1)$

\ssk

g) $\sum_{k=1}^5 (2n-1)$

\ssk

h) $\sum_{n=4}^6 f(10n)$

\ssk

i) $\sum_{n=4}^6 f(10n)$, onde $f(x) = 10x$


\newpage

% «para-todo-e-existe»  (to ".para-todo-e-existe")
% (c2m211substp 23 "para-todo-e-existe")
% (c2m211substa    "para-todo-e-existe")
% (lodp 5 "dm-layer-1")
% (loda   "dm-layer-1")

{\bf ``Para todo'' ($∀$) e ``existe'' ($∃$)}

\msk

$\scalebox{0.9}{$
  \begin{array}{rcl}
  (∀a∈\{2,3,5\}.a^2<10) &=& (a^2<10)[a:=2] \;∧ \\&&
                            (a^2<10)[a:=3] \;∧ \\&&
                            (a^2<10)[a:=5] \\
                        &=& (2^2<10) ∧
                            (3^2<10) ∧
                            (4^2<10) \\
                        &=& (4<10) ∧
                            (9<10) ∧
                            (16<10) \\
                        &=& \V ∧ \V ∧ \F \\
                        &=& \F \\[5pt]
  (∃a∈\{2,3,5\}.a^2<10) &=& (a^2<10)[a:=2] \;∨ \\&&
                            (a^2<10)[a:=3] \;∨ \\&&
                            (a^2<10)[a:=5] \\
                        &=& (2^2<10) ∨
                            (3^2<10) ∨
                            (4^2<10) \\
                        &=& (4<10) ∨
                            (9<10) ∨
                            (16<10) \\
                        &=& \V ∨ \V ∨ \F \\
                        &=& \V \\
  \end{array}
 $}
$

\newpage

% «visualizando-fas-e-exs»  (to ".visualizando-fas-e-exs")
% (c2m211substp 24 "visualizando-fas-e-exs")
% (c2m211substa    "visualizando-fas-e-exs")

{\bf Visualizando `$∀$'s e `$∃$'s}

Repare...

\msk

{
\def\V    {\mathbf{V}}
\def\F    {\mathbf{F}}
\def\mbc#1{\hbox to 8pt{\hss$#1$\hss}}
\def\V    {\mbc{\mathbf{V}}}
\def\F    {\mbc{\mathbf{F}}}

$\scalebox{0.9}{$
  \begin{array}{lcl}
  (∀x∈\{1,\ldots,7\}.2≤x)            &=& \F∧\V∧\V∧\V∧\V∧\V∧\V \\
  (∀x∈\{1,\ldots,7\}.\ph{mm}x<4)     &=& \V∧\V∧\V∧\F∧\F∧\F∧\F \\
  (∀x∈\{1,\ldots,7\}.2≤x<4)          &=& \F∧\V∧\V∧\F∧\F∧\F∧\F \\
  (∀x∈\{1,\ldots,7\}.\ph{mmmmmm}x=6) &=& \F∧\F∧\F∧\F∧\F∧\V∧\F \\
  (∀x∈\{1,\ldots,7\}.2≤x<4∨     x=6) &=& \F∧\V∧\V∧\F∧\F∧\V∧\F \\
  \end{array}
  $}
$
}

\msk

...que dá pra {\sl visualizar} o que a expressão

$(∀x∈\{1,\ldots,7\}.2≤x<4∨x=6)$

``quer dizer'' visualizando os `$\V$'s e `$\F$'s

de expressões mais simples, e combinando

esses ``mapas'' de `$\V$'s e `$\F$'s.

\newpage

% «visualizando-fas-e-exs-2»  (to ".visualizando-fas-e-exs-2")
% (c2m211substp 20 "visualizando-fas-e-exs-2")
% (c2m211substa    "visualizando-fas-e-exs-2")

{\bf Visualizando `$∀$'s e `$∃$'s (2)}

Às vezes vai valer a pena \ColorRed{definir proposições}

como nomes mais curtos, como $F(x) = (2≤x)$,

$G(x) = (x≤4)$, $H(x) = (x=6)$... Aí:

\msk

{
\def\mbc#1{\hbox to 8pt{\hss$#1$\hss}}
\def\V    {\mbc{\mathbf{V}}}
\def\F    {\mbc{\mathbf{F}}}

$\scalebox{0.9}{$
  \begin{array}{lcl}
  (∀x∈\{1,\ldots,7\}.F(x))              &=& \F∧\V∧\V∧\V∧\V∧\V∧\V \\
  (∀x∈\{1,\ldots,7\}.\ph{mmmii}G(x))    &=& \V∧\V∧\V∧\F∧\F∧\F∧\F \\
  (∀x∈\{1,\ldots,7\}.F(x)∧G(x))         &=& \F∧\V∧\V∧\F∧\F∧\F∧\F \\
  (∀x∈\{1,\ldots,7\}.\ph{mmmmmmmi}H(x)) &=& \F∧\F∧\F∧\F∧\F∧\V∧\F \\
  (∀x∈\{1,\ldots,7\}.F(x)∧G(x)∨ H(x))   &=& \F∧\V∧\V∧\F∧\F∧\V∧\F \\
  \end{array}
  $}
$
}

\msk

É isso que a gente vai fazer pra analisar expressões

como $(∀x∈A.▁▁▁)$ e $(∃x∈A.▁▁▁)$ e descobrir quais

são verdadeiras e quais não --- \ColorRed{mesmo quando o conjunto

$A$ é um conjunto infinito}, como $\N$, $\R$ ou $[2,10]$.


\newpage

% «visualizando-fas-e-exs-3»  (to ".visualizando-fas-e-exs-3")
% (c2m211substp 26 "visualizando-fas-e-exs-3")
% (c2m211substa    "visualizando-fas-e-exs-3")

{\bf Visualizando `$∀$'s e `$∃$'s (3)}

Às vezes vamos ter que fazer figuras com muitos `$\V$'s e `$\F$'s,

e vai ser mais fácil visualizar onde estão os `$\V$'s e `$\F$'s

delas se usarmos sinais mais fáceis de distinguir...

\msk

Por exemplo, se $•:=\V$ e $∘:=\F$ então:

\msk

{
\def\mbc#1{\hbox to 8pt{\hss$#1$\hss}}
\def\V    {\mbc{\mathbf{V}}}
\def\V    {\mbc{•}}
\def\F    {\mbc{∘}}

$\scalebox{0.9}{$
  \begin{array}{lcl}
  (∀x∈\{1,\ldots,7\}.F(x))              &=& \F∧\V∧\V∧\V∧\V∧\V∧\V \\
  (∀x∈\{1,\ldots,7\}.\ph{mmmii}G(x))    &=& \V∧\V∧\V∧\F∧\F∧\F∧\F \\
  (∀x∈\{1,\ldots,7\}.F(x)∧G(x))         &=& \F∧\V∧\V∧\F∧\F∧\F∧\F \\
  (∀x∈\{1,\ldots,7\}.\ph{mmmmmmmi}H(x)) &=& \F∧\F∧\F∧\F∧\F∧\V∧\F \\
  (∀x∈\{1,\ldots,7\}.F(x)∧G(x)∨ H(x))   &=& \F∧\V∧\V∧\F∧\F∧\V∧\F \\
  \end{array}
  $}
$
}

\bsk

Você \ColorRed{pode} fazer as suas próprias definições ---

como o meu ``$•:=\V$ e $∘:=\F$'' acima --- mas elas

têm que ficar claras o suficiente... lembre desta dica:

% (c2m211somas1dp 7 "dica-7")
% (c2m211somas1da   "dica-7")
% http://angg.twu.net/LATEX/2021-1-C2-somas-1-dicas.pdf#page=7

\ssk

{\footnotesize

\url{http://angg.twu.net/LATEX/2021-1-C2-somas-1-dicas.pdf\#page=7}

}


\newpage






% http://angg.twu.net/LATEX/2020-2-C2-int-subst.pdf#page=7
% http://angg.twu.net/LATEX/2020-2-C2-subst-trig.pdf#page=9
% http://angg.twu.net/LATEX/2020-2-C2-subst-trig.pdf#page=14


%\printbibliography

\GenericWarning{Success:}{Success!!!}  % Used by `M-x cv'

\end{document}

%  ____  _             _         
% |  _ \(_)_   ___   _(_)_______ 
% | | | | \ \ / / | | | |_  / _ \
% | |_| | |\ V /| |_| | |/ /  __/
% |____// | \_/  \__,_|_/___\___|
%     |__/                       
%
% «djvuize»  (to ".djvuize")
% (find-LATEXgrep "grep --color -nH --null -e djvuize 2020-1*.tex")

 (eepitch-shell)
 (eepitch-kill)
 (eepitch-shell)
# (find-fline "~/2021.1-C2/")
# (find-fline "~/LATEX/2021-1-C2/")
# (find-fline "~/bin/djvuize")

cd /tmp/
for i in *.jpg; do echo f $(basename $i .jpg); done

f () { rm -fv $1.png $1.pdf; djvuize $1.pdf }
f () { rm -fv $1.png $1.pdf; djvuize WHITEBOARDOPTS="-m 1.0" $1.pdf; xpdf $1.pdf }
f () { rm -fv $1.png $1.pdf; djvuize WHITEBOARDOPTS="-m 0.5" $1.pdf; xpdf $1.pdf }
f () { rm -fv $1.png $1.pdf; djvuize WHITEBOARDOPTS="-m 0.25" $1.pdf; xpdf $1.pdf }
f () { cp -fv $1.png $1.pdf       ~/2021.1-C2/
       cp -fv        $1.pdf ~/LATEX/2021-1-C2/
       cat <<%%%
% (find-latexscan-links "C2" "$1")
%%%
}

f 20201213_area_em_funcao_de_theta
f 20201213_area_em_funcao_de_x
f 20201213_area_fatias_pizza



%  __  __       _        
% |  \/  | __ _| | _____ 
% | |\/| |/ _` | |/ / _ \
% | |  | | (_| |   <  __/
% |_|  |_|\__,_|_|\_\___|
%                        
% <make>

 (eepitch-shell)
 (eepitch-kill)
 (eepitch-shell)
# (find-LATEXfile "2019planar-has-1.mk")
make -f 2019.mk STEM=2021-1-C2-subst veryclean
make -f 2019.mk STEM=2021-1-C2-subst pdf

% Local Variables:
% coding: utf-8-unix
% ee-tla: "c2m211subst"
% End:
