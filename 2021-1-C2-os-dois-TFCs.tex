% (find-LATEX "2021-1-C2-os-dois-TFCs.tex")
% (defun c () (interactive) (find-LATEXsh "lualatex -record 2021-1-C2-os-dois-TFCs.tex" :end))
% (defun C () (interactive) (find-LATEXsh "lualatex 2021-1-C2-os-dois-TFCs.tex" "Success!!!"))
% (defun D () (interactive) (find-pdf-page      "~/LATEX/2021-1-C2-os-dois-TFCs.pdf"))
% (defun d () (interactive) (find-pdftools-page "~/LATEX/2021-1-C2-os-dois-TFCs.pdf"))
% (defun e () (interactive) (find-LATEX "2021-1-C2-os-dois-TFCs.tex"))
% (defun o () (interactive) (find-LATEX "2020-2-C2-TFC.tex"))
% (defun u () (interactive) (find-latex-upload-links "2021-1-C2-os-dois-TFCs"))
% (defun v () (interactive) (find-2a '(e) '(d)))
% (defun d0 () (interactive) (find-ebuffer "2021-1-C2-os-dois-TFCs.pdf"))
% (defun cv () (interactive) (C) (ee-kill-this-buffer) (v) (g))
%          (code-eec-LATEX "2021-1-C2-os-dois-TFCs")
% (find-pdf-page   "~/LATEX/2021-1-C2-os-dois-TFCs.pdf")
% (find-sh0 "cp -v  ~/LATEX/2021-1-C2-os-dois-TFCs.pdf /tmp/")
% (find-sh0 "cp -v  ~/LATEX/2021-1-C2-os-dois-TFCs.pdf /tmp/pen/")
%     (find-xournalpp "/tmp/2021-1-C2-os-dois-TFCs.pdf")
%   file:///home/edrx/LATEX/2021-1-C2-os-dois-TFCs.pdf
%               file:///tmp/2021-1-C2-os-dois-TFCs.pdf
%           file:///tmp/pen/2021-1-C2-os-dois-TFCs.pdf
% http://angg.twu.net/LATEX/2021-1-C2-os-dois-TFCs.pdf
% (find-LATEX "2019.mk")
% (find-CN-aula-links "2021-1-C2-os-dois-TFCs" "2" "c2m211tfcs" "c2t")

% «.video-1»			(to "video-1")
% «.defs»			(to "defs")
% «.title»			(to "title")
% «.diferenca»			(to "diferenca")
% «.exercicio-1»		(to "exercicio-1")
% «.o-que-vai-ser-o-TFC2»	(to "o-que-vai-ser-o-TFC2")
% «.truques-simplificar»	(to "truques-simplificar")
% «.exercicio-2»		(to "exercicio-2")
% «.integrar-chutando»		(to "integrar-chutando")
% «.exercicio-3»		(to "exercicio-3")
% «.limites-de-integracao»	(to "limites-de-integracao")
% «.limites-de-integracao-2»	(to "limites-de-integracao-2")
% «.limites-de-integracao-3»	(to "limites-de-integracao-3")
% «.integral-indefinida»	(to "integral-indefinida")
% «.exercicio-4»		(to "exercicio-4")
% «.exercicio-5»		(to "exercicio-5")
%
% «.djvuize»			(to "djvuize")

% «video-1»  (to ".video-1")
% (c2m211tfcsa    "video-1")
% (find-ssr-links "c2m211tfcs" "2021-1-C2-os-dois-TFCs" "78Th0InLPX4")
% (code-video     "c2m211tfcsvideo" "$S/http/angg.twu.net/eev-videos/2021-1-C2-os-dois-TFCs.mp4")
% (find-c2m211tfcsvideo "0:00" "19/ago/2021")
% (find-c2m211tfcsvideo "2:50")
% (find-c2m211tfcsvideo "6:30" "altura do f depois e do f antes")
% (find-c2m211tfcsvideo "18:00" "aqui a derivada dava 2")
% (find-c2m211tfcsvideo "18:30" "nos outros ponto F'(x)=f(x) era verdade")

\documentclass[oneside,12pt]{article}
\usepackage[colorlinks,citecolor=DarkRed,urlcolor=DarkRed]{hyperref} % (find-es "tex" "hyperref")
\usepackage{amsmath}
\usepackage{amsfonts}
\usepackage{amssymb}
\usepackage{pict2e}
\usepackage[x11names,svgnames]{xcolor} % (find-es "tex" "xcolor")
\usepackage{colorweb}                  % (find-es "tex" "colorweb")
%\usepackage{tikz}
%
% (find-dn6 "preamble6.lua" "preamble0")
%\usepackage{proof}   % For derivation trees ("%:" lines)
%\input diagxy        % For 2D diagrams ("%D" lines)
%\xyoption{curve}     % For the ".curve=" feature in 2D diagrams
%
\usepackage{edrx21}               % (find-LATEX "edrx21.sty")
\input edrxaccents.tex            % (find-LATEX "edrxaccents.tex")
\input edrx21chars.tex            % (find-LATEX "edrx21chars.tex")
\input edrxheadfoot.tex           % (find-LATEX "edrxheadfoot.tex")
\input edrxgac2.tex               % (find-LATEX "edrxgac2.tex")
%
%\usepackage[backend=biber,
%   style=alphabetic]{biblatex}            % (find-es "tex" "biber")
%\addbibresource{catsem-slides.bib}        % (find-LATEX "catsem-slides.bib")
%
% (find-es "tex" "geometry")
\usepackage[a6paper, landscape,
            top=1.5cm, bottom=.25cm, left=1cm, right=1cm, includefoot
           ]{geometry}
%
\begin{document}

%\catcode`\^^J=10
%\directlua{dofile "dednat6load.lua"}  % (find-LATEX "dednat6load.lua")

% %L dofile "edrxtikz.lua"  -- (find-LATEX "edrxtikz.lua")
% %L dofile "edrxpict.lua"  -- (find-LATEX "edrxpict.lua")
% \pu

% «defs»  (to ".defs")
% (find-LATEX "edrx15.sty" "colors-2019")
%\long\def\ColorRed   #1{{\color{Red1}#1}}
%\long\def\ColorViolet#1{{\color{MagentaVioletLight}#1}}
%\long\def\ColorViolet#1{{\color{Violet!50!black}#1}}
%\long\def\ColorGreen #1{{\color{SpringDarkHard}#1}}
%\long\def\ColorGreen #1{{\color{SpringGreenDark}#1}}
%\long\def\ColorGreen #1{{\color{SpringGreen4}#1}}
%\long\def\ColorGray  #1{{\color{GrayLight}#1}}
%\long\def\ColorGray  #1{{\color{black!30!white}#1}}
%\long\def\ColorBrown #1{{\color{Brown}#1}}
%\long\def\ColorBrown #1{{\color{brown}#1}}
%\long\def\ColorOrange#1{{\color{orange}#1}}
%
%\long\def\ColorShort #1{{\color{SpringGreen4}#1}}
%\long\def\ColorLong  #1{{\color{Red1}#1}}
%
%\def\frown{\ensuremath{{=}{(}}}
%\def\True {\mathbf{V}}
%\def\False{\mathbf{F}}
%\def\D    {\displaystyle}

\def\Rd{\ColorRed}
\def\pfo#1{\ensuremath{\mathsf{[#1]}}}

\def\drafturl{http://angg.twu.net/LATEX/2021-1-C2.pdf}
\def\drafturl{http://angg.twu.net/2021.1-C2.html}
\def\draftfooter{\tiny \href{\drafturl}{\jobname{}} \ColorBrown{\shorttoday{} \hours}}



%  _____ _ _   _                               
% |_   _(_) |_| | ___   _ __   __ _  __ _  ___ 
%   | | | | __| |/ _ \ | '_ \ / _` |/ _` |/ _ \
%   | | | | |_| |  __/ | |_) | (_| | (_| |  __/
%   |_| |_|\__|_|\___| | .__/ \__,_|\__, |\___|
%                      |_|          |___/      
%
% «title»  (to ".title")
% (c2m211tfcsp 1 "title")
% (c2m211tfcsa   "title")

\thispagestyle{empty}

\begin{center}

\vspace*{1.2cm}

{\bf \Large Cálculo 2 - 2021.1}

\bsk

Aula 19: os dois TFCs

\bsk

Eduardo Ochs - RCN/PURO/UFF

\url{http://angg.twu.net/2021.1-C2.html}

\end{center}

\newpage

% «diferenca»  (to ".diferenca")
% (c2m211tfcsp 2 "diferenca")
% (c2m211tfcsa   "diferenca")

{\bf A operação ``diferença''}

Def:
%
$$\begin{array}{rcl}
  \Difx{a}{b}{\textsf{expr}} &=&
    (\textsf{expr})[x:=b] -
    (\textsf{expr})[x:=a] \\
  \Difx{a}{b}{f(x)} &=& f(b) - f(a) \\
  \end{array}
$$

Os livros costumam usar a segunda forma.

\bsk

% «exercicio-1»  (to ".exercicio-1")
% (c2m211tfcsp 2 "exercicio-1")
% (c2m211tfcsa   "exercicio-1")

{\bf Exercício 1.}

Expanda e simplifique o máximo possível:

\msk

\begin{tabular}[t]{cl}
a) & $\difx{4}{5}{x^2} $ \\[5pt]
b) & $\difx{5}{4}{x^2} $ \\[5pt]
c) & $\difx{4}{5}{2}   $ \\[5pt]
d) & $\dift{4}{5}{t^2} $ \\[5pt]
e) & $\dift{4}{5}{x^2} $ \\[5pt]
\end{tabular}
\qquad
\begin{tabular}[t]{cl}
f) & $\difx{2}{10}{(x^3-x^2)} $ \\[5pt]
g) & $\difx{2}{10}{x^3} -  \difx{2}{10}{x^2}$ \\[5pt]
h) & $             x^3  - (\difx{2}{10}{x^2})$ \\[5pt]
\end{tabular}

\newpage

% «o-que-vai-ser-o-TFC2»  (to ".o-que-vai-ser-o-TFC2")
% (c2m211tfcsp 3 "o-que-vai-ser-o-TFC2")
% (c2m211tfcsa   "o-que-vai-ser-o-TFC2")

{\bf O que vai ser o TFC2}

%\msk

No MT2 vocês viram que:
%
$$\begin{array}{rcl}
  \Intt{3.5}{6.5}{f(t)}  &=& \Intt{2}{6.5}{f(t)} - \Intt{2}{3.5}{f(t)} \\[5pt]
                         &=& F(6.5) - F(3.5)        \\[5pt]
                         &=& \difx{3.5}{6.5}{F(x)}  \\[10pt]
  \Intt{3.5}{6.5}{F'(t)} &=& \difx{3.5}{6.5}{F(x)}  \\
  \end{array}
$$

\ColorRed{Queremos} generalizar isto para:
%
$$\begin{array}{rcl}
  \Intt{a}{b}{f(t)}  &=& \difx{a}{b}{F(x)}  \\[10pt]
  \Intt{a}{b}{F'(t)} &=& \difx{a}{b}{F(x)}  \\
  \end{array}
$$

\msk

Quais são as condições pra estas últimas igualdades valerem?

\newpage

% «truques-simplificar»  (to ".truques-simplificar")
% (c2m211tfcsp 4 "truques-simplificar")
% (c2m211tfcsa   "truques-simplificar")

{\bf Alguns truques pra simplificar os enunciados}

%\ssk

Vamos começar com algumas suposições que

vão deixar os enunciados mais fáceis...

\msk

$f$, $F$ e $G$ vão ser funções de $\R$ em $\R$,

deriváveis em todo ponto, e $a,b,c,k∈\R$.

\msk

Vamos deixar os casos mais complicados,

em que os domínios não são todo o $\R$

e algumas funções não são deriváveis ou

não são contínuas, pra depois...

\msk

Ou seja, o que você fez no MT2 é um

``caso difícil'', porque usava funções

escada e o domínio era $[0,10]$.

\msk

Isto é um ``caso fácil'': $\D \Intx{a}{b}{x^2} \; = \; \difx{a}{b}{\frac{x^3}{3}}$ \; .


\newpage


\def\Ps  #1{\left( #1 \right) }
\def\ps  #1{     ( #1       ) }
\def\nops#1{       #1         }
\def\righte{\quad\text{e}}

{\bf O TFC1 e algumas consequências dele}

\msk

$
\begin{array}{lrcl}
\text{TFC1:}
  &    \Ps{ F(x)  = \D \Intt{a}{x}{f(t)} }
   &→& \ps{ F'(x) = f(x) }                  \\[20pt]
  &    \Ps{ F(x)  = \D \Intt{a}{x}{f(t)} }
   &→& \pmat{ F(a) = 0 \righte \\ F'(x) = f(x) } \\[10pt]
  &    \Ps{ F(x)  = \D \Intt{a}{x}{f(t)} }
   &↔& \pmat{ F(a) = 0 \righte \\ F'(x) = f(x) } \\[10pt]
  &    \Ps{ F(x)  = \D \Intt{a}{x}{F'(t)} }
   &↔& \pmat{ F(a) = 0 \righte \\ F'(x) = F'(x) } \\[10pt]
  &&↔& \pmat{ F(a) = 0 }                          \\[10pt]
\text{TFC1a:}
  &    \Ps{ F(x)  = \D \Intt{a}{x}{F'(t)} }
   &↔& \pmat{ F(a) = 0 }
\end{array}
$

\newpage

$
\begin{array}{lrcl}
  & \pmat{ F(a) = 0 }
   &→& \Ps{\begin{array}{rcl}
              F(c)  &=& \D \Intt{a}{c}{F'(t)}, \\[10pt]
              F(b)  &=& \D \Intt{a}{b}{F'(t)}, \\[15pt]
              \difx{b}{c}{F(x)} &=& \D \Intt{a}{c}{F'(t)} - \Intt{a}{b}{F'(t)} \\[10pt]
                                &=& \D \Intt{b}{c}{F'(t)} \\[10pt]
           \end{array}}
\end{array}
$

\bsk

$
\begin{array}{lrcl}
  \text{TFC1b:}
  & \pmat{ F(a) = 0 }
   &→& \Ps{\begin{array}{rcl}
              \difx{b}{c}{F(x)} &=& \D \Intt{b}{c}{F'(t)} \\[10pt]
           \end{array}}
\end{array}
$

\newpage

De novo...

\msk

$
\begin{array}{lrcl}
  \text{TFC1b:}
  & \pmat{ F(a) = 0 }
   &→& \Ps{\begin{array}{rcl}
              \difx{b}{c}{F(x)} &=& \D \Intt{b}{c}{F'(t)} \\[10pt]
           \end{array}}
\end{array}
$

\msk

Vamos acrescentar mais uma hipótese: $G(x) = F(x) + k$.

Lembre que tem um `$∀x$' implícito aí: $∀x. \; G(x) = F(x) + k$.

Então, quando $a,b,c$ e $k$ são números reais fixos,

e $F$ e $G$ são funções deriváveis de $\R$ em $\R$

que obedecem $F(a)=0$ e $G(x) = F(x) + k$,

temos isto aqui:
%
$$
\begin{array}{lrcl}
  & \Ps{\begin{array}{rcl}
              \D \Intt{b}{c}{G'(t)} &=& \difx{b}{c}{G(x)}
           \end{array}}
  \\[10pt]
  \text{TFC2:}
    & \Ps{\begin{array}{rcl}
              \D \Intt{a}{b}{F'(t)} &=& \difx{a}{b}{F(x)}
           \end{array}}

\end{array}
$$




\newpage

{\bf A demonstração do TFC2}

(Ainda não digitei)

\newpage

(Ela vai ocupar dois slides)

\newpage

% «exercicio-2»  (to ".exercicio-2")
% (c2m211tfcsp 10 "exercicio-2")
% (c2m211tfcsa    "exercicio-2")

{\bf Exercício 2.}

\def\ddt{\frac{d}{dt}}
\def\ddx{\frac{d}{dx}}
\def\ddy{\frac{d}{dy}}
\def\ddz{\frac{d}{dz}}

Lembre que:
%
$$\begin{array}{rcl}
  f(x) &=& 4 - (x-2)^2 \\
       &=& 4 - (x^2 - 4x + 4) \\
       &=& 4 -  x^2 + 4x - 4) \\
       &=& 4x - x^2           \\
  \ddx(2x^2 - \frac{x^3}{3})        &=& 4x - x^2           \\
  \ddx(2x^2 - \frac{x^3}{3} + 200 ) &=& 4x - x^2           \\
  \end{array}
$$

a) Faça esta substituição aqui:
%
$$[\text{TFC2}] \pmat{F(x) := 2x^2 - \frac{x^3}{3} \\
                      b:=4 \\
                      a:=0 }
$$


\newpage

% «integrar-chutando»  (to ".integrar-chutando")
% (c2m211tfcsp 11 "integrar-chutando")
% (c2m211tfcsa    "integrar-chutando")
% (c2m202tfcp 4 "integrar-chutando")
% (c2m202tfca   "integrar-chutando")

Digamos que queremos ``integrar'' isto:
%
$$\Intx{3}{4}{e^{2x} \cos(e^{2x})} = \Rd{?}$$

\def\TFCDOIS#1#2#3#4{
  \pfo{TFC2} \subst{a:=#2 \\ b:=#1 \\ F(x):=#3 \\ F'(x):=#4}
  & = &
  \left(
  \D \Intx{#1}{#2}{#4} = \Difx{#1}{#2}{\left( #3 \right)}
  \right)
  }

Podemos usar o TFC2 várias vezes, chutando `$a$'s, `$b$'s e `$F$'s...

\msk

$\scalebox{0.80}{$
 \begin{array}{rcl}
  \TFCDOIS{42}{200}{\sen x}{\cos x} \\
  \TFCDOIS{3}{4} {        \sen(e^{2x})} {(2 e^{2x}) \cos(e^{2x}))} \\
  \TFCDOIS{3}{4} {\frac12 \sen(e^{2x})} {   e^{2x}  \cos(e^{2x}))} \\
  \end{array}
  $}
$

\bsk

Ou seja: $\Rd{?} = \Difx{3}{4}{\left( \frac12 \sen(e^{2x}) \right)}$,

que dá pra calcular \Rd{em tempo finito} --- se soubermos

calcular senos e exponenciais em tempo finito.

\newpage

% «exercicio-3»  (to ".exercicio-3")
% (c2m211tfcsp 12 "exercicio-3")
% (c2m211tfcsa    "exercicio-3")
% (c2m202tfcp 5 "exercicio-1")
% (c2m202tfc    "exercicio-1")

Vamos chamar o método do slide anterior de

``integração por TFC2 e chutar-e-testar''.

\msk

{\bf Exercício 3.}

Integre por TFC2 e chutar-e-testar:

\msk

a) $\D \Intx{0}{π/2}{\cos x} = \Rd{?}$

\msk

b) $\D \Intx{0}{π}{\sen x} = \Rd{?}$

\msk

c) $\D \Intx{π/2}{π}{\sen x} = \Rd{?}$

\msk

d) $\D \Intx{5}{6}{\sen(2x + 3)} = \Rd{?}$



\newpage

% «limites-de-integracao»  (to ".limites-de-integracao")
% (c2m211tfcsp 11 "limites-de-integracao")
% (c2m211tfcsa    "limites-de-integracao")

{\bf (Apagando) Os limites de integração}

Quando a gente escreve algo como
%
$$\Intx{42}{99}{x^4} = \difx{42}{99}{\frac{x^5}{5}}$$

esses `42' e `99' são chamados de ``limites de integração''

da integral. Lembre que a gente diz que está integrando

``de 42 até 99'', porque a ordem deles importa --- se a

gente mudasse pra ``de 99 até 42'' isso inverteria o sinal

do resultado. Ah, o 42 e o 99 na barra de diferenção

não têm um nome oficial, então também vou chamá-los de

``limites de integração'' (!!!)...


\newpage

% «limites-de-integracao-2»  (to ".limites-de-integracao-2")
% (c2m211tfcsp 12 "limites-de-integracao-2")
% (c2m211tfcsa    "limites-de-integracao-2")

{\bf (Apagando) Os limites de integração (2)}

Se a gente apagar os limites de integração em todo lugares

na igualdade do slide anterior a gente obtém isso aqui:
%
$$\intx{x^4} \;\; = \;\; {\frac{x^5}{5}}$$

Repare que eu também apaguei a barra de diferença

pra gente não ficar com algo como ``$\frac{x^5}{5}|$''.

\msk

Essa coisa aí em cima --- essa integral sem limites

de integração --- é chamada de {\sl integral indefinida},

e a com limites de integração é a {\sl integral definida}.


\newpage

% «limites-de-integracao-3»  (to ".limites-de-integracao-3")
% (c2m211tfcsp 13 "limites-de-integracao-3")
% (c2m211tfcsa    "limites-de-integracao-3")

{\bf (Apagando) Os limites de integração (3)}

Em muitos casos a gente consegue fazer as contas sem

os limites de integração, com integrais indefinidas,

e colocar os limites de integração só no final.

\msk

Alguns livros começam por integrais indefinidas e só

apresentam as integrais definidas depois... por exemplo:


\ssk

{\footnotesize

\url{http://angg.twu.net/2021.1-C2/martins_martins__secs_4.2-4.4.pdf}

}

\msk

Algumas coisas ficam bem difíceis de entender quando

a gente faz as coisas nessa ordem --- por exemplo integrais

de funções escada e uma regra de integração chamada

``integração por substituição'', que a gente vai ver daqui

a pouco --- então eu prefiro começar por integrais definidas.

\newpage

% «integral-indefinida»  (to ".integral-indefinida")
% (c2m211tfcsp 16 "integral-indefinida")
% (c2m211tfcsa    "integral-indefinida")
% (c2m202tfcp 7 "integral-indefinida")
% (c2m202tfca   "integral-indefinida")
% (c2m201tfc22p 2 "integral-indefinida")
% (c2m201tfc22    "integral-indefinida")
% (find-martinscdipage (+ 10 109) "4.2.2       Integral Indefinida")
% (find-martinscditext (+ 10 109) "4.2.2       Integral Indefinida")

{\bf (Uma definição para) a integral indefinida}

Dê uma olhada na seção 4.2.2 do Martins/Martins.

Eles usam o ``$+ \; C$'' na definição de integral indefinida.

A maioria dos livros faz isso, mas isso gera algumas

ambiguidades que eu prefiro evitar...

\msk

Eu vou usar esta definição aqui para a integral indefinida.

As duas igualdades abaixo são \Rd{exatamente equivalentes}:
%
$$\begin{array}{ccr}
  \displaystyle \int {f(x)} \, dx &=& F(x) \\
                f(x)  &=& \frac{d}{dx} F(x) \\
  \end{array}
$$

Ou seja: pra determinar se uma igualdade da forma

``$\int {f(x)} \, dx \;=\; F(x)$'' é verdade, \Rd{traduza} ela pra forma

da linha de baixo e teste se a igualdade de baixo,

``$f(x) \;=\; \frac{d}{dx} F(x)$'', é verdade.


\newpage

% «exercicio-4»  (to ".exercicio-4")
% (c2m211tfcsp 17 "exercicio-4")
% (c2m211tfcsa    "exercicio-4")

{\bf Exercício 4.}

Quais das igualdades abaixo são verdade?

a) $\intx{\sen x} = \cos x$ 

b) $\intx{\cos x} = \sen x$ 

c) $\intx{x^4} = 5 x^5$ 

d) $\intx{x^4} = \frac15 x^5$ 

\ssk

e) $\intx{x^4} = \frac15 x^5 + 42$ 

\bsk

% «exercicio-5»  (to ".exercicio-5")
% (c2m211tfcsp 17 "exercicio-5")
% (c2m211tfcsa    "exercicio-5")

{\bf Exercício 5 (difícil).}

As duas igualdades em
%
$$42 \;\;=\;\; \intx{0\,} \;\;=\;\; 200$$

são verdadeiras. Porque é que isto não implica em $42 = 200$?


% (c2m202tfca "title")
% (c2m202tfca "title" "Aula 12: o TFC2.")
% (c2m202isa "title")
% (c2m202isa "title" "Aula 14: integração por substituição.")



% (c2m202tfcp 2 "uma-versao-do-TFC2")
% (c2m202tfca   "uma-versao-do-TFC2")
% (c2m211mt2p 6 "gabarito")
% (c2m211mt2a   "gabarito")
% (c2m211prp 45 "TFC1-escadas")
% (c2m211pra    "TFC1-escadas")



\newpage

%\printbibliography

\GenericWarning{Success:}{Success!!!}  % Used by `M-x cv'

\end{document}

%  ____  _             _         
% |  _ \(_)_   ___   _(_)_______ 
% | | | | \ \ / / | | | |_  / _ \
% | |_| | |\ V /| |_| | |/ /  __/
% |____// | \_/  \__,_|_/___\___|
%     |__/                       
%
% «djvuize»  (to ".djvuize")
% (find-LATEXgrep "grep --color -nH --null -e djvuize 2020-1*.tex")

 (eepitch-shell)
 (eepitch-kill)
 (eepitch-shell)
# (find-fline "~/2021.1-C2/")
# (find-fline "~/LATEX/2021-1-C2/")
# (find-fline "~/bin/djvuize")

cd /tmp/
for i in *.jpg; do echo f $(basename $i .jpg); done

f () { rm -fv $1.png $1.pdf; djvuize $1.pdf }
f () { rm -fv $1.png $1.pdf; djvuize WHITEBOARDOPTS="-m 1.0 -f 15" $1.pdf; xpdf $1.pdf }
f () { rm -fv $1.png $1.pdf; djvuize WHITEBOARDOPTS="-m 1.0 -f 30" $1.pdf; xpdf $1.pdf }
f () { rm -fv $1.png $1.pdf; djvuize WHITEBOARDOPTS="-m 1.0 -f 45" $1.pdf; xpdf $1.pdf }
f () { rm -fv $1.png $1.pdf; djvuize WHITEBOARDOPTS="-m 0.5" $1.pdf; xpdf $1.pdf }
f () { rm -fv $1.png $1.pdf; djvuize WHITEBOARDOPTS="-m 0.25" $1.pdf; xpdf $1.pdf }
f () { cp -fv $1.png $1.pdf       ~/2021.1-C2/
       cp -fv        $1.pdf ~/LATEX/2021-1-C2/
       cat <<%%%
% (find-latexscan-links "C2" "$1")
%%%
}

f 20201213_area_em_funcao_de_theta
f 20201213_area_em_funcao_de_x
f 20201213_area_fatias_pizza



%  __  __       _        
% |  \/  | __ _| | _____ 
% | |\/| |/ _` | |/ / _ \
% | |  | | (_| |   <  __/
% |_|  |_|\__,_|_|\_\___|
%                        
% <make>

 (eepitch-shell)
 (eepitch-kill)
 (eepitch-shell)
# (find-LATEXfile "2019planar-has-1.mk")
make -f 2019.mk STEM=2021-1-C2-os-dois-TFCs veryclean
make -f 2019.mk STEM=2021-1-C2-os-dois-TFCs pdf

% Local Variables:
% coding: utf-8-unix
% ee-tla: "c2t"
% ee-tla: "c2m211tfcs"
% End:
