% (find-LATEX "2021-1-C3-gradiente.tex")
% (defun c () (interactive) (find-LATEXsh "lualatex -record 2021-1-C3-gradiente.tex" :end))
% (defun C () (interactive) (find-LATEXsh "lualatex 2021-1-C3-gradiente.tex" "Success!!!"))
% (defun D () (interactive) (find-pdf-page      "~/LATEX/2021-1-C3-gradiente.pdf"))
% (defun d () (interactive) (find-pdftools-page "~/LATEX/2021-1-C3-gradiente.pdf"))
% (defun e () (interactive) (find-LATEX "2021-1-C3-gradiente.tex"))
% (defun o () (interactive) (find-LATEX "2021-1-C3-gradiente.tex"))
% (defun u () (interactive) (find-latex-upload-links "2021-1-C3-gradiente"))
% (defun v () (interactive) (find-2a '(e) '(d)))
% (defun d0 () (interactive) (find-ebuffer "2021-1-C3-gradiente.pdf"))
% (defun cv () (interactive) (C) (ee-kill-this-buffer) (v) (g))
%          (code-eec-LATEX "2021-1-C3-gradiente")
% (find-pdf-page   "~/LATEX/2021-1-C3-gradiente.pdf")
% (find-sh0 "cp -v  ~/LATEX/2021-1-C3-gradiente.pdf /tmp/")
% (find-sh0 "cp -v  ~/LATEX/2021-1-C3-gradiente.pdf /tmp/pen/")
%     (find-xournalpp "/tmp/2021-1-C3-gradiente.pdf")
%   file:///home/edrx/LATEX/2021-1-C3-gradiente.pdf
%               file:///tmp/2021-1-C3-gradiente.pdf
%           file:///tmp/pen/2021-1-C3-gradiente.pdf
% http://angg.twu.net/LATEX/2021-1-C3-gradiente.pdf
% (find-LATEX "2019.mk")
% (find-CN-aula-links "2021-1-C3-gradiente" "3" "c3mgrad" "c3g")
%
% Video:
% (find-ssr-links "c3mgrad" "2021-1-C3-gradiente" "QJj_sFvnoT4")
% (code-video     "c3mgradvideo" "$S/http/angg.twu.net/eev-videos/2021-1-C3-gradiente.mp4")
% (find-c3mgradvideo "0:00")

% (find-ssr-links "c3mgrad2" "2021-1-C3-gradiente-2" "oJOhVCwIBB4")
% (code-video "c3mgrad2video" "$S/http/angg.twu.net/eev-videos/2021-1-C3-gradiente-2.mp4")
% (find-c3mgrad2video "0:00")

% (find-ssr-links "c3mgrad3" "2021-1-C3-gradiente-3")




% «.defs»		(to "defs")
% «.title»		(to "title")
% «.introducao»		(to "introducao")
% «.exercicio-1»	(to "exercicio-1")
% «.exercicio-1-cont»	(to "exercicio-1-cont")
% «.exercicio-2»	(to "exercicio-2")
% «.exercicio-2-cont»	(to "exercicio-2-cont")
% «.exercicio-3»	(to "exercicio-3")
%
% «.djvuize»		(to "djvuize")

\documentclass[oneside,12pt]{article}
\usepackage[colorlinks,citecolor=DarkRed,urlcolor=DarkRed]{hyperref} % (find-es "tex" "hyperref")
\usepackage{amsmath}
\usepackage{amsfonts}
\usepackage{amssymb}
\usepackage{pict2e}
\usepackage[x11names,svgnames]{xcolor} % (find-es "tex" "xcolor")
\usepackage{colorweb}                  % (find-es "tex" "colorweb")
%\usepackage{tikz}
%
% (find-dn6 "preamble6.lua" "preamble0")
%\usepackage{proof}   % For derivation trees ("%:" lines)
%\input diagxy        % For 2D diagrams ("%D" lines)
%\xyoption{curve}     % For the ".curve=" feature in 2D diagrams
%
\usepackage{edrx21}               % (find-LATEX "edrx21.sty")
\input edrxaccents.tex            % (find-LATEX "edrxaccents.tex")
\input edrx21chars.tex            % (find-LATEX "edrx21chars.tex")
\input edrxheadfoot.tex           % (find-LATEX "edrxheadfoot.tex")
\input edrxgac2.tex               % (find-LATEX "edrxgac2.tex")
%
%\usepackage[backend=biber,
%   style=alphabetic]{biblatex}            % (find-es "tex" "biber")
%\addbibresource{catsem-slides.bib}        % (find-LATEX "catsem-slides.bib")
%
% (find-es "tex" "geometry")
\usepackage[a6paper, landscape,
            top=1.5cm, bottom=.25cm, left=1cm, right=1cm, includefoot
           ]{geometry}
%
\begin{document}

%\catcode`\^^J=10
%\directlua{dofile "dednat6load.lua"}  % (find-LATEX "dednat6load.lua")

% %L dofile "edrxtikz.lua"  -- (find-LATEX "edrxtikz.lua")
% %L dofile "edrxpict.lua"  -- (find-LATEX "edrxpict.lua")
% \pu

% «defs»  (to ".defs")
% (find-LATEX "edrx15.sty" "colors-2019")
%\long\def\ColorRed   #1{{\color{Red1}#1}}
%\long\def\ColorViolet#1{{\color{MagentaVioletLight}#1}}
%\long\def\ColorViolet#1{{\color{Violet!50!black}#1}}
%\long\def\ColorGreen #1{{\color{SpringDarkHard}#1}}
%\long\def\ColorGreen #1{{\color{SpringGreenDark}#1}}
%\long\def\ColorGreen #1{{\color{SpringGreen4}#1}}
%\long\def\ColorGray  #1{{\color{GrayLight}#1}}
%\long\def\ColorGray  #1{{\color{black!30!white}#1}}
%\long\def\ColorBrown #1{{\color{Brown}#1}}
%\long\def\ColorBrown #1{{\color{brown}#1}}
%\long\def\ColorOrange#1{{\color{orange}#1}}
%
%\long\def\ColorShort #1{{\color{SpringGreen4}#1}}
%\long\def\ColorLong  #1{{\color{Red1}#1}}
%
%\def\frown{\ensuremath{{=}{(}}}
%\def\True {\mathbf{V}}
%\def\False{\mathbf{F}}
%\def\D    {\displaystyle}

\def\drafturl{http://angg.twu.net/LATEX/2021-1-C3.pdf}
\def\drafturl{http://angg.twu.net/2021.1-C3.html}
\def\draftfooter{\tiny \href{\drafturl}{\jobname{}} \ColorBrown{\shorttoday{} \hours}}



%  _____ _ _   _                               
% |_   _(_) |_| | ___   _ __   __ _  __ _  ___ 
%   | | | | __| |/ _ \ | '_ \ / _` |/ _` |/ _ \
%   | | | | |_| |  __/ | |_) | (_| | (_| |  __/
%   |_| |_|\__|_|\___| | .__/ \__,_|\__, |\___|
%                      |_|          |___/      
%
% «title»  (to ".title")
% (c3mgradp 1 "title")
% (c3mgrada   "title")

\thispagestyle{empty}

\begin{center}

\vspace*{1.2cm}

{\bf \Large Cálculo 3 - 2021.1}

\bsk

Aula 23: o vetor gradiente

\bsk

Eduardo Ochs - RCN/PURO/UFF

\url{http://angg.twu.net/2021.1-C3.html}

\end{center}

\newpage

% «introducao»  (to ".introducao")
% (c3mgradp 2 "introducao")
% (c3mgrada   "introducao")

{\bf Introdução}

Quando nós vimos trajetórias em $\R^2$ nós aprendemos

a ver a derivada de uma trajetória como o vetor velocidade...

\msk

No mini-teste 2 vocês aprenderam a visualizar a derivada

de uma função de $\R^2$ em $\R^2$ como dois vetores ---

veja o vídeo!!!

\msk

Agora nós vamos ver um truque que nos permite interpretar

a derivada de uma superfície como um vetor --- o gradiente.

\msk

Comece lendo as páginas 298 até 302 do capítulo 8 do

Bortolossi. Nós vamos ver as idéias que ele apresenta numa

outra ordem: nós vamos começar desenhando os vetores

gradientes de algumas superfícies simples e verificando no

olhômetro que eles realmente são ortogonais às curvas de

nível e apontam pra direção de maior crescimento da função.


\newpage

% «exercicio-1»  (to ".exercicio-1")
% (c3mgradp 3 "exercicio-1")
% (c3mgrada   "exercicio-1")

{\bf Exercício 1}

Sejam:

$F(x,y) = (x-y)y = xy - y^2$,

$G(x,y) = \frac{1}{10} F(x,y) = \frac{(x-y)y}{10} = (xy - y^2)/10$.

\bsk

Faça os diagramas de numerozinhos das funções

de $\R^2$ em $\R$ abaixo, desenhando os valores delas

nos pontos que têm $x,y∈\{-3,\ldots,3\}$ (49 pontos!):

\begin{tabular}[t]{l}
a) $x-y$          \\
b) $y$            \\
c) $(x-y)y$       \\
d) $((x-y)y)/10$  \\
\end{tabular}
\qquad
\begin{tabular}[t]{l}
e) $G_x(x,y)$     \\
f) $G_y(x,y)$     \\
\end{tabular}

\newpage

% «exercicio-1-cont»  (to ".exercicio-1-cont")
% (c3mgradp 4 "exercicio-1-cont")
% (c3mgrada   "exercicio-1-cont")

{\bf Exercício 1 (cont.)}

\msk

g) Desenhe as curvas de nível da função $G(x,y)$.

\msk

h) Em cada ponto $(x,y)∈\{-3,\ldots,3\}^2$ desenhe o vetor

gradiente $∇G(x,y)$. Mais precisamente: para cada ponto

$(x,y)∈\{-3,\ldots,3\}^2$ desenhe $(x,y) + \VEC{G_x(x,y), G_y(x,y)}$.



\newpage

% «exercicio-2»  (to ".exercicio-2")
% (c3mgradp 5 "exercicio-2")
% (c3mgrada   "exercicio-2")

{\bf Exercício 2.}

Assista este vídeo:

\ssk

{\footnotesize

% http://angg.twu.net/eev-videos/2021-1-C3-gradiente-2.mp4
\url{http://angg.twu.net/eev-videos/2021-1-C3-gradiente-2.mp4}

}

\msk

Sejam:

$z=F(x,y)$,

$(x_0,y_0) = (0,1)$,

$z_0 = z(x_0,y_0)$,

$C = \setofxyst{z(x,y) = z_0}$,

\msk

O conjunto $C$ é formado de duas curvas.

\msk

a) Encontre as funções $h_\text{cima}(x)$ e $h_\text{baixo}(x)$

que percorrem essas duas curvas; ou seja,
%
$$\begin{array}{rcl}
  C &=& \setofxyst{y = h_\text{cima}(x)} \\
    &∪& \setofxyst{y = h_\text{baixo}(x)} \\
  \end{array}
$$



\newpage

% «exercicio-2-cont»  (to ".exercicio-2-cont")
% (c3mgradp 6 "exercicio-2-cont")
% (c3mgrada   "exercicio-2-cont")

{\bf Exercício 2 (cont.)}

\msk

b) Represente elas graficamente.

c) Verifique que $h_\text{cima}(x_0) = y_0$.

d) Calcule $h'_\text{cima}(x)$.

e) Verifique (no olhômetro) se o valor de $h'_\text{cima}(x_0)$ faz sentido.

\msk

f) Seja $\vv = \VEC{1,h'_\text{cima}(x_0)}$. Desenhe $(x_0,y_0) + \vv$ e verifique ---

no olhômetro --- se este vetor $\vv$ é (ou parece ser...) paralelo

ao gráfico da função $h_\text{cima}$ no ponto $(x_0,y_0)$.

\msk

g) Verifique se este vetor $\vv$ é ortogonal ao vetor gradiente

$∇F$, isto é, $∇F(x_0,y_0)$. Aqui você vai fazer a verificação

por contas: dois vetores $\VEC{a,b}$ e $\VEC{c,d}$ são ortogonais

se e só se $\VEC{a,b}·\VEC{c,d} = ac+bd = 0$.


% (find-bortolossi8page (+ -290 291) "8. Derivadas direcionais e o vetor gradiente")
% (find-bortolossi8page (+ -290 296)   "Definição 8.1. (Derivada direcional)")
% (find-bortolossi8page (+ -290 298) "8.2. O vetor gradiente")
% (find-bortolossi8page (+ -290 302)   "direção de maior crescimento")



\newpage

% «exercicio-3»  (to ".exercicio-3")
% (c3mgradp 7 "exercicio-3")
% (c3mgrada   "exercicio-3")

{\bf Exercício 3.}

Assista este vídeo:

\ssk

{\footnotesize

% http://angg.twu.net/eev-videos/2021-1-C3-gradiente-3.mp4
\url{http://angg.twu.net/eev-videos/2021-1-C3-gradiente-3.mp4}

}

\msk

Digamos que $y=y(x)$ e que $z(x,y(x))$ é constante.

Então $\ddx z(x,y(x)) = 0$.

A partir dá pra conseguir uma fórmula que calcula $y_x$

em função de $x$ e de $y$ --- sem precisamos encontrar uma

fórmula para $y(x)$ (!!!)...

\msk

a) Encontre a fórmula para $y_x(x,y)$.

b) Encontre o valor de $y_x$ no ponto $(x,y) = (x_0,y_0) = (0,1)$.

c) Verifique que o vetor $\VEC{1,y_x}$ é ortogonal ao gradiente $∇F$

no ponto $(x_0,y_0)$.




\newpage

%\printbibliography

\GenericWarning{Success:}{Success!!!}  % Used by `M-x cv'

\end{document}

%  ____  _             _         
% |  _ \(_)_   ___   _(_)_______ 
% | | | | \ \ / / | | | |_  / _ \
% | |_| | |\ V /| |_| | |/ /  __/
% |____// | \_/  \__,_|_/___\___|
%     |__/                       
%
% «djvuize»  (to ".djvuize")
% (find-LATEXgrep "grep --color -nH --null -e djvuize 2020-1*.tex")

 (eepitch-shell)
 (eepitch-kill)
 (eepitch-shell)
# (find-fline "~/2021.1-C3/")
# (find-fline "~/LATEX/2021-1-C3/")
# (find-fline "~/bin/djvuize")

cd /tmp/
for i in *.jpg; do echo f $(basename $i .jpg); done

f () { rm -v $1.pdf;  textcleaner -f 50 -o  5 $1.jpg $1.png; djvuize $1.pdf; xpdf $1.pdf }
f () { rm -v $1.pdf;  textcleaner -f 50 -o 10 $1.jpg $1.png; djvuize $1.pdf; xpdf $1.pdf }
f () { rm -v $1.pdf;  textcleaner -f 50 -o 20 $1.jpg $1.png; djvuize $1.pdf; xpdf $1.pdf }

f () { rm -fv $1.png $1.pdf; djvuize $1.pdf }
f () { rm -fv $1.png $1.pdf; djvuize WHITEBOARDOPTS="-m 1.0 -f 15" $1.pdf; xpdf $1.pdf }
f () { rm -fv $1.png $1.pdf; djvuize WHITEBOARDOPTS="-m 1.0 -f 30" $1.pdf; xpdf $1.pdf }
f () { rm -fv $1.png $1.pdf; djvuize WHITEBOARDOPTS="-m 1.0 -f 45" $1.pdf; xpdf $1.pdf }
f () { rm -fv $1.png $1.pdf; djvuize WHITEBOARDOPTS="-m 0.5" $1.pdf; xpdf $1.pdf }
f () { rm -fv $1.png $1.pdf; djvuize WHITEBOARDOPTS="-m 0.25" $1.pdf; xpdf $1.pdf }
f () { cp -fv $1.png $1.pdf       ~/2021.1-C3/
       cp -fv        $1.pdf ~/LATEX/2021-1-C3/
       cat <<%%%
% (find-latexscan-links "C3" "$1")
%%%
}

f 20201213_area_em_funcao_de_theta
f 20201213_area_em_funcao_de_x
f 20201213_area_fatias_pizza



%  __  __       _        
% |  \/  | __ _| | _____ 
% | |\/| |/ _` | |/ / _ \
% | |  | | (_| |   <  __/
% |_|  |_|\__,_|_|\_\___|
%                        
% <make>

 (eepitch-shell)
 (eepitch-kill)
 (eepitch-shell)
# (find-LATEXfile "2019planar-has-1.mk")
make -f 2019.mk STEM=2021-1-C3-gradiente veryclean
make -f 2019.mk STEM=2021-1-C3-gradiente pdf

% Local Variables:
% coding: utf-8-unix
% ee-tla: "c3g"
% ee-tla: "c3mgrad"
% End:
