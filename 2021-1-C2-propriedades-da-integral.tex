% (find-LATEX "2021-1-C2-propriedades-da-integral.tex")
% (defun c () (interactive) (find-LATEXsh "lualatex -record 2021-1-C2-propriedades-da-integral.tex" :end))
% (defun C () (interactive) (find-LATEXsh "lualatex 2021-1-C2-propriedades-da-integral.tex" "Success!!!"))
% (defun D () (interactive) (find-pdf-page      "~/LATEX/2021-1-C2-propriedades-da-integral.pdf"))
% (defun d () (interactive) (find-pdftools-page "~/LATEX/2021-1-C2-propriedades-da-integral.pdf"))
% (defun e () (interactive) (find-LATEX "2021-1-C2-propriedades-da-integral.tex"))
% (defun o () (interactive) (find-LATEX "2021-1-C2-propriedades-da-integral.tex"))
% (defun l () (interactive) (find-LATEX "2021-1-C2-critical-points.lua"))
% (defun u () (interactive) (find-latex-upload-links "2021-1-C2-propriedades-da-integral"))
% (defun v () (interactive) (find-2a '(e) '(d)))
% (defun d0 () (interactive) (find-ebuffer "2021-1-C2-propriedades-da-integral.pdf"))
% (defun cv () (interactive) (C) (ee-kill-this-buffer) (v) (g))
%          (code-eec-LATEX "2021-1-C2-propriedades-da-integral")
% (find-pdf-page   "~/LATEX/2021-1-C2-propriedades-da-integral.pdf")
% (find-sh0 "cp -v  ~/LATEX/2021-1-C2-propriedades-da-integral.pdf /tmp/")
% (find-sh0 "cp -v  ~/LATEX/2021-1-C2-propriedades-da-integral.pdf /tmp/pen/")
%     (find-xournalpp "/tmp/2021-1-C2-propriedades-da-integral.pdf")
%   file:///home/edrx/LATEX/2021-1-C2-propriedades-da-integral.pdf
%               file:///tmp/2021-1-C2-propriedades-da-integral.pdf
%           file:///tmp/pen/2021-1-C2-propriedades-da-integral.pdf
% http://angg.twu.net/LATEX/2021-1-C2-propriedades-da-integral.pdf
% (find-LATEX "2019.mk")
% (find-CN-aula-links "2021-1-C2-propriedades-da-integral" "2" "c2m211pr" "c2pr")
%
% Video:
% (find-ssr-links "c2m211pr" "2021-1-C2-propriedades-da-integral" "ORfsWiwelV8")
% (code-video     "c2m211prvideo" "$S/http/angg.twu.net/eev-videos/2021-1-C2-propriedades-da-integral.mp4")
% (find-c2m211prvideo "0:00" "4/agosto")

% (find-ssr-links "c2m211prs" "2021-1-C2-propriedades-da-integral-2" "MmlQTtH5jFo")
% (code-video "c2m211prsvideo" "$S/http/angg.twu.net/eev-videos/2021-1-C2-propriedades-da-integral-2.mp4")
% (find-c2m211prsvideo "0:00" "11/agosto")

% (find-ssr-links "c2m211prex9" "2021-1-C2-propriedades-da-integral-3" "J97x7MNpr90")
% (code-video "c2m211prex9video" "$S/http/angg.twu.net/eev-videos/2021-1-C2-propriedades-da-integral-3.mp4")
% (find-c2m211prex9video "0:00" "12/agosto")
% (find-c2m211prex9video "0:25" "TFC1")
% (find-c2m211prex9video "0:45" "resolvendo uma EDO")
% (find-c2m211prex9video "1:10" "no MT1 eu dava essa função f(x)")
% (find-c2m211prex9video "2:00" "F(2)=0")



% «.defs»			(to "defs")
% «.title»			(to "title")
% «.introducao»			(to "introducao")
% «.introducao-2»		(to "introducao-2")
% «.introducao-3»		(to "introducao-3")
% «.introducao-4»		(to "introducao-4")
% «.parabola-complicada»	(to "parabola-complicada")
% «.largura-particao»		(to "largura-particao")
% «.particoes-cvmf»		(to "particoes-cvmf")
% «.particoes-cvmf-2»		(to "particoes-cvmf-2")
% «.seq-parts-n-imp»		(to "seq-parts-n-imp")
% «.seq-parts-n-imp-2»		(to "seq-parts-n-imp-2")
% «.seq-parts-n-imp-3»		(to "seq-parts-n-imp-3")
% «.figuras-teorema-horrivel»	(to "figuras-teorema-horrivel")
% «.dirichlet-incl»		(to "dirichlet-incl")
% «.pierluigi»			(to "pierluigi")
% «.aditividade-d»		(to "aditividade-d")
% «.integrais-de-fs-constantes»	(to "integrais-de-fs-constantes")
% «.mudando-finitos-pontos»	(to "mudando-finitos-pontos")
% «.integrando-escadas»		(to "integrando-escadas")
% «.mudando-limites»		(to "mudando-limites")
% «.exercicio-1»		(to "exercicio-1")
% «.mt1-semestre-passado»	(to "mt1-semestre-passado")
% «.exercicio-3»		(to "exercicio-3")
% «.retangulos-degenerados»	(to "retangulos-degenerados")
% «.retangulos-degenerados-2»	(to "retangulos-degenerados-2")
% «.exercicio-4»		(to "exercicio-4")
% «.dicas-plotar»		(to "dicas-plotar")
% «.exercicio-5»		(to "exercicio-5")
% «.exercicio-5-cont»		(to "exercicio-5-cont")
% «.exercicio-6»		(to "exercicio-6")
% «.exercicio-6-cont»		(to "exercicio-6-cont")
% «.exercicio-7»		(to "exercicio-7")
% «.exercicio-8»		(to "exercicio-8")
% «.TFC1-escadas»		(to "TFC1-escadas")
% «.exercicio-9»		(to "exercicio-9")
%
% «.djvuize»			(to "djvuize")

\documentclass[oneside,12pt]{article}
\usepackage[colorlinks,citecolor=DarkRed,urlcolor=DarkRed]{hyperref} % (find-es "tex" "hyperref")
\usepackage{amsmath}
\usepackage{amsfonts}
\usepackage{amssymb}
\usepackage{pict2e}
\usepackage[x11names,svgnames]{xcolor} % (find-es "tex" "xcolor")
\usepackage{colorweb}                  % (find-es "tex" "colorweb")
%\usepackage{tikz}
%
% (find-dn6 "preamble6.lua" "preamble0")
%\usepackage{proof}   % For derivation trees ("%:" lines)
%\input diagxy        % For 2D diagrams ("%D" lines)
%\xyoption{curve}     % For the ".curve=" feature in 2D diagrams
%
\usepackage{edrx15}               % (find-LATEX "edrx15.sty")
\input edrxaccents.tex            % (find-LATEX "edrxaccents.tex")
\input edrx21chars.tex            % (find-LATEX "edrx21chars.tex")
\input edrxheadfoot.tex           % (find-LATEX "edrxheadfoot.tex")
\input edrxgac2.tex               % (find-LATEX "edrxgac2.tex")
%
%\usepackage[backend=biber,
%   style=alphabetic]{biblatex}            % (find-es "tex" "biber")
%\addbibresource{catsem-slides.bib}        % (find-LATEX "catsem-slides.bib")
%
% (find-es "tex" "geometry")
\usepackage[a6paper, landscape,
            top=1.5cm, bottom=.25cm, left=1cm, right=1cm, includefoot
           ]{geometry}
%
\begin{document}

\catcode`\^^J=10
\directlua{dofile "dednat6load.lua"}  % (find-LATEX "dednat6load.lua")

%L dofile "edrxtikz.lua"  -- (find-LATEX "edrxtikz.lua")
%L dofile "edrxpict.lua"  -- (find-LATEX "edrxpict.lua")
\pu

% «defs»  (to ".defs")
% (find-LATEX "edrx15.sty" "colors-2019")
\long\def\ColorRed   #1{{\color{Red1}#1}}
\long\def\ColorViolet#1{{\color{MagentaVioletLight}#1}}
\long\def\ColorViolet#1{{\color{Violet!50!black}#1}}
\long\def\ColorGreen #1{{\color{SpringDarkHard}#1}}
\long\def\ColorGreen #1{{\color{SpringGreenDark}#1}}
\long\def\ColorGreen #1{{\color{SpringGreen4}#1}}
\long\def\ColorGray  #1{{\color{GrayLight}#1}}
\long\def\ColorGray  #1{{\color{black!30!white}#1}}
\long\def\ColorBrown #1{{\color{Brown}#1}}
\long\def\ColorBrown #1{{\color{brown}#1}}
\long\def\ColorOrange#1{{\color{orange}#1}}

\long\def\ColorShort #1{{\color{SpringGreen4}#1}}
\long\def\ColorLong  #1{{\color{Red1}#1}}

\def\frown{\ensuremath{{=}{(}}}
\def\True {\mathbf{V}}
\def\False{\mathbf{F}}
\def\D    {\displaystyle}

\def\Intover     #1#2{\overline {∫}_{#1}#2\,dx}
\def\Intunder    #1#2{\underline{∫}_{#1}#2\,dx}
\def\Intoverunder#1#2{\Intover{#1}{#2} - \Intunder{#1}{#2}}

\def\Intxover     #1#2#3{\overline {∫}_{x=#1}^{x=#2}#3\,dx}
\def\Intxunder    #1#2#3{\underline{∫}_{x=#1}^{x=#2}#3\,dx}

\def\Intoverunder   #1#2{\overline{\underline{∫}}_{#1}      #2\,dx}
\def\Intxoverunder#1#2#3{\overline{\underline{∫}}_{x=#1}^{x=#2} #3\,dx}

\def\sumiN#1{\sum_{i=1}^N #1 (b_i-a_i)}
\def\mname#1{\text{[#1]}}

\def\drafturl{http://angg.twu.net/LATEX/2021-1-C2.pdf}
\def\drafturl{http://angg.twu.net/2021.1-C2.html}
\def\draftfooter{\tiny \href{\drafturl}{\jobname{}} \ColorBrown{\shorttoday{} \hours}}


% (c2m211somas2p 37 "exercicio-16-defs")
% (c2m211somas2a    "exercicio-16-defs")

% (find-LATEX "2021-1-C2-critical-points.lua" "Approxer-tests")
%L dofile     "2021-1-C2-critical-points.lua"
%L appr = Approxer {
%L     f      = f_do_slide_8,
%L     allcps = {3,8},
%L     a      = 2,
%L     b      = 10,
%L     N      = 4,
%L     method = "supin",
%L     what   = "ac",
%L   }
\pu

\long\def\ColorUpperA#1{{\color{red!20!white}#1}}
\long\def\ColorUpperB#1{{\color{Gold1!20!white}#1}}
\long\def\ColorUpperC#1{{\color{Green1!20!white}#1}}
\long\def\ColorUpperD#1{{\color{Blue1!20!white}#1}}
\long\def\ColorLowerA#1{{\color{red!80!white}#1}}
\long\def\ColorLowerB#1{{\color{Gold1!80!white}#1}}
\long\def\ColorLowerC#1{{\color{Green1!80!white}#1}}
\long\def\ColorLowerD#1{{\color{Blue1!80!white}#1}}
\long\def\ColorRealInt#1{{\color{Purple0!90!white}#1}}

\def\fwithapprs#1{%
  \vcenter{\hbox{%
    \beginpicture(0,0)(11,7)
    \pictgrid%
    #1%
    \pictpiecewise{(0,3)--(3,6)--(8,1)--(11,4)}%
    \pictaxes%
    \end{picture}%
  }}}

%L pol = function (x,y,dx,dy)
%L     local x0, y0, x1, y1 = x,y,x+dx,y+dy
%L     return pformat("\\polygon*(%s,%s)(%s,%s)(%s,%s)(%s,%s)",
%L                    x0,y0, x0,y1, x1,y1, x1,y0)
%L   end
\pu





%  _____ _ _   _                               
% |_   _(_) |_| | ___   _ __   __ _  __ _  ___ 
%   | | | | __| |/ _ \ | '_ \ / _` |/ _` |/ _ \
%   | | | | |_| |  __/ | |_) | (_| | (_| |  __/
%   |_| |_|\__|_|\___| | .__/ \__,_|\__, |\___|
%                      |_|          |___/      
%
% «title»  (to ".title")
% (c2m211prp 1 "title")
% (c2m211pra   "title")

\thispagestyle{empty}

\begin{center}

\vspace*{1.2cm}

{\bf \Large Cálculo 2 - 2021.1}

\bsk

Aula 15: Propriedades da integral

\bsk

Eduardo Ochs - RCN/PURO/UFF

\url{http://angg.twu.net/2021.1-C2.html}

\end{center}

\newpage

% «introducao»  (to ".introducao")
% (c2m211prp 1 "introducao")
% (c2m211pra   "introducao")

{\bf Introdução}

No último PDF vocês aprenderam a visualizar coisas como:
%
\unitlength=7pt
%
$$\begin{array}{l}
  \Intoverunder{[2,10]_{2^1}}{f(x)} \\[10pt]
  = \Intoverunder{\{2,6,10\}}{f(x)} \\[10pt]
  =    \Intover {\{2,6,10\}}{f(x)}     \\
  - \; \Intunder{\{2,6,10\}}{f(x)}     \\[10pt]
  =    (\sup(F([2,6])) - \inf(F([2,6]))(6 - 2)       \\
  + \; (\sup(F([6,10])) - \inf(F([6,10]))(10 - 6) \\[10pt]
  = \fwithapprs{{%
     \color{Orange1}%
     \expr{pol(2,3, 4,3)}%
     \expr{pol(6,1, 4,2)}%
    }}
  \end{array}
$$

\newpage

% «introducao-2»  (to ".introducao-2")
% (c2m211prp 3 "introducao-2")
% (c2m211pra   "introducao-2")

{\bf Introdução (2)}

\def\Iou#1{\Intoverunder {[a,b]_{2^{#1}}} {f(x)}}

...e vocês aprenderam a visualizar isto aqui,

para várias `$f(x)$'s diferentes:
%
$$\lim_{k→∞} \left( \Iou{k} \right)$$

e viram que existem funções não integráveis,

como a função de Dirichlet, e viram argumentos

olhométricos que devem ter convencido vocês de que

isto aqui é verdade:

\begin{quotation}

{\bf Corolário 11.} Seja $f: [a, b] → \R$ contínua

com a possível exceção de um número finito

de pontos e limitada. Então, $f$ é integrável.

\end{quotation}


\newpage

% «introducao-3»  (to ".introducao-3")
% (c2m211prp 4 "introducao-3")
% (c2m211pra   "introducao-3")

{\bf Introdução (3)}

\ssk

Esse ``Corolário 11'' é da página 9 das notas

do Pierluigi Beneveri. Dê uma olhada:

\ssk

{\footnotesize

% (find-books "__analysis/__analysis.el" "beneveri")
% (find-pierluigipage  9    "Corolário 11")
% (find-pierluigitext  9    "Corolário 11")
\url{https://www.ime.usp.br/~pluigi/registro-MAT121-15.pdf#page=9}

}

\bsk

A abordagem dele é bem diferente da nossa ---

quase todos os exercícios dele são da forma

``demonstre a afirmação tal''... mas eu vou

pedir pra vocês consultarem as notas dele

de vez em quando, e vou tentar complementar

as notas dele mostrando como visualizar certas

coisas que ele afirma.

\newpage

% «introducao-4»  (to ".introducao-4")
% (c2m211prp 5 "introducao-4")
% (c2m211pra   "introducao-4")

{\bf Introdução (4)}

\ssk

\def\Iou#1{\Intoverunder {[a,b]_{2^{#1}}} {f(x)}}
\def\Iou#1{\Intoverunder {[0,8]_{2^{#1}}} {f(x)}}

Dá pra gente se convencer de que o Corolário 11 é verdade

olhando um exemplo ``que seja suficientemente não-trivial''...

Tente visualizar $\lim_{k→∞} \Iou{k}$ para a função abaixo.

Você vai ver que em torno dos pontos de descontinuidade

os retângulos continuam com a mesma altura mas se tornam

cada vez mais finos, e fora desses lugares os retângulos

se tornam cada vez mais baixos.

% «parabola-complicada»  (to ".parabola-complicada")
% (c2m211prp 5 "parabola-complicada")
% (c2m211pra   "parabola-complicada")
% (c2m211somas2p 43 "claramente-integravel-p")
% (c2m211somas2a    "claramente-integravel-p")

% (find-LATEX "2021-1-C2-critical-points.lua" "f_parabola_preferida")
% (find-LATEX "2021-1-C2-critical-points.lua" "Piecewisify-tests")
%
%L f_parabola_complicada = function (x)
%L     if x <= 4 then return f_parabola_preferida(x) end
%L     if x <  5 then return 5 - x end
%L     if x <  6 then return 7 - x end
%L     if x <  7 then return 3 end
%L     if x == 7 then return 4 end
%L     return 0.5
%L   end
%L f_funcao_complicada = f_parabola_complicada
%L
%L pwi = Piecewisify.new(f_funcao_complicada, seq(0, 4, 0.25), 5, 6, 7)
\pu

\def\fwithapprs#1{%
  \vcenter{\hbox{%
    \beginpicture(0,0)(8,5)
    \pictgrid%
    #1%
    \pictaxes%
    \expr{pwi:pw(0, 8)}
    \end{picture}%
  }}}

\unitlength=16pt

$$f(x) =
 \fwithapprs{%
  %\ColorOrange{%
  %\expr{pwi:pol(0, 8, "*")}
  %}
  }
$$



\newpage

\unitlength=25pt

\def\Iou#1{\Intoverunder {[a,b]_{2^{#1}}} {f(x)}}
\def\Iou#1{\Intoverunder {[0,8]_{2^{#1}}} {f(x)}}

\def\FIG#1{%
  \fwithapprs{%
  \ColorOrange{%
  \expr{pwi:rects(Partition.new(0, 8):splitn(2^#1), "sup", "inf")}
  }}}
\def\FFIG#1{\Iou{#1} \;\; = \;\; \FIG{#1}}

$$\FFIG2$$
\newpage

$$\FFIG3$$
\newpage

$$\FFIG4$$
\newpage

$$\FFIG5$$
\newpage

$$\FFIG6$$
\newpage

$$\FFIG7$$
\newpage

% «largura-particao»  (to ".largura-particao")
% (c2m211prp 12 "largura-particao")
% (c2m211pra    "largura-particao")

{\bf A largura de uma partição}

\ssk

Def: a \ColorRed{largura} de uma partição $P$ é

a ``largura de seu \ColorRed{maior} subintervalo''.

A notação para a largura de uma partição $P$ é $||P||$.

Exemplo: $||\{2, 2.5, 3, 7, 7.5\}|| = 4$.

Formalmente:
%
$$||P|| = \sup(\setofst {b_i-a_i} {i∈\{1,\ldots,N\}})$$

No exemplo:
%
$$\begin{array}{rcl}
  ||\{2, 2.5, 3, 7, 7.5\}|| &=& \sup(\{0.5, 0.5, 4, 0.5\}) \\
                            &=& \sup(\{0.5, 4 \}) \\
                            &=& 4. \\
  \end{array}
$$

\newpage

% «particoes-cvmf»  (to ".particoes-cvmf")
% (c2m211prp 13 "particoes-cvmf")
% (c2m211pra    "particoes-cvmf")

{\bf Partições cada vez mais finas}

\ssk

Def: $(P_1, P_2, P_3, \ldots)$ é uma sequência de partições

\ColorRed{cada vez mais finas} do intervalo $[a,b]$ se:

\ssk

1) Cada $P_i$ é uma partição de $[a,b]$, e

2) $\lim_{i→∞} ||P_i|| = 0$.

\msk

Vamos usar esta notação (estranha!):
%
$$(P_1, P_2, P_3, \ldots) \dashrightarrow [a,b]$$

pra indicar que $(P_1, P_2, P_3, \ldots)$ é uma sequência

de partições cada vez mais finas do intervalo $[a,b]$.

\msk

Lembre que cada $P_i$ é um conjunto finito,

mas $[a,b]$ é um conjunto infinito.


\newpage

% «particoes-cvmf-2»  (to ".particoes-cvmf-2")
% (c2m211prp 14 "particoes-cvmf-2")
% (c2m211pra    "particoes-cvmf-2")

{\bf Partições cada vez mais finas (2)}

\ssk

Exemplo óbvio:
%
$$\begin{array}{rcl}
  ([a,b]_{2^1},
   [a,b]_{2^2},
   [a,b]_{2^3},
   \ldots) &\dashrightarrow& [a,b]
  \end{array}
$$

Um exemplo menos óbvio:
%
$$\begin{array}{rcl}
  ([a,b]_1,
   [a,b]_2,
   [a,b]_3,
   \ldots) &\dashrightarrow& [a,b], \\
  ([0,6]_1,
   [0,6]_2,
   [0,6]_3,
   \ldots) &\dashrightarrow& [0,6], \\
  (\{0,6\},
   \{0,3,6\},
   \{0,2,4,6\},
   \ldots) &\dashrightarrow& [0,6], \\
  \end{array}
$$

Note que o subintervalo $[2,4]$ da partição $[0,6]_3 = \{0,2,4,6\}$ contém

uma parte do subintervalo $[0,3]$ da partição $[0,6]_2 = \{0,3,6\}$ e

uma parte do subintervalo $[3,6]$ da partição $[0,6]_2 = \{0,3,6\}$...

\newpage


% «seq-parts-n-imp»  (to ".seq-parts-n-imp")
% (c2m211prp 12 "seq-parts-n-imp")
% (c2m211pra    "seq-parts-n-imp")

{\bf A sequência de partições não importa}

\ssk

Lembra que nós definimos ``$f$ é integrável em $[a,b]$'' usando

esta sequência de partições cada vez mais finas de $[a,b]$:
%
$$([a,b]_{2^1},
   [a,b]_{2^2},
   [a,b]_{2^3},
   \ldots) \dashrightarrow [a,b]
$$

Lembrando a definição:

$f$ é integrável em $[a,b]$ se e só se:
%
$$\lim_{k→∞} \mname{sup}_{[a,b]_{2^k}} =
  \lim_{k→∞} \mname{inf}_{[a,b]_{2^k}}
$$

Vamos fazer uma versão mais flexível dessa definição...

$f$ é \ColorRed{$(P_1, P_2, P_3, \ldots)$-integrável} em $[a,b]$ se e só se:
%
$$\lim_{k→∞} \mname{sup}_{P_k} =
  \lim_{k→∞} \mname{inf}_{P_k}
$$



\newpage

% «seq-parts-n-imp-2»  (to ".seq-parts-n-imp-2")
% (c2m211prp 16 "seq-parts-n-imp-2")
% (c2m211pra    "seq-parts-n-imp-2")

{\bf A sequência de partições não importa (2)}

{\bf Teorema (horrível).}

Sejam
%
$$\begin{array}{rcl}
  (P_1, P_2, P_3, \ldots) &\dashrightarrow& [a,b], \\
  (Q_1, Q_2, Q_3, \ldots) &\dashrightarrow& [a,b]
  \end{array}
$$

duas sequências de partições cada vez mais finas

do intervalo $[a,b]$. Então 
``$(P_1, P_2, P_3, \ldots)$-integrabilidade''

e ``$(Q_1, Q_2, Q_3, \ldots)$-integrabilidade'' são equivalentes, no

seguinte sentido:

\msk

Pegue \ColorRed{qualquer} função $f:[a,b]→\R$.

Então $f$ é $(P_1, P_2, P_3, \ldots)$-integrável em $[a,b]$

se e só se $f$ é $(Q_1, Q_2, Q_3, \ldots)$-integrável em $[a,b]$, e:
%
$$\begin{array}{rcl}
  \lim_{k→∞} \mname{sup}_{P_k} &=& \lim_{k→∞} \mname{sup}_{Q_k} \\
  \lim_{k→∞} \mname{inf}_{P_k} &=& \lim_{k→∞} \mname{inf}_{Q_k} \\
  \end{array}
$$

\newpage

% «seq-parts-n-imp-3»  (to ".seq-parts-n-imp-3")
% (c2m211prp 17 "seq-parts-n-imp-3")
% (c2m211pra    "seq-parts-n-imp-3")

{\bf A sequência de partições não importa (3)}

A demonstração do Teorema Horrível é bem trabalhosa, e é

bem difícil visualizar o que certos passos dela querem dizer...

\msk


Alguns textos, como o livro dos dois Martins/Martins, as notas

de aula da Cristiane Hernández, e a página da Wikipedia sobre

Somas de Riemann usam o Teorema Horrível implicitamente,

sem nem contarem quanta sujeira eles estão escondendo

debaixo do tapete.

% (find-books "__analysis/__analysis.el" "hernandez")
% (find-hernandezpage (+ 10 4) "sobre todas as possíveis partições")
% (find-hernandeztext (+ 10 4) "sobre todas as possíveis partições")
%
% (find-books "__analysis/__analysis.el" "martins-martins")
% (find-martinscdipage (+ 10 203) "6.5        Integral Definida")
% (find-martinscditext (+ 10 203) "6.5        Integral Definida")
%
% https://pt.wikipedia.org/wiki/Soma_de_Riemann

\msk

Quando nós usamos a sequência 

$([a,b]_{2^1},
   [a,b]_{2^2},
   [a,b]_{2^3},
   \ldots) \dashrightarrow [a,b]$

as nossas aproximação pelos métodos do sup e do inf melhoram

a cada passo, mas se usamos outras sequências, como

$([a,b]_1,
   [a,b]_2,
   [a,b]_3,
   \ldots) \dashrightarrow [a,b]$

os resultados podem oscilar bastante antes de convergir...


\newpage

% «figuras-teorema-horrivel»  (to ".figuras-teorema-horrivel")
% (c2m211prp 18 "figuras-teorema-horrivel")
% (c2m211pra    "figuras-teorema-horrivel")

\unitlength=25pt

\def\Iou#1{\Intoverunder {[a,b]_{2^{#1}}} {f(x)}}
\def\Iou#1{\Intoverunder {[0,8]_   {#1} } {f(x)}}

\def\FIG#1{%
  \fwithapprs{%
  \ColorOrange{%
  \expr{pwi:rects(Partition.new(0, 8):splitn( #1), "sup", "inf")}
  }}}
\def\FFIG#1{\Iou{#1} \;\; = \;\; \FIG{#1}}

$$\FFIG4$$
\newpage

$$\FFIG5$$
\newpage

$$\FFIG6$$
\newpage

$$\FFIG7$$
\newpage

$$\FFIG8$$
\newpage

$$\FFIG9$$
\newpage

% «dirichlet-incl»  (to ".dirichlet-incl")
% (c2m211prp 24 "dirichlet-incl")
% (c2m211pra    "dirichlet-incl")

{\bf Relembrando funções não integráveis...}

\ssk

Sejam $g(x) = \begin{cases}
              x     & \text{quando $x∈\Q$}, \\
              x + 1 & \text{quando $x∈\R∖\Q$} \\
              \end{cases}$

e $d_k = \D \Intoverunder{[0,1]_{2^k}}{g(x)}$.


\msk

%L dirichlet_incl_Q = SetL.new()
%L dirichlet_incl   = function (x)
%L     if dirichlet_incl_Q:has(x) then return x end
%L     return x+1
%L   end
%L for _,x in ipairs(seq(0, 1, 1/64)) do
%L   dirichlet_incl_Q:add(x)
%L end
%L pwid = Piecewisify.new(dirichlet_incl, seq(0, 1, 1/64))
\pu
%
\def\gwithapprs#1{%
  \vcenter{\hbox{%
    \beginpicture(0,0)(1,2)
    \pictgrid%
    #1%
    \pictpiecewise{(0,0)--(1,1) (0,1)--(1,2)}%
    \pictaxes%
    \end{picture}%
  }}}
\def\dirichletincldk#1{%
  \expr{pwid:rects(Partition.new(0, 1):splitn(2^#1), "sup", "inf")}}
\def\gwithapprsdk#1{\gwithapprs{\ColorOrange{\dirichletincldk{#1}}}}


Então a sequência $(d_0, d_1, d_2, d_3, d_4, d_5, \ldots)$

pode ser representada \ColorRed{graficamente} como:
%
$$\unitlength=25pt
  \left(
  \gwithapprsdk{0},
  \gwithapprsdk{1},
  \gwithapprsdk{2},
  \gwithapprsdk{3},
  \gwithapprsdk{4},
  \gwithapprsdk{5},
  \ldots
  \right)
$$

e se interpretarmos cada $d_k$ como um número

temos $\lim_{k→∞} d_k = 1$.

\newpage

% «pierluigi»  (to ".pierluigi")
% (c2m211prp 25 "pierluigi")
% (c2m211pra    "pierluigi")

{\bf Introdução às propriedade da integral do Pierluigi}

\ssk

A partir de agora eu vou tentar convencer vocês

de que algumas propriedades da integral são verdade,

mas ao invés de demonstrá-las eu vou mostrar o que elas

``querem dizer'' graficamente e geometricamente, usando

exemplos. Eu vou tentar \ColorRed{complementar} as explicações

das páginas 6 até 8 das notas do Pierluigi Beneveri,

\ssk

{\footnotesize

% https://www.ime.usp.br/~pluigi/registro-MAT121-15.pdf#page=6
\url{https://www.ime.usp.br/~pluigi/registro-MAT121-15.pdf#page=6}

}

\ssk

...mas tente ler as notas dele, e considere que as explicações

``de verdade'' estão lá, e não aqui.


\newpage

% «aditividade-d»  (to ".aditividade-d")
% (c2m211prp 26 "aditividade-d")
% (c2m211pra    "aditividade-d")
% (find-books "__analysis/__analysis.el" "beneveri")
% (find-pierluigipage 7 "Proposição 8 (Propriedade 4: aditividade a respeito do domínio")
% (find-pierluigitext 7 "Proposição 8 (Propriedade 4: aditividade a respeito do domínio")

{\bf Aditividade no domínio}

Leia a ``Proposição 8'' do Pierluigi --- que ele chama de

``Propriedade 4: aditividade a respeito do domínio''.

Exemplo:

\unitlength=8pt


$$\Intx{0}{4}{f(x)} + 
  \Intx{4}{7}{f(x)} =
  \Intx{0}{7}{f(x)}
$$
%
$$
  \fwithapprs{%
  \ColorOrange{%
  \expr{pwi:pol(0, 4, "*")}
  }}
  +
  \fwithapprs{%
  \ColorOrange{%
  \expr{pwi:pol(4, 7, "*")}
  }}
  =
  \fwithapprs{%
  \ColorOrange{%
  \expr{pwi:pol(0, 7, "*")}
  }}
$$

\newpage

% «integrais-de-fs-constantes»  (to ".integrais-de-fs-constantes")
% (c2m211prp 27 "integrais-de-fs-constantes")
% (c2m211pra    "integrais-de-fs-constantes")
% (find-pierluigipage 5 "Exercício 18. Seja f (x) = c constante")
% (find-pierluigitext 5 "Exercício 18. Seja f (x) = c constante")

{\bf Integrais de funções constantes (e áreas negativas)}

Leia o ``Exercício 18'' do Pierluigi,

que diz que $\Intx{a}{b}{c} = c(b-a)$.

Isto vale também para $c$ negativo...

\msk

Leia a ``Definição 9'' do Pierluigi na página 8 da notas dele.

Ele usa um truque parecido com o que usamos aqui,

\ssk

{\footnotesize

\url{http://angg.twu.net/LATEX/2021-1-C2-somas-2.pdf#page=22}

}

\ssk

em que \ColorRed{redefinimos} (temporariamente!) os termos ``acima''

e ``abaixo'' pra adequá-los a conceitos matemáticos que

queríamos ter como pronunciar em português...

Então:
%
%L pwic = Piecewisify.new(function () return -3 end)
\pu
%
$$\unitlength=7.5pt
  %
  \Area \left(
    \vcenter{\hbox{%
      \beginpicture(-1,-4)(5,1)
      \pictgrid%
      \ColorOrange{\expr{pwic:pol(2, 4, "*")}}%
      \pictpiecewise{(-1,-3)--(5,-3)}%
      \pictaxes%
      \end{picture}%
    }}
    \right)
  =
  \Intx{2}{4}{-3}
  =
  (-3)(4-2)
  =
  -6
$$

% (find-pierluigipage 8 "Definição 9" "ideia intuitiva de área")
% (find-pierluigitext 8 "Definição 9" "ideia intuitiva de área")
% (c2m211somas2p 22 "sups-e-infs-em-portugues")
% (c2m211somas2a    "sups-e-infs-em-portugues")

\newpage

% «mudando-finitos-pontos»  (to ".mudando-finitos-pontos")
% (c2m211prp 28 "mudando-finitos-pontos")
% (c2m211pra    "mudando-finitos-pontos")

{\bf Mudando um número finito de pontos}

Exemplo: digamos que $f(x)$ seja a nossa parábola preferida,

e $g(x)$ seja esta ``parabola com anteninhas'':
%
$$g(x) = 
  \begin{cases}
  f(x) & \text{quando $x≠1$ e  $x≠3$}, \\
  4    & \text{quando $x=1$ ou $x=3$}. \\
  \end{cases}
$$

Então:
%
%L f_parabola_com_antenas = function (x)
%L     if x == 1 then return 4 end
%L     if x == 3 then return 4 end
%L     return f_parabola_preferida(x)
%L   end
%L f_funcao_complicada = f_parabola_complicada
%L
%L pwip  = Piecewisify.new(f_parabola_preferida,   seq(0, 4, 0.25))
%L pwipa = Piecewisify.new(f_parabola_com_antenas, seq(0, 4, 0.25), 1, 2, 3)
\pu
%
$$\unitlength=15pt
  %
  \Area \left(
    \vcenter{\hbox{%
      \beginpicture(0,0)(4,4)
      \pictgrid%
      \ColorOrange{\expr{pwip:pol(0, 4, "*")}}%
      \expr{pwip:pw(0, 4)}%
      \pictaxes%
      \end{picture}%
    }}
    \right)
  %
  =
  %
  \Area \left(
    \vcenter{\hbox{%
      \beginpicture(0,0)(4,4)
      \pictgrid%
      \ColorOrange{\expr{pwipa:pol(0, 4, "*")}}%
      \expr{pwipa:pw(0, 4)}%
      \pictaxes%
      \end{picture}%
    }}
    \right)
$$



\newpage

% «integrando-escadas»  (to ".integrando-escadas")
% (c2m211prp 29 "integrando-escadas")
% (c2m211pra    "integrando-escadas")

{\bf Integrando funções escada}

%L f_funcao_complicada = function (x)
%L     if x <=  2 then return 0 end
%L     if x <=  4 then return 2 end
%L     if x <=  6 then return 4 end
%L     if x <=  8 then return 6 end
%L     if x <= 10 then return 4 end
%L     if x <= 12 then return 2 end
%L     return 0
%L   end
%L
%L pwi = Piecewisify.new(f_funcao_complicada, 2, 3, 4, 6, 8, 10, 12, 14)
\pu

\def\fwithapprs#1{%
  \vcenter{\hbox{%
    \beginpicture(0,0)(14,7)
    \pictgrid%
    #1%
    \pictaxes%
    \expr{pwi:pw(0, 14)}
    \end{picture}%
  }}}
\def\fwithapprsc#1{\fwithapprs{\ColorOrange{#1}}}

\unitlength=6.5pt

Digamos que $f(x)$ seja esta função aqui:

$$f(x) =
  \fwithapprsc{%\expr{pwi:pol(0, 14, "*")}
              }
$$

Então:
%
$$\Intx{3}{7}{f(x)} =
  \fwithapprsc{\expr{pwi:pol(3, 7, "*")}}
  = \pmat{2·(4-3) \\
     + \; 4·(6-4) \\
     + \; 6·(7-6) },
$$
$$\Intx{5}{11}{f(x)} =
  \fwithapprsc{\expr{pwi:pol(5, 11, "*")}}
  = \pmat{4·(6-5) \\
     + \; 6·(8-6) \\
     + \; 4·(10-8) \\
     + \; 2·(11-10) },
$$

\newpage

% «mudando-limites»  (to ".mudando-limites")
% (c2m211prp 30 "mudando-limites")
% (c2m211pra    "mudando-limites")

{\bf Mudando os limites de integração }

Em $\Intx{3}{7}{f(x)}$ o intervalo de integração ia de $x=3$ até $x=7$,

e pra expressar $\Intx{3}{7}{f(x)}$ como uma soma de retângulos nós

precisamos de:

um retângulo com $y=2$ indo de $x=3$ até $x=4$,

um retângulo com $y=4$ indo de $x=4$ até $x=6$,

um retângulo com $y=6$ indo de $x=6$ até $x=7$...


\msk

e pra expressar $\Intx{5}{11}{f(x)}$ como uma soma de retângulos nós

precisamos de:

um retângulo com $y=4$ indo de $x=5$ até $x=6$,

um retângulo com $y=6$ indo de $x=6$ até $x=8$,

um retângulo com $y=4$ indo de $x=8$ até $x=10$,

um retângulo com $y=2$ indo de $x=10$ até $x=11$...

\msk

{\bf O número de intervalos e retângulos é diferente!!!!!!}


\newpage

% «exercicio-1»  (to ".exercicio-1")
% (c2m211prp 31 "exercicio-1")
% (c2m211pra    "exercicio-1")

{\bf Exercício 1.}

Seja $f(x)$ a função definida dois slides atrás.

Em cada um dos itens abaixo represente graficamente a

integral --- lembre que integrais são áreas!!! --- e expresse

ela como uma soma, como o que fizemos dois slides atrás.

\msk

{\bf MUITO, MUITO, MUITO IMPORTANTE:}

{\bf O NÚMERO DE INTERVALOS PODE MUDAR}

{\bf DE UM ITEM PRO OUTRO!!!}

\msk

a) $\Intx{3}{5}{f(x)}$

\ssk

b) $\Intx{3}{6.5}{f(x)}$

\ssk

c) $\Intx{3}{9}{f(x)}$

\ssk

d) $\Intx{4.5}{9}{f(x)}$

\ssk

e) $\Intx{7.5}{9}{f(x)}$


\newpage

% «mt1-semestre-passado»  (to ".mt1-semestre-passado")
% (c2m211prp 32 "mt1-semestre-passado")
% (c2m211pra    "mt1-semestre-passado")

{\bf O mini-teste 1 do semestre passado}

\ssk

Dê uma olhada nele:

{\footnotesize

% (c2m202mt1p 4 "miniteste-funcao")
% (c2m202mt1a   "miniteste-funcao")
%    http://angg.twu.net/LATEX/2020-2-C2-MT1.pdf#page=4
\url{http://angg.twu.net/LATEX/2020-2-C2-MT1.pdf#page=4}

}

\ssk

Nos próximos exercícios nós vamos resolver uns problemas

bem parecidos com as questões desse mini-teste, mas

vamos fazer eles bem passo a passo.


\newpage

{\bf Exercício 2.}

Sejam:
%
$$f(x) = 
  \unitlength=12pt
  \vcenter{\hbox{%
    \beginpicture(0,-3)(9,3)
    \pictgrid%
    \pictaxes%
    \pictpiecewise{(0,0)--(1,0)o
                   (1,1)c--(2,1)o
                   (2,2)c--(3,2)o
                   (3,1)c--(4,1)o
                   (4,0)c--(5,0)o
                   (5,-1)c--(6,-1)o
                   (6,-2)c--(7,-2)o
                   (7,-1)c--(8,-1)o
                   (8,0)c--(9,0)
                  }%
    \celllower=2.5pt%
    \def\cellfont{\scriptsize}%
    \put(1,-0.5){\cell{1}}%
    \put(2,-0.5){\cell{2}}%
    \put(3,-0.5){\cell{3}}%
    \put(4,-0.5){\cell{4}}%
    \put(5,-0.5){\cell{5}}%
    \put(6,-0.5){\cell{6}}%
    \put(7,-0.5){\cell{7}}%
    \put(8,-0.5){\cell{8}}%
    \end{picture}%
  }}
$$

e $F(b) = \Intx{0}{b}{f(x)}$.

\msk

a) Tente visualizar $F(2.5)$ e $F(3)$ de cabeça, sem desenhar nada.

b) Tente visualizar $F(3) - F(2.5)$ de cabeça, sem desenhar nada.

c) A diferença $F(3) - F(2.5)$ é um retângulo. Diga a largura da base

dele, a altura dele, e a área dele. Faça tudo de cabeça.

d) Visualize $F(3.5) - F(2.5)$ de cabeça e veja que não é um retângulo.


\newpage

% «exercicio-3»  (to ".exercicio-3")
% (c2m211prp 34 "exercicio-3")
% (c2m211pra    "exercicio-3")

{\bf Exercício 3.}

Sejam:
%
$$f(x) = 
  \unitlength=12pt
  \vcenter{\hbox{%
    \beginpicture(0,-3)(9,3)
    \pictgrid%
    \pictaxes%
    \pictpiecewise{(0,0)--(1,0)o
                   (1,1)c--(2,1)o
                   (2,2)c--(3,2)o
                   (3,1)c--(4,1)o
                   (4,0)c--(5,0)o
                   (5,-1)c--(6,-1)o
                   (6,-2)c--(7,-2)o
                   (7,-1)c--(8,-1)o
                   (8,0)c--(9,0)
                  }%
    \celllower=2.5pt%
    \def\cellfont{\scriptsize}%
    \put(1,-0.5){\cell{1}}%
    \put(2,-0.5){\cell{2}}%
    \put(3,-0.5){\cell{3}}%
    \put(4,-0.5){\cell{4}}%
    \put(5,-0.5){\cell{5}}%
    \put(6,-0.5){\cell{6}}%
    \put(7,-0.5){\cell{7}}%
    \put(8,-0.5){\cell{8}}%
    \end{picture}%
  }}
$$

e $F(b) = \Intx{0}{b}{f(x)}$.

\msk

Calcule as áreas das figuras abaixo de cabeça quando elas forem

retângulos. Quando a figura não for um retângulo basta dizer

``não é um retângulo''.

\ssk

\begin{tabular}{l}
a) $F(2.6)-F(2.5)$ \\
b) $F(3.9)-F(3.8)$ \\
c) $F(4.0)-F(3.9)$ \\
\end{tabular}
\qquad
\begin{tabular}{l}
d) $F(4.1)-F(4.0)$ \\
e) $F(5.3)-F(5.2)$ \\
f) $F(6.1)-F(5.9)$ \\
\end{tabular}


\newpage

% «retangulos-degenerados»  (to ".retangulos-degenerados")
% (c2m211prp 35 "retangulos-degenerados")
% (c2m211pra    "retangulos-degenerados")

{\bf Retângulos degenerados}

\ssk

Várias pessoas ficaram em dúvida sobre se os retângulos com

altura 0 do exercício 3 deveriam ser considerados retângulos

ou não... eu tinha certeza que sim, mas aí a gente foi olhar

a definição de retângulo na Wikipedia e a gente descobriu

que segundo a definição usual de retângulo eles \ColorRed{não são}

considerados retângulos... $\frown$

\msk

...mas eles são \ColorRed{retângulos degenerados}. Links:

\ssk

{\scriptsize

% https://pt.wikipedia.org/wiki/Degenera%C3%A7%C3%A3o_(matem%C3%A1tica)
\url{https://pt.wikipedia.org/wiki/Degenera%C3%A7%C3%A3o_(matem%C3%A1tica)}

% https://pt.wikipedia.org/wiki/Ret%C3%A2ngulo
\url{https://pt.wikipedia.org/wiki/Ret%C3%A2ngulo}

% https://en.wikipedia.org/wiki/Rectangle
\url{https://en.wikipedia.org/wiki/Rectangle}

}

\newpage

% «retangulos-degenerados-2»  (to ".retangulos-degenerados-2")
% (c2m211prp 36 "retangulos-degenerados-2")
% (c2m211pra    "retangulos-degenerados-2")

{\bf Retângulos degenerados (2)}

\ssk

Trechos principais:

\begin{quotation}

Em matemática, um caso degenerado é um caso li\-mite no qual uma
classe de objeto altera sua natureza para aproximar-se muito a um
objeto de outra classe, normal\-mente, mais simples.

\msk

(...)

\msk

Um segmento é uma forma degenerada de um retângulo se este tem um dos
lados de comprimento zero.

\end{quotation}





\newpage

% «exercicio-4»  (to ".exercicio-4")
% (c2m211prp 37 "exercicio-4")
% (c2m211pra    "exercicio-4")

{\bf Exercício 4.}

Sejam:
%
$$f(x) = 
  \unitlength=10pt
  \vcenter{\hbox{%
    \beginpicture(0,-3)(9,3)
    \pictgrid%
    \pictaxes%
    \pictpiecewise{(0,0)--(1,0)o
                   (1,1)c--(2,1)o
                   (2,2)c--(3,2)o
                   (3,1)c--(4,1)o
                   (4,0)c--(5,0)o
                   (5,-1)c--(6,-1)o
                   (6,-2)c--(7,-2)o
                   (7,-1)c--(8,-1)o
                   (8,0)c--(9,0)
                  }%
    \celllower=2.5pt%
    \def\cellfont{\scriptsize}%
    \put(1,-0.5){\cell{1}}%
    \put(2,-0.5){\cell{2}}%
    \put(3,-0.5){\cell{3}}%
    \put(4,-0.5){\cell{4}}%
    \put(5,-0.5){\cell{5}}%
    \put(6,-0.5){\cell{6}}%
    \put(7,-0.5){\cell{7}}%
    \put(8,-0.5){\cell{8}}%
    \end{picture}%
  }}
$$

e $F(b) = \Intx{0}{b}{f(x)}$.

\ssk

Agora você vai fazer um gráfico da função $F(b)$. O \ColorRed{primeiro passo}

é plotar nesse gráfico os pontos $(b,F(b))$ com $b∈\{0, 0.5, 1,
\ldots, 9\}$.

\ColorRed{Faça isso direto no gráfico, fazendo todas as contas de cabeça.}

O truque é que $(0,F(0)) = (0,0)$ e é fácil encontrar cada ponto

novo a partir do anterior... por exemplo, $F(3.5)-F(3)=0.5$,

então pra passar de $(3,F(3))$ pra $(3.5,F(3.5))$ você anda 0.5

pra direita e 0.5 pra cima.



\newpage

% «dicas-plotar»  (to ".dicas-plotar")
% (c2m211prp 38 "dicas-plotar")
% (c2m211pra    "dicas-plotar")

{\bf Dicas sobre como plotar os pontos do exercício 4}

\newpage

% «exercicio-5»  (to ".exercicio-5")
% (c2m211prp 39 "exercicio-5")
% (c2m211pra    "exercicio-5")

{\bf Exercício 5.}

A gente ainda não tem o gráfico da função $F(b)$, só alguns 

pontos dele... qual é o jeito certo de ligar esses pontos?

Vamos começar desenhando mais pontos desse gráfico.

No exercício 4 você desenhou uma série de pontos do gráfico

de $F(b)$: os pontos correspondentes a $b∈\{0, 0.5, 1, 1.5, \ldots, 9\}$.

A distância horizontal entre cada ponto desses e o seguinte

era 0.5; agora nós vamos acrescentar mais pontos a esse

gráfico, até a gente ter todos os pontos correspondentes

a $b∈\{0, 0.1, 0.2, \ldots, 9\}$, com espaçamento horizontal 0.1

entre cada ponto e o seguinte...

\newpage

% «exercicio-5-cont»  (to ".exercicio-5-cont")
% (c2m211prp 40 "exercicio-5-cont")
% (c2m211pra    "exercicio-5-cont")

{\bf Exercício 5 (cont.)}

...descubra como fazer isso. É possível que nos primeiros

pontos você vá ter que fazer algumas contas --- faça todas

de cabeça!!! --- mas assim que você descobrir os padrões

você vai ser capaz de desenhar todos os pontos muito rápido.

\msk

{\bf IMPORTANTE:} faça esse gráfico com mais pontos

como se você estivesse fazendo ele pra um ``leitor

que seja muito amigo seu'' que não vai contar quantos

pontos você desenhou entre, por exemplo, $x=3$ e $x=4$.

Se você desenhar só 7 pontos ali ao invés de 9 (ou ao

invés de 10, ou de 11... depende do jeito de contar)

esse seu amigo não vai notar. Lembre destes truques:

\ssk

{\footnotesize

% (c2m211somas2p 48)
%    http://angg.twu.net/LATEX/2021-1-C2-somas-2.pdf#page=48
\url{http://angg.twu.net/LATEX/2021-1-C2-somas-2.pdf#page=48}

% (c2m211somas24p 34)
%    http://angg.twu.net/LATEX/2021-1-C2-somas-2-4.pdf#page=34
\url{http://angg.twu.net/LATEX/2021-1-C2-somas-2-4.pdf#page=34}

}

% (c2m211somas2p 48 "dirichlet-3")
% (c2m211somas2a    "dirichlet-3")
% (c2m211somas24p 34 "que-finja-ter-infinitas")
% (c2m211somas24a    "que-finja-ter-infinitas")


\newpage

% «exercicio-6»  (to ".exercicio-6")
% (c2m211prp 41 "exercicio-6")
% (c2m211pra    "exercicio-6")

{\bf Exercício 6.}

Agora vamos fazer algo mais chique.

Em Cálculo 1 você deve ter visto muitos argumentos

que começavam com ``considere que $ε$ é um número real

muito pequeno''. Esses argumentos eram sempre meio

informais, e eles às vezes até usavam passos como

``então $ε^2$ é desprezível''... e depois eles eram formalizados

usando limites.

\msk

Ainda usando a $f(x)$ e a $F(x)$ dos slides anteriores,

calcule o resultado das expressões abaixo considerando

que $ε$ é um real positivo muito pequeno. Quase todos

os seus resultados vão dar expressões contendo $ε$.

% (c2m211somas2p 44 "funcoes-escada")
% (c2m211somas2a    "funcoes-escada")

\msk

a) $F(1.5+ε)$, $F(1.5+ε) - F(1.5)$, $\frac{F(1.5+ε) - F(1.5)}{ε}$

b) $F(2.5+ε)$, $F(2.5+ε) - F(2.5)$, $\frac{F(2.5+ε) - F(2.5)}{ε}$

\newpage

% «exercicio-6-cont»  (to ".exercicio-6-cont")
% (c2m211prp 42 "exercicio-6-cont")
% (c2m211pra    "exercicio-6-cont")

{\bf Exercício 6 (cont.)}

\msk

c) $F(3.5+ε)$, $F(3.5+ε) - F(3.5)$, $\frac{F(3.5+ε) - F(3.5)}{ε}$

d) $F(3.2+ε)$, $F(3.2+ε) - F(3.2)$, $\frac{F(3.2+ε) - F(3.2)}{ε}$

e) $F(3.9+ε)$, $F(3.9+ε) - F(3.9)$, $\frac{F(3.9+ε) - F(3.9)}{ε}$

\bsk

E agora lembre da definição de derivada.

Para cada $b_0$ no domínio da $F$ temos:
%
$$F'(b_0) = \lim_{ε→0} \frac{F(b_0+ε) - F(b_0)}{ε}$$

Use isto pra calcular:

\begin{tabular}{l}
f) $F'(1.5)$, \\
g) $F'(2.5)$, \\
h) $F'(3.5)$, \\
\end{tabular}
\quad
\begin{tabular}{l}
i) $F'(3.2)$, \\
j) $F'(3.9)$, \\
k) $F'(4.5)$, \\
\end{tabular}
\quad
\begin{tabular}{l}
l) $F'(5.2)$, \\
m) $F'(6.3)$, \\
n) $F'(2.0)$. \\
\end{tabular}


\newpage

% «exercicio-7»  (to ".exercicio-7")
% (c2m211prp 43 "exercicio-7")
% (c2m211pra    "exercicio-7")

{\bf Exercício 7.}

(Vai ser sobre derivadas pela esquerda...)

\newpage

% «exercicio-8»  (to ".exercicio-8")
% (c2m211prp 44 "exercicio-8")
% (c2m211pra    "exercicio-8")

{\bf Exercício 8.}

(Vai ser sobre a continuidade da $F$...)



\newpage

% «TFC1-escadas»  (to ".TFC1-escadas")
% (c2m211prp 45 "TFC1-escadas")
% (c2m211pra    "TFC1-escadas")

{\bf O TFC1 (para funções escada)}

\ssk

Nos exercícios 6, 7 e 8 você descobriu --- num caso particular,

mas dá pra provar que isso vale sempre --- que quando $f$ é

uma função escada e
%
$$\begin{array}{rcl}
  F(b) &=& \D \Intx{a}{b}{f(x)} \,, \quad \text{ou:} \\[10pt]
  F(x) &=& \D \Intt{a}{x}{f(t)} \\
  \end{array}
$$

então:

1) $F(a) = 0$,

2) a função $F$ é contínua,

3) $F'(x) = f(x)$ em todo $x$ onde a derivada $f'(x)$ existe...

\msk

Ou seja, dá pra encontrar a função $F$ \ColorRed{resolvendo uma EDO}.

\newpage

{\bf O TFC1 para funções escada: um método}

\ssk

Quando a função $f$ é uma função escada simples --- como as

que estamos vendo nos exercícios, ou como as do MT1 do

semestre passado --- a gente consegue encontrar a função

$F(x) = \Intt{a}{x}{f(t)}$ desenhando ela no gráfico...

\msk

O método é o seguinte. Vou mostrar ele pra função $G$ do MT1,

mas chamando ela de $F$. As figuras estão no próximo slide.

Repare que na função $G$ do MT1 tínhamos $a=2$...

\newpage

\unitlength=10pt

$$
 f(x) \;\; = \;\;
 \vcenter{\hbox{%
 \beginpicture(0,-4)(10,4)
   \pictgrid%
   \pictpiecewise{(0,1)--(1,1)o
                  (1,2)c--(2,2)o
                  (2,3)c--(3,3)o
                  (3,-3)c--(4,-3)o
                  (4,-2)c--(5,-2)o
                  (5,-1)c--(6,-1)o
                  (6,0)c--(7,0)o
                  (7,1)c--(8,1)o
                  (8,2)c--(9,2)o
                  (9,3)c--(10,3)o
                  }%
   \pictaxes%
 \end{picture}%
 }}
 \qquad
 \begin{array}{rcl}
  \end{array}
$$

$$
  F(x)
  \;\; = \;\;
  \D \Intt{2}{x}{f(t)}
  \;\; = \;\;
   \vcenter{\hbox{%
   \beginpicture(0,-4)(10,4)
     \pictgrid%
     \pictpiecewise{          (0,-3)--(1,-2)--(2,0)--(3,3)--%
       (4,0)--(5,-2)--(6,-3)--(7,-3)--(8,-2)--(9,0)--(10,3)
                    }%
     \pictaxes%
   \end{picture}%
  }}
$$

\newpage

{\bf O TFC1 para funções escada: um método (2)}

\ssk

Sabemos que $F(2)=0$.

Então o gráfico da $F$ passa pelo ponto $(2,F(2)) = (2,0)$.

Para todo $x∈(2,3)$ temos $f(x)=3$,

então para todo $x∈(2,3)$ temos $F'(x)=3$,

e então entre $x=2$ e $x=3$ o gráfico da $F$ é um

segmento de reta com coeficiente angular 3.

Esse segmento termina no ponto $(3,3)$.

\msk

O gráfico da $F$ passa pelo ponto $(3,3)$.

Entre $x=3$ e $x=4$ o gráfico da $F$ é um

segmento de reta com coeficiente angular -3.

Esse segmento termina no ponto $(4,0)$.

\msk

Entre $x=4$ e $x=5$ o gráfico da $F$ é um

segmento de reta com coeficiente angular -2...


\newpage

% «exercicio-9»  (to ".exercicio-9")
% (c2m211prp 46 "exercicio-9")
% (c2m211pra    "exercicio-9")

{\bf Exercício 9.}

\ssk

Faça as questões a e b do MT1 do semestre passado.

\ssk

Tem link pro MT1 do semestre passado no slide 32,

e as dicas pra este exercício estão neste vídeo:

\ssk

{\scriptsize

\url{http://angg.twu.net/eev-videos/2021-1-C2-propriedades-da-integral-3.mp4}

\url{https://www.youtube.com/watch?v=J97x7MNpr90 YT}

}

\bsk
\bsk

Obs: a gente ainda não viu como interpretar integrais

``com os limites de integração na ordem errada'', como:
%
$$\Intx{0}{-2}{f(x)}$$

% (c2m202mt1a "title")
% (c2m202mt1a "title" "Mini-teste 1")
% (c2m202mt1p 5 "miniteste-questoes")
% (c2m202mt1a   "miniteste-questoes")

Vamos ver em breve! Prepare-se!



\newpage




% \newpage
% 
% area
% 
% soma horizontal
% 
% soma vertical
% 
% multiplicacao por constante h
% 
% multiplicacao por constante v

% (find-books "__analysis/__analysis.el" "beneveri")

\newpage

%\printbibliography

\GenericWarning{Success:}{Success!!!}  % Used by `M-x cv'

\end{document}

%  ____  _             _         
% |  _ \(_)_   ___   _(_)_______ 
% | | | | \ \ / / | | | |_  / _ \
% | |_| | |\ V /| |_| | |/ /  __/
% |____// | \_/  \__,_|_/___\___|
%     |__/                       
%
% «djvuize»  (to ".djvuize")
% (find-LATEXgrep "grep --color -nH --null -e djvuize 2020-1*.tex")

 (eepitch-shell)
 (eepitch-kill)
 (eepitch-shell)
# (find-fline "~/2021.1-C2/")
# (find-fline "~/LATEX/2021-1-C2/")
# (find-fline "~/bin/djvuize")

cd /tmp/
for i in *.jpg; do echo f $(basename $i .jpg); done

f () { rm -fv $1.png $1.pdf; djvuize $1.pdf }
f () { rm -fv $1.png $1.pdf; djvuize WHITEBOARDOPTS="-m 1.0" $1.pdf; xpdf $1.pdf }
f () { rm -fv $1.png $1.pdf; djvuize WHITEBOARDOPTS="-m 0.5" $1.pdf; xpdf $1.pdf }
f () { rm -fv $1.png $1.pdf; djvuize WHITEBOARDOPTS="-m 0.25" $1.pdf; xpdf $1.pdf }
f () { cp -fv $1.png $1.pdf       ~/2021.1-C2/
       cp -fv        $1.pdf ~/LATEX/2021-1-C2/
       cat <<%%%
% (find-latexscan-links "C2" "$1")
%%%
}

f 20201213_area_em_funcao_de_theta
f 20201213_area_em_funcao_de_x
f 20201213_area_fatias_pizza





%  __  __       _        
% |  \/  | __ _| | _____ 
% | |\/| |/ _` | |/ / _ \
% | |  | | (_| |   <  __/
% |_|  |_|\__,_|_|\_\___|
%                        
% <make>

 (eepitch-shell)
 (eepitch-kill)
 (eepitch-shell)
# (find-LATEXfile "2019planar-has-1.mk")
make -f 2019.mk STEM=2021-1-C2-propriedades-da-integral veryclean
make -f 2019.mk STEM=2021-1-C2-propriedades-da-integral pdf

% Local Variables:
% coding: utf-8-unix
% ee-tla: "c2pr"
% ee-tla: "c2m211pr"
% End:
