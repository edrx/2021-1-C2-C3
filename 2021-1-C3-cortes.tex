% (find-LATEX "2021-1-C3-cortes.tex")
% (defun c () (interactive) (find-LATEXsh "lualatex -record 2021-1-C3-cortes.tex" :end))
% (defun C () (interactive) (find-LATEXsh "lualatex 2021-1-C3-cortes.tex" "Success!!!"))
% (defun D () (interactive) (find-pdf-page      "~/LATEX/2021-1-C3-cortes.pdf"))
% (defun d () (interactive) (find-pdftools-page "~/LATEX/2021-1-C3-cortes.pdf"))
% (defun e () (interactive) (find-LATEX "2021-1-C3-cortes.tex"))
% (defun o () (interactive) (find-LATEX "2021-1-C3-cortes.tex"))
% (defun u () (interactive) (find-latex-upload-links "2021-1-C3-cortes"))
% (defun v () (interactive) (find-2a '(e) '(d)))
% (defun d0 () (interactive) (find-ebuffer "2021-1-C3-cortes.pdf"))
% (defun cv () (interactive) (C) (ee-kill-this-buffer) (v) (g))
%          (code-eec-LATEX "2021-1-C3-cortes")
% (find-pdf-page   "~/LATEX/2021-1-C3-cortes.pdf")
% (find-sh0 "cp -v  ~/LATEX/2021-1-C3-cortes.pdf /tmp/")
% (find-sh0 "cp -v  ~/LATEX/2021-1-C3-cortes.pdf /tmp/pen/")
%     (find-xournalpp "/tmp/2021-1-C3-cortes.pdf")
%   file:///home/edrx/LATEX/2021-1-C3-cortes.pdf
%               file:///tmp/2021-1-C3-cortes.pdf
%           file:///tmp/pen/2021-1-C3-cortes.pdf
% http://angg.twu.net/LATEX/2021-1-C3-cortes.pdf
% (find-LATEX "2019.mk")
% (find-CN-aula-links "2021-1-C3-cortes" "3" "c3m211c" "c3c")
%
% Video:
% (find-ssr-links "c3m211c" "2021-1-C3-cortes")
% (code-video     "c3m211cvideo" "$S/http/angg.twu.net/eev-videos/2021-1-C3-cortes.mp4")
% (find-c3m211cvideo "0:00")

% «.defs»	(to "defs")
% «.title»	(to "title")
%
% «.djvuize»	(to "djvuize")

\documentclass[oneside,12pt]{article}
\usepackage[colorlinks,citecolor=DarkRed,urlcolor=DarkRed]{hyperref} % (find-es "tex" "hyperref")
\usepackage{amsmath}
\usepackage{amsfonts}
\usepackage{amssymb}
\usepackage{pict2e}
\usepackage[x11names,svgnames]{xcolor} % (find-es "tex" "xcolor")
\usepackage{colorweb}                  % (find-es "tex" "colorweb")
%\usepackage{tikz}
%
% (find-dn6 "preamble6.lua" "preamble0")
%\usepackage{proof}   % For derivation trees ("%:" lines)
%\input diagxy        % For 2D diagrams ("%D" lines)
%\xyoption{curve}     % For the ".curve=" feature in 2D diagrams
%
\usepackage{edrx15}               % (find-LATEX "edrx15.sty")
\input edrxaccents.tex            % (find-LATEX "edrxaccents.tex")
\input edrxchars.tex              % (find-LATEX "edrxchars.tex")
\input edrxheadfoot.tex           % (find-LATEX "edrxheadfoot.tex")
\input edrxgac2.tex               % (find-LATEX "edrxgac2.tex")
%
%\usepackage[backend=biber,
%   style=alphabetic]{biblatex}            % (find-es "tex" "biber")
%\addbibresource{catsem-slides.bib}        % (find-LATEX "catsem-slides.bib")
%
% (find-es "tex" "geometry")
\usepackage[a6paper, landscape,
            top=1.5cm, bottom=.25cm, left=1cm, right=1cm, includefoot
           ]{geometry}
%
\begin{document}

%\catcode`\^^J=10
%\directlua{dofile "dednat6load.lua"}  % (find-LATEX "dednat6load.lua")

% %L dofile "edrxtikz.lua"  -- (find-LATEX "edrxtikz.lua")
% %L dofile "edrxpict.lua"  -- (find-LATEX "edrxpict.lua")
% \pu

% «defs»  (to ".defs")
% (find-LATEX "edrx15.sty" "colors-2019")
\long\def\ColorRed   #1{{\color{Red1}#1}}
\long\def\ColorViolet#1{{\color{MagentaVioletLight}#1}}
\long\def\ColorViolet#1{{\color{Violet!50!black}#1}}
\long\def\ColorGreen #1{{\color{SpringDarkHard}#1}}
\long\def\ColorGreen #1{{\color{SpringGreenDark}#1}}
\long\def\ColorGreen #1{{\color{SpringGreen4}#1}}
\long\def\ColorGray  #1{{\color{GrayLight}#1}}
\long\def\ColorGray  #1{{\color{black!30!white}#1}}
\long\def\ColorBrown #1{{\color{Brown}#1}}
\long\def\ColorBrown #1{{\color{brown}#1}}
\long\def\ColorOrange#1{{\color{orange}#1}}

\long\def\ColorShort #1{{\color{SpringGreen4}#1}}
\long\def\ColorLong  #1{{\color{Red1}#1}}

\def\frown{\ensuremath{{=}{(}}}
\def\True {\mathbf{V}}
\def\False{\mathbf{F}}
\def\D    {\displaystyle}

\def\drafturl{http://angg.twu.net/LATEX/2021-1-C3.pdf}
\def\drafturl{http://angg.twu.net/2021.1-C3.html}
\def\draftfooter{\tiny \href{\drafturl}{\jobname{}} \ColorBrown{\shorttoday{} \hours}}



%  _____ _ _   _                               
% |_   _(_) |_| | ___   _ __   __ _  __ _  ___ 
%   | | | | __| |/ _ \ | '_ \ / _` |/ _` |/ _ \
%   | | | | |_| |  __/ | |_) | (_| | (_| |  __/
%   |_| |_|\__|_|\___| | .__/ \__,_|\__, |\___|
%                      |_|          |___/      
%
% «title»  (to ".title")
% (c3m211cp 1 "title")
% (c3m211ca   "title")

\thispagestyle{empty}

\begin{center}

\vspace*{1.2cm}

{\bf \Large Cálculo 3 - 2021.1}

\bsk

Aula 7: Cortes em superfícies

\bsk

Eduardo Ochs - RCN/PURO/UFF

\url{http://angg.twu.net/2021.1-C3.html}

\end{center}

\newpage

{\bf Introdução}

No semestre passado eu achei que todo mundo ia achar que visualizar
cortes em superfícies era a coisa mais fácil do mundo, deixei isso pro
final do semestre, e me ferrei. Dessa vez vou apresentar cortes o mais
cedo possível, de vários jeitos diferentes, e nós vamos usar esses
cortes pra muitas coisas (pra vocês exercitarem o olhômetro de vocês
bastante).

Comece lendo as páginas 81--100 do capítulo 3 do Bortolossi.

Repare que ele prefere usar funções que têm definições curtas, como $z
= f(x,y) = x^2 + y^2$. Nós vamos usar bastante funções que dão
superfícies fáceis de lembrar e em que é fácil calcular a altura da
superfície em cada ponto de cabeça, mas que são definidas ou por casos
usando mínimos e máximos.

\newpage

{\bf Exercício 1.}

% (c3m202p1p 3 "questao-1")
% (c3m202p1a   "questao-1")

Faça toda a questão 1 da P1 do semestre passado:

{\footnotesize

% http://angg.twu.net/LATEX/2020-2-C3-P1.pdf#page=3
\url{http://angg.twu.net/LATEX/2020-2-C3-P1.pdf\#page=3}

}



%\printbibliography

% (find-20202C3M12page 11 "outros cortes")
% (find-20202C3M12text 11 "outros cortes")
% (find-books "__analysis/__analysis.el" "bortolossi")
% (find-bortolossi3page (+ -78  81) "3.2. Funções de duas variáveis")
% (find-bortolossi3page (+ -78  86)   "Vamos tentar outros cortes. (Figs: pp.90-95)")
% (find-bortolossi3page (+ -78  93)   "Exemplo 3.2. Sela de cavalo.")
% (find-LATEXgrep "grep --color=auto -nH --null -e numerozi 2020-2-C3*.tex")

% (c3m202rcadeia1p 14 "video-numerozinhos")
% (c3m202rcadeia1a    "video-numerozinhos")


\GenericWarning{Success:}{Success!!!}  % Used by `M-x cv'

\end{document}

%  ____  _             _         
% |  _ \(_)_   ___   _(_)_______ 
% | | | | \ \ / / | | | |_  / _ \
% | |_| | |\ V /| |_| | |/ /  __/
% |____// | \_/  \__,_|_/___\___|
%     |__/                       
%
% «djvuize»  (to ".djvuize")
% (find-LATEXgrep "grep --color -nH --null -e djvuize 2020-1*.tex")

 (eepitch-shell)
 (eepitch-kill)
 (eepitch-shell)
# (find-fline "~/2021.1-C3/")
# (find-fline "~/LATEX/2021-1-C3/")
# (find-fline "~/bin/djvuize")

cd /tmp/
for i in *.jpg; do echo f $(basename $i .jpg); done

f () { rm -fv $1.png $1.pdf; djvuize $1.pdf }
f () { rm -fv $1.png $1.pdf; djvuize WHITEBOARDOPTS="-m 1.0" $1.pdf; xpdf $1.pdf }
f () { rm -fv $1.png $1.pdf; djvuize WHITEBOARDOPTS="-m 0.5" $1.pdf; xpdf $1.pdf }
f () { rm -fv $1.png $1.pdf; djvuize WHITEBOARDOPTS="-m 0.25" $1.pdf; xpdf $1.pdf }
f () { cp -fv $1.png $1.pdf       ~/2021.1-C3/
       cp -fv        $1.pdf ~/LATEX/2021-1-C3/
       cat <<%%%
% (find-latexscan-links "C3" "$1")
%%%
}

f 20201213_area_em_funcao_de_theta
f 20201213_area_em_funcao_de_x
f 20201213_area_fatias_pizza



%  __  __       _        
% |  \/  | __ _| | _____ 
% | |\/| |/ _` | |/ / _ \
% | |  | | (_| |   <  __/
% |_|  |_|\__,_|_|\_\___|
%                        
% <make>

 (eepitch-shell)
 (eepitch-kill)
 (eepitch-shell)
# (find-LATEXfile "2019planar-has-1.mk")
make -f 2019.mk STEM=2021-1-C3-cortes veryclean
make -f 2019.mk STEM=2021-1-C3-cortes pdf

% Local Variables:
% coding: utf-8-unix
% ee-tla: "c3c"
% ee-tla: "c3m211c"
% End:
