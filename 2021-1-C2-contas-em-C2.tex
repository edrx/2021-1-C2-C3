% (find-LATEX "2021-1-C2-contas-em-C2.tex")
% (defun c () (interactive) (find-LATEXsh "lualatex -record 2021-1-C2-contas-em-C2.tex" :end))
% (defun C () (interactive) (find-LATEXsh "lualatex 2021-1-C2-contas-em-C2.tex" "Success!!!"))
% (defun D () (interactive) (find-pdf-page      "~/LATEX/2021-1-C2-contas-em-C2.pdf"))
% (defun d () (interactive) (find-pdftools-page "~/LATEX/2021-1-C2-contas-em-C2.pdf"))
% (defun e () (interactive) (find-LATEX "2021-1-C2-contas-em-C2.tex"))
% (defun o () (interactive) (find-LATEX "2021-1-C2-contas-em-C2.tex"))
% (defun u () (interactive) (find-latex-upload-links "2021-1-C2-contas-em-C2"))
% (defun v () (interactive) (find-2a '(e) '(d)))
% (defun d0 () (interactive) (find-ebuffer "2021-1-C2-contas-em-C2.pdf"))
% (defun cv () (interactive) (C) (ee-kill-this-buffer) (v) (g))
%          (code-eec-LATEX "2021-1-C2-contas-em-C2")
% (find-pdf-page   "~/LATEX/2021-1-C2-contas-em-C2.pdf")
% (find-sh0 "cp -v  ~/LATEX/2021-1-C2-contas-em-C2.pdf /tmp/")
% (find-sh0 "cp -v  ~/LATEX/2021-1-C2-contas-em-C2.pdf /tmp/pen/")
%     (find-xournalpp "/tmp/2021-1-C2-contas-em-C2.pdf")
%   file:///home/edrx/LATEX/2021-1-C2-contas-em-C2.pdf
%               file:///tmp/2021-1-C2-contas-em-C2.pdf
%           file:///tmp/pen/2021-1-C2-contas-em-C2.pdf
% http://angg.twu.net/LATEX/2021-1-C2-contas-em-C2.pdf
% (find-LATEX "2019.mk")
% (find-CN-aula-links "2021-1-C2-contas-em-C2" "2" "c2m211cc2" "c2c")

% «.video-1»		(to "video-1")
% «.video-2»		(to "video-2")
%
% «.defs»		(to "defs")
% «.title»		(to "title")
% «.exemplo-1»		(to "exemplo-1")
% «.exemplo-1-cont»	(to "exemplo-1-cont")
% «.d-ln»		(to "d-ln")
% «.d-arcsen»		(to "d-arcsen")
% «.d-arcsen-2»		(to "d-arcsen-2")
% «.d-arcsen-3»		(to "d-arcsen-3")
%
% «.djvuize»	(to "djvuize")


% «video-1»  (to ".video-1")
% (find-ssr-links     "c2m211cc2" "2021-1-C2-contas-em-C2" "P3fbFkCI_Bo")
% (code-eevvideo      "c2m211cc2" "2021-1-C2-contas-em-C2" "P3fbFkCI_Bo")
% (code-eevlinksvideo "c2m211cc2" "2021-1-C2-contas-em-C2" "P3fbFkCI_Bo")
% (find-c2m211cc2video "0:00" "1º/set/2021")

% «video-2»  (to ".video-2")
% (find-ssr-links     "c2m211cc2" "2021-1-C2-contas-em-C2-2")
% (code-eevvideo      "c2m211cc2" "2021-1-C2-contas-em-C2-2")
% (code-eevlinksvideo "c2m211cc2" "2021-1-C2-contas-em-C2-2")
% (find-c2m211cc2video "0:00" "3/set/2021")


\documentclass[oneside,12pt]{article}
\usepackage[colorlinks,citecolor=DarkRed,urlcolor=DarkRed]{hyperref} % (find-es "tex" "hyperref")
\usepackage{amsmath}
\usepackage{amsfonts}
\usepackage{amssymb}
\usepackage{pict2e}
\usepackage[x11names,svgnames]{xcolor} % (find-es "tex" "xcolor")
\usepackage{colorweb}                  % (find-es "tex" "colorweb")
%\usepackage{tikz}
%
% (find-dn6 "preamble6.lua" "preamble0")
%\usepackage{proof}   % For derivation trees ("%:" lines)
%\input diagxy        % For 2D diagrams ("%D" lines)
%\xyoption{curve}     % For the ".curve=" feature in 2D diagrams
%
\usepackage{edrx21}               % (find-LATEX "edrx21.sty")
\input edrxaccents.tex            % (find-LATEX "edrxaccents.tex")
\input edrx21chars.tex            % (find-LATEX "edrx21chars.tex")
\input edrxheadfoot.tex           % (find-LATEX "edrxheadfoot.tex")
\input edrxgac2.tex               % (find-LATEX "edrxgac2.tex")
%
%\usepackage[backend=biber,
%   style=alphabetic]{biblatex}            % (find-es "tex" "biber")
%\addbibresource{catsem-slides.bib}        % (find-LATEX "catsem-slides.bib")
%
% (find-es "tex" "geometry")
\usepackage[a6paper, landscape,
            top=1.5cm, bottom=.25cm, left=1cm, right=1cm, includefoot
           ]{geometry}
%
\begin{document}

%\catcode`\^^J=10
%\directlua{dofile "dednat6load.lua"}  % (find-LATEX "dednat6load.lua")

% %L dofile "edrxtikz.lua"  -- (find-LATEX "edrxtikz.lua")
% %L dofile "edrxpict.lua"  -- (find-LATEX "edrxpict.lua")
% \pu

% «defs»  (to ".defs")
% (find-LATEX "edrx15.sty" "colors-2019")
%\long\def\ColorRed   #1{{\color{Red1}#1}}
%\long\def\ColorViolet#1{{\color{MagentaVioletLight}#1}}
%\long\def\ColorViolet#1{{\color{Violet!50!black}#1}}
%\long\def\ColorGreen #1{{\color{SpringDarkHard}#1}}
%\long\def\ColorGreen #1{{\color{SpringGreenDark}#1}}
%\long\def\ColorGreen #1{{\color{SpringGreen4}#1}}
%\long\def\ColorGray  #1{{\color{GrayLight}#1}}
%\long\def\ColorGray  #1{{\color{black!30!white}#1}}
%\long\def\ColorBrown #1{{\color{Brown}#1}}
%\long\def\ColorBrown #1{{\color{brown}#1}}
%\long\def\ColorOrange#1{{\color{orange}#1}}
%
%\long\def\ColorShort #1{{\color{SpringGreen4}#1}}
%\long\def\ColorLong  #1{{\color{Red1}#1}}
%
%\def\frown{\ensuremath{{=}{(}}}
%\def\True {\mathbf{V}}
%\def\False{\mathbf{F}}
%\def\D    {\displaystyle}

\def\pfo#1{\ensuremath{\mathsf{[#1]}}}

\def\drafturl{http://angg.twu.net/LATEX/2021-1-C2.pdf}
\def\drafturl{http://angg.twu.net/2021.1-C2.html}
\def\draftfooter{\tiny \href{\drafturl}{\jobname{}} \ColorBrown{\shorttoday{} \hours}}



%  _____ _ _   _                               
% |_   _(_) |_| | ___   _ __   __ _  __ _  ___ 
%   | | | | __| |/ _ \ | '_ \ / _` |/ _` |/ _ \
%   | | | | |_| |  __/ | |_) | (_| | (_| |  __/
%   |_| |_|\__|_|\___| | .__/ \__,_|\__, |\___|
%                      |_|          |___/      
%
% «title»  (to ".title")
% (c2m211cc2p 1 "title")
% (c2m211cc2a   "title")

\thispagestyle{empty}

\begin{center}

\vspace*{1.2cm}

{\bf \Large Cálculo 2 - 2021.1}

\bsk

Aula 23: como contas de integração

costumam ser organizadas

\bsk

Eduardo Ochs - RCN/PURO/UFF

\url{http://angg.twu.net/2021.1-C2.html}

\end{center}

\newpage

% «exemplo-1»  (to ".exemplo-1")
% (c2m211cc2p 2 "exemplo-1")
% (c2m211cc2a   "exemplo-1")

{\bf Exemplo 1.}

\ssk


{\footnotesize

% http://angg.twu.net/eev-videos/2021-1-C2-contas-em-C2.mp4
\url{http://angg.twu.net/eev-videos/2021-1-C2-contas-em-C2.mp4}

(Ele está explicado neste $↑$ vídeo! Assista!)

}

\msk


Temos:
%
$$\begin{array}{rcl}
  \ddx(f(x)g(x)) &=& f'(x)g(x) + f(x)g'(x) \\
       f(x)g(x)  &=& \intx {f'(x)g(x) + f(x)g'(x)} \\
                 &=& \intx {f'(x)g(x)} + \intx {f(x)g'(x)} \\[10pt]
  \end{array}
$$

Então:
%
$$\begin{array}{rcl}
  \intx {f'(x)g(x)} + \intx {f(x)g'(x)} &=& f(x)g(x) \\
                      \intx {f(x)g'(x)} &=& f(x)g(x) - \intx {f'(x)g(x)} \\
                      \intx {f'(x)g(x)} \phantom{mmmmmmmmi} &=& f(x)g(x) - \intx {f(x)g'(x)} \\
  \end{array}
$$


\newpage

% «exemplo-1-cont»  (to ".exemplo-1-cont")
% (c2m211cc2p 3 "exemplo-1-cont")
% (c2m211cc2a   "exemplo-1-cont")

Sejam:
%
$$\begin{array}{lrcl}
  \text{IP1}. &  \intx {f(x)g'(x)} &=& f(x)g(x) - \intx {f'(x)g(x)} \\
  \text{IP2}. &  \intx {f'(x)g(x)} &=& f(x)g(x) - \intx {f(x)g'(x)} \\
  \end{array}
$$

Então:
%
$$\begin{array}{rcl}
  \intx {x^3 e^x} &=& x^3 e^x - \intx {3 x^2 e^x} \\
                  &=& x^3 e^x - 3 \intx {x^2 e^x} \\
  \intx {x^2 e^x} &=& x^2 e^x -   \intx {2 x e^x} \\
                  &=& x^2 e^x -   2 \intx {x e^x} \\
  \intx {x   e^x} &=& x   e^x -       \intx {e^x} \\
                  &=& x   e^x -              e^x  \\[10pt]
  %
  \intx {x^3 e^x} &=& x^3 e^x - 3 \intx {x^2 e^x} \\
                  &=& x^3 e^x - 3 (x^2 e^x - 2 \intx {x e^x}) \\
                  &=& x^3 e^x - 3 (x^2 e^x - 2 (x e^x - \intx {e^x})) \\
                  &=& x^3 e^x - 3 (x^2 e^x - 2 (x e^x - e^x)) \\
  \end{array}
$$


\newpage

{\bf Derivada da função inversa}

\def\Dom#1{\ColorRed{#1}}

$$\begin{array}{rcl}
  \pfo{DFI1} &=&
    \left(
    \begin{array}{ll}
    \text{Se}    & \Dom{∀x∈D.\;} f(g(x)) = x     \\[2pt]
    \text{Então} & \ddx f(g(x)) = \ddx x = 1,    \\[2pt]
                 & \ddx f(g(x)) = f'(g(x))g'(x), \\[2pt]
                 & f'(g(x))g'(x) = 1,            \\[2pt]
                 & g'(x) = 1 / f'(g(x))          \\
    \end{array}
    \right)
    \\
    [40pt]
  \pfo{DFI2} &=&
    \left(
    \begin{array}{ll}
    \text{Se}    & \Dom{∀x∈D.\;} f(g(x)) = x     \\[2pt]
    \text{Então} & g'(x) = 1 / f'(g(x))          \\
    \end{array}
    \right)
  \end{array}
$$

$$\pfo{DFI2}
  \bmat{f(y) := e^y \\
        g(x) := \ln x \\
        D := (0,+∞) \\
       }
  \;\;=\;\;
    \left(
    \begin{array}{ll}
    \text{Se}    & \Dom{∀x∈(0,+∞).\;} e^{\ln x} = x  \\[2pt]
    \text{Então} & \ln'(x) = 1 / e^{\ln x}           \\
    \end{array}
    \right)
$$

\newpage

% «d-ln»  (to ".d-ln")
% (c2m211cc2p 5 "d-ln")
% (c2m211cc2a   "d-ln")

{\bf Derivada do $\ln$}
%
$$\pfo{DFI2}
  \bmat{f(y) := e^y \\
        g(x) := \ln x \\
        D := (0,+∞) \\
       }
  \;\;=\;\;
    \left(
    \begin{array}{ll}
    \text{Se}    & \Dom{∀x∈(0,+∞).\;} e^{\ln x} = x  \\[2pt]
    \text{Então} & \ln'(x) = 1 / e^{\ln x}           \\
    \end{array}
    \right)
$$

$$\begin{array}{rcl}
  \ln'(x) &=& 1/e^{\ln x} \qquad \text{(pelo \pfo{DFI2})}\\
          &=& 1/x \\
  \end{array}
$$


\newpage

% «d-arcsen»  (to ".d-arcsen")
% (c2m211cc2p 6 "d-arcsen")
% (c2m211cc2a   "d-arcsen")

{\bf Derivada do $\arcsen$}
%
$$\scalebox{0.85}{$
  \begin{array}{c}
  \pfo{DFI2}
  \;\;=\;\;
    \left(
    \begin{array}{ll}
    \text{Se}    & \Dom{∀x∈D.\;} f(g(x)) = x     \\[2pt]
    \text{Então} & g'(x) = 1 / f'(g(x))          \\
    \end{array}
    \right)
  %
  \\[15pt]
  %
  \pfo{DFI2}
  \bmat{x := s \\
        f(θ) := \sen θ \\
        g(s) := \arcsen s \\
        D := (-1,+1) \\
       }
  \;\;=\;\;
    \left(
    \begin{array}{ll}
    \text{Se}    & \Dom{∀s∈(-1,+1).\;} \sen \arcsen s = s  \\[2pt]
    \text{Então} & \arcsen'(s) = 1 / (\cos \arcsen s)      \\
    \end{array}
    \right)
  \end{array}
  $}
$$

$$\scalebox{0.85}{$
  \begin{array}{rcl}
  \cos^2 θ + \sen^2 θ &=& 1 \\
  \cos^2 θ &=& 1 - \sen^2 θ \\
  \cos θ &=& \sqrt{1 - \sen^2 θ} \\
  \cos (\arcsen s) &=& \sqrt{1 - (\sen(\arcsen s))^2} \\
                   &=& \sqrt{1 - s^2} \\
  \frac{d}{ds} \arcsen(s) &=& 1 / (\cos \arcsen s) \\
                          &=& 1 / \sqrt{1 - s^2} \\
               \arcsen(s) &=& \ints {1 / \sqrt{1 - s^2}} \\
  \end{array}
  $}
$$

\newpage

% «d-arcsen-2»  (to ".d-arcsen-2")
% (c2m211cc2p 7 "d-arcsen-2")
% (c2m211cc2a   "d-arcsen-2")

{\bf Derivada do $\arcsen$ (2)}

Se $s = \senθ$ então $\frac{ds}{dθ} = \cos θ$, $ds = \cosθ \, dθ$, e:
%
$$\begin{array}{l}
  \ints {1 / \sqrt{1 - s^2}} \\
  =\;\; \intth {\frac{1}{\sqrt{1 - (\senθ)^2}} \; \cosθ} \\
  =\;\; \intth {\frac{1}{\sqrt{(\cosθ)^2}} \; \cosθ} \\
  =\;\; \intth {\frac{1}{\cosθ} \; \cosθ} \\
  =\;\; \intth {1} \\
  =\;\; θ \\
  =\;\; \arcsen \sen θ \\
  =\;\; \arcsen s \\
  \end{array}
$$


\newpage

% «d-arcsen-3»  (to ".d-arcsen-3")
% (c2m211cc2p 8 "d-arcsen-3")
% (c2m211cc2a   "d-arcsen-3")

{\bf Derivada do $\arcsen$ (3)}

Se usarmos uma caixa de anotações bem maior

podemos fazer essa conta bem mais rápido...

$$\begin{array}{l}
  \ints {1 / \sqrt{1 - s^2}} \\
  =\;\; \intth {\frac{1}{\cosθ} \; \cosθ} \\
  =\;\; \intth {1} \\
  =\;\; θ \\
  =\;\; \arcsen s \\
  \end{array}
  \qquad
  \bmat{ s = \sen θ \\
         \frac{ds}{dθ} = \frac{d}{dθ} \sen θ = \cos θ \\
         ds = \cos θ \, dθ\\
         1 - s^2 = \cos^2 θ \\
         \sqrt{1 - s^2} = \cos θ \\
         θ = \arcsen s \\
       }
$$

\bsk

Essa caixa de anotações grande vai ser chamada de

\ColorRed{substituição trigonométrica} (para $s=\sen θ$).

\msk

Outras substituições trigonométricas

famosas: $t=\tanθ$, $z=\secθ$.


\newpage

Normalmente a gente aprende substituições trigonométricas

depois do método pra integrar potências de senos e cossenos...


\msk

Entenda os exemplos 1 e 2 daqui,

{\footnotesize

% (c2m202isp 12 "senos-e-cossenos")
% (c2m202isa    "senos-e-cossenos")
%    http://angg.twu.net/LATEX/2020-2-C2-int-subst.pdf#page=12
\url{http://angg.twu.net/LATEX/2020-2-C2-int-subst.pdf#page=12}

}

e tente fazer o ``exercício 3'' desse PDF.

Obs: no semestre passado eu usei convenções

um pouco diferentes das de agora pras

caixas de anotações...






%\printbibliography

\GenericWarning{Success:}{Success!!!}  % Used by `M-x cv'

\end{document}

%  ____  _             _         
% |  _ \(_)_   ___   _(_)_______ 
% | | | | \ \ / / | | | |_  / _ \
% | |_| | |\ V /| |_| | |/ /  __/
% |____// | \_/  \__,_|_/___\___|
%     |__/                       
%
% «djvuize»  (to ".djvuize")
% (find-LATEXgrep "grep --color -nH --null -e djvuize 2020-1*.tex")

 (eepitch-shell)
 (eepitch-kill)
 (eepitch-shell)
# (find-fline "~/2021.1-C2/")
# (find-fline "~/LATEX/2021-1-C2/")
# (find-fline "~/bin/djvuize")

cd /tmp/
for i in *.jpg; do echo f $(basename $i .jpg); done

f () { rm -v $1.pdf;  textcleaner -f 50 -o  5 $1.jpg $1.png; djvuize $1.pdf; xpdf $1.pdf }
f () { rm -v $1.pdf;  textcleaner -f 50 -o 10 $1.jpg $1.png; djvuize $1.pdf; xpdf $1.pdf }
f () { rm -v $1.pdf;  textcleaner -f 50 -o 20 $1.jpg $1.png; djvuize $1.pdf; xpdf $1.pdf }

f () { rm -fv $1.png $1.pdf; djvuize $1.pdf }
f () { rm -fv $1.png $1.pdf; djvuize WHITEBOARDOPTS="-m 1.0 -f 15" $1.pdf; xpdf $1.pdf }
f () { rm -fv $1.png $1.pdf; djvuize WHITEBOARDOPTS="-m 1.0 -f 30" $1.pdf; xpdf $1.pdf }
f () { rm -fv $1.png $1.pdf; djvuize WHITEBOARDOPTS="-m 1.0 -f 45" $1.pdf; xpdf $1.pdf }
f () { rm -fv $1.png $1.pdf; djvuize WHITEBOARDOPTS="-m 0.5" $1.pdf; xpdf $1.pdf }
f () { rm -fv $1.png $1.pdf; djvuize WHITEBOARDOPTS="-m 0.25" $1.pdf; xpdf $1.pdf }
f () { cp -fv $1.png $1.pdf       ~/2021.1-C2/
       cp -fv        $1.pdf ~/LATEX/2021-1-C2/
       cat <<%%%
% (find-latexscan-links "C2" "$1")
%%%
}

f 20201213_area_em_funcao_de_theta
f 20201213_area_em_funcao_de_x
f 20201213_area_fatias_pizza



%  __  __       _        
% |  \/  | __ _| | _____ 
% | |\/| |/ _` | |/ / _ \
% | |  | | (_| |   <  __/
% |_|  |_|\__,_|_|\_\___|
%                        
% <make>

 (eepitch-shell)
 (eepitch-kill)
 (eepitch-shell)
# (find-LATEXfile "2019planar-has-1.mk")
make -f 2019.mk STEM=2021-1-C2-contas-em-C2 veryclean
make -f 2019.mk STEM=2021-1-C2-contas-em-C2 pdf

% Local Variables:
% coding: utf-8-unix
% ee-tla: "c2c"
% ee-tla: "c2mcc2"
% ee-tla: "c2m211cc2"
% End:
