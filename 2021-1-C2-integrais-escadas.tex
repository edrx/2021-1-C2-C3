% (find-LATEX "2021-1-C2-integrais-escadas.tex")
% (defun c () (interactive) (find-LATEXsh "lualatex -record 2021-1-C2-integrais-escadas.tex" :end))
% (defun C () (interactive) (find-LATEXsh "lualatex 2021-1-C2-integrais-escadas.tex" "Success!!!"))
% (defun D () (interactive) (find-pdf-page      "~/LATEX/2021-1-C2-integrais-escadas.pdf"))
% (defun d () (interactive) (find-pdftools-page "~/LATEX/2021-1-C2-integrais-escadas.pdf"))
% (defun e () (interactive) (find-LATEX "2021-1-C2-integrais-escadas.tex"))
% (defun o () (interactive) (find-LATEX "2021-1-C2-integrais-escadas.tex"))
% (defun u () (interactive) (find-latex-upload-links "2021-1-C2-integrais-escadas"))
% (defun v () (interactive) (find-2a '(e) '(d)))
% (defun d0 () (interactive) (find-ebuffer "2021-1-C2-integrais-escadas.pdf"))
% (defun cv () (interactive) (C) (ee-kill-this-buffer) (v) (g))
%          (code-eec-LATEX "2021-1-C2-integrais-escadas")
% (find-pdf-page   "~/LATEX/2021-1-C2-integrais-escadas.pdf")
% (find-sh0 "cp -v  ~/LATEX/2021-1-C2-integrais-escadas.pdf /tmp/")
% (find-sh0 "cp -v  ~/LATEX/2021-1-C2-integrais-escadas.pdf /tmp/pen/")
%     (find-xournalpp "/tmp/2021-1-C2-integrais-escadas.pdf")
%   file:///home/edrx/LATEX/2021-1-C2-integrais-escadas.pdf
%               file:///tmp/2021-1-C2-integrais-escadas.pdf
%           file:///tmp/pen/2021-1-C2-integrais-escadas.pdf
% http://angg.twu.net/LATEX/2021-1-C2-integrais-escadas.pdf
% (find-LATEX "2019.mk")
% (find-CN-aula-links "2021-1-C2-integrais-escadas" "2" "c2m211ie" "c2ie")
%
% Video (not yet):
% (find-ssr-links "c2m211ie" "2021-1-C2-integrais-escadas")
% (code-video     "c2m211ievideo" "$S/http/angg.twu.net/eev-videos/2021-1-C2-integrais-escadas.mp4")
% (find-c2m211ievideo "0:00")

% «.defs»	(to "defs")
% «.title»	(to "title")
%
% «.djvuize»	(to "djvuize")

\documentclass[oneside,12pt]{article}
\usepackage[colorlinks,citecolor=DarkRed,urlcolor=DarkRed]{hyperref} % (find-es "tex" "hyperref")
\usepackage{amsmath}
\usepackage{amsfonts}
\usepackage{amssymb}
\usepackage{pict2e}
\usepackage[x11names,svgnames]{xcolor} % (find-es "tex" "xcolor")
\usepackage{colorweb}                  % (find-es "tex" "colorweb")
%\usepackage{tikz}
%
% (find-dn6 "preamble6.lua" "preamble0")
%\usepackage{proof}   % For derivation trees ("%:" lines)
%\input diagxy        % For 2D diagrams ("%D" lines)
%\xyoption{curve}     % For the ".curve=" feature in 2D diagrams
%
\usepackage{edrx21}               % (find-LATEX "edrx21.sty")
\input edrxaccents.tex            % (find-LATEX "edrxaccents.tex")
\input edrxchars.tex              % (find-LATEX "edrxchars.tex")
\input edrxheadfoot.tex           % (find-LATEX "edrxheadfoot.tex")
\input edrxgac2.tex               % (find-LATEX "edrxgac2.tex")
%
%\usepackage[backend=biber,
%   style=alphabetic]{biblatex}            % (find-es "tex" "biber")
%\addbibresource{catsem-slides.bib}        % (find-LATEX "catsem-slides.bib")
%
% (find-es "tex" "geometry")
\usepackage[a6paper, landscape,
            top=1.5cm, bottom=.25cm, left=1cm, right=1cm, includefoot
           ]{geometry}
%
\begin{document}

\catcode`\^^J=10
\directlua{dofile "dednat6load.lua"}  % (find-LATEX "dednat6load.lua")

%L dofile "edrxtikz.lua"  -- (find-LATEX "edrxtikz.lua")
%L dofile "edrxpict.lua"  -- (find-LATEX "edrxpict.lua")
\pu

% «defs»  (to ".defs")
% (find-LATEX "edrx15.sty" "colors-2019")
%\long\def\ColorRed   #1{{\color{Red1}#1}}
%\long\def\ColorViolet#1{{\color{MagentaVioletLight}#1}}
%\long\def\ColorViolet#1{{\color{Violet!50!black}#1}}
%\long\def\ColorGreen #1{{\color{SpringDarkHard}#1}}
%\long\def\ColorGreen #1{{\color{SpringGreenDark}#1}}
%\long\def\ColorGreen #1{{\color{SpringGreen4}#1}}
%\long\def\ColorGray  #1{{\color{GrayLight}#1}}
%\long\def\ColorGray  #1{{\color{black!30!white}#1}}
%\long\def\ColorBrown #1{{\color{Brown}#1}}
%\long\def\ColorBrown #1{{\color{brown}#1}}
%\long\def\ColorOrange#1{{\color{orange}#1}}
%
%\long\def\ColorShort #1{{\color{SpringGreen4}#1}}
%\long\def\ColorLong  #1{{\color{Red1}#1}}
%
%\def\frown{\ensuremath{{=}{(}}}
%\def\True {\mathbf{V}}
%\def\False{\mathbf{F}}
%\def\D    {\displaystyle}

\def\drafturl{http://angg.twu.net/LATEX/2021-1-C2.pdf}
\def\drafturl{http://angg.twu.net/2021.1-C2.html}
\def\draftfooter{\tiny \href{\drafturl}{\jobname{}} \ColorBrown{\shorttoday{} \hours}}



%  _____ _ _   _                               
% |_   _(_) |_| | ___   _ __   __ _  __ _  ___ 
%   | | | | __| |/ _ \ | '_ \ / _` |/ _` |/ _ \
%   | | | | |_| |  __/ | |_) | (_| | (_| |  __/
%   |_| |_|\__|_|\___| | .__/ \__,_|\__, |\___|
%                      |_|          |___/      
%
% «title»  (to ".title")
% (c2m211iep 1 "title")
% (c2m211iea   "title")

\thispagestyle{empty}

\begin{center}

\vspace*{1.2cm}

{\bf \Large Cálculo 2 - 2021.1}

\bsk

Aula 13: integrais de funções-escada

(Obsoleto! Deletar!)

\bsk

Eduardo Ochs - RCN/PURO/UFF

\url{http://angg.twu.net/2021.1-C2.html}

\end{center}

\newpage

{\bf Funções escada}

Uma {\sl função escada} é uma função definida por casos que é

constante em cada um dos casos, e em que todos os casos são

da forma ``quando $x∈〈\textit{intervalo}〉$''. Por exemplo,
%
\unitlength=15pt
%
$$f(x) =
  \vcenter{\hbox{%
    \beginpicture(0,-2)(6,3)
    \pictgrid%
    \pictaxes%
    \pictpiecewise{(0,1)--(2,1)c (2,2)o-(3,2)o (3,-1)c (3,0)o--(4,0)c (4,2)o--(6,2)}%
    \end{picture}%
  }}
  =
  \scalebox{0.9}{$
  \begin{cases}
     1 & \text{quando $x∈(-∞,2]$}, \\
     2 & \text{quando $x∈(2,3)$}, \\
    -1 & \text{quando $x∈[3,3]$}, \\
     0 & \text{quando $x∈(3,4]$}, \\
     2 & \text{quando $x∈(4,+∞)$} \\
  \end{cases}
  $}
$$

Note que também poderíamos ter escrito

$x≤2$ ao invés de $x∈(-∞,2]$,

$x=3$ ao invés de $x∈[3,3]$, etc...

Ah, e o número de casos tem que ser {\sl finito}.

\newpage

{\bf A função de Dirichlet}

A {\sl função de Dirichlet} é definida por:
%
$$f(x) =
  \begin{cases}
     0 & \text{quando $x∈\Q$}, \\
     1 & \text{quando $x∈\R∖\Q$} \\
  \end{cases}
$$

Ela não tem um nome oficial, então

vamos chamá-la de `$f$' nos próximos slides.

\msk

O gráfico dela alterna freneticamente entre $y=0$ e $y=1$.

\msk

Lembre que:

os números racionais são os cuja expansão decimal é

``periódica'', e os irracionais são os que não são assim;

entre cada dois racionais diferentes há um irracional, e

entre cada dois irracionais diferentes há um racional...

\newpage

{\bf A função de Dirichlet (2)}

\def\ui#1{\underline{#1}}

Lembre que podemos obter um irracional entre, digamos,

$a=\frac{10}{7}=1.42857\ui{142857}$ e
$b=\frac{1285715}{900000}=1.42857\ui{2}$,

modificando a expansão decimal de um dele e trocando-a

pela expansão decimal de $\sqrt{2}$ a partir de um

certo ponto... Por exemplo:

$$\begin{array}{rcl}
  \sqrt{2} &=& 1.41421356237... \\[5pt]
  b        &=& 1.42857\ui{222222}... \\
  c        &=& 1.42857156237... \\
  a        &=& 1.42857\ui{142857}... \\
  \end{array}
$$

Neste caso temos $a<c<b$, com $a,b∈\Q$ e $c∈\R∖\Q$.

Dá pra fazer algo parecido pra obter um racional

entre dois irracionais.


\newpage

{\bf A função de Dirichlet (3)}

Dá pra desenhar o gráfico da função de Dirichlet assim:
%
\unitlength=20pt
%
$$f(x) =
  \begin{cases}
     0 & \text{quando $x∈\Q$}, \\
     1 & \text{quando $x∈\R∖\Q$} \\
  \end{cases}
  \;\;
  =
  \;\;
  \vcenter{\hbox{%
    \beginpicture(0,0)(2,1)
    \pictgrid%
    \pictaxes%
    \pictpiecewise{(0.1,1)c (0.3,1)c (0.5,1)c (0.7,1)c (0.9,1)c
                   (1.1,1)c (1.3,1)c (1.5,1)c (1.7,1)c (1.9,1)c 
                   (0.0,0)c (0.2,0)c (0.4,0)c (0.6,0)c (0.8,0)c
                   (1.0,0)c (1.2,0)c (1.4,0)c (1.6,0)c (1.8,0)c (2.0,0)c 
                   }%
    \end{picture}%
  }}
$$

Repare que isso só funciona porque o desenho é claramente

ambíguo... um leitor ``normal'' não consegue descobrir no olho

quais são as coordenadas da bolinhas em $y=1$ e em $y=0$,

então ele é obrigado a olhar pra definição formal da $f(x)$...

\msk

e aí quando ele entende a definição formal da $f(x)$ ele

descobre que o desenho quer dizer ``muitas bolinhas em $y=1$,

muito próximas umas das outras, e muitas bolinhas em $y=0$

muito próximas das outras''...

\msk

...e ele entende que esse ``muitas'' quer dizer ``infinitas''.

% (sqrt 2)
% (/ 10.0 7) 
% (* (/ 1.0 7) 9999)
% (* 1.428572222 900000)
% (/ 1285715 900000.0)



\newpage

{\bf A função de Dirichlet (4)}


A função de Dirichlet é um dos exemplos mais simples

de uma função que não é integrável.


\newpage





{\bf Propriedades da integral}





% (find-books "__analysis/__analysis.el" "beneveri")



%\printbibliography

\GenericWarning{Success:}{Success!!!}  % Used by `M-x cv'

\end{document}

%  ____  _             _         
% |  _ \(_)_   ___   _(_)_______ 
% | | | | \ \ / / | | | |_  / _ \
% | |_| | |\ V /| |_| | |/ /  __/
% |____// | \_/  \__,_|_/___\___|
%     |__/                       
%
% «djvuize»  (to ".djvuize")
% (find-LATEXgrep "grep --color -nH --null -e djvuize 2020-1*.tex")

 (eepitch-shell)
 (eepitch-kill)
 (eepitch-shell)
# (find-fline "~/2021.1-C2/")
# (find-fline "~/LATEX/2021-1-C2/")
# (find-fline "~/bin/djvuize")

cd /tmp/
for i in *.jpg; do echo f $(basename $i .jpg); done

f () { rm -fv $1.png $1.pdf; djvuize $1.pdf }
f () { rm -fv $1.png $1.pdf; djvuize WHITEBOARDOPTS="-m 1.0" $1.pdf; xpdf $1.pdf }
f () { rm -fv $1.png $1.pdf; djvuize WHITEBOARDOPTS="-m 0.5" $1.pdf; xpdf $1.pdf }
f () { rm -fv $1.png $1.pdf; djvuize WHITEBOARDOPTS="-m 0.25" $1.pdf; xpdf $1.pdf }
f () { cp -fv $1.png $1.pdf       ~/2021.1-C2/
       cp -fv        $1.pdf ~/LATEX/2021-1-C2/
       cat <<%%%
% (find-latexscan-links "C2" "$1")
%%%
}

f 20201213_area_em_funcao_de_theta
f 20201213_area_em_funcao_de_x
f 20201213_area_fatias_pizza



%  __  __       _        
% |  \/  | __ _| | _____ 
% | |\/| |/ _` | |/ / _ \
% | |  | | (_| |   <  __/
% |_|  |_|\__,_|_|\_\___|
%                        
% <make>

 (eepitch-shell)
 (eepitch-kill)
 (eepitch-shell)
# (find-LATEXfile "2019planar-has-1.mk")
make -f 2019.mk STEM=2021-1-C2-integrais-escadas veryclean
make -f 2019.mk STEM=2021-1-C2-integrais-escadas pdf

% Local Variables:
% coding: utf-8-unix
% ee-tla: "c2ie"
% ee-tla: "c2m211ie"
% End:
