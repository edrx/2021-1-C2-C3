% (find-LATEX "2021-1-C3-intro.tex")
% (defun c () (interactive) (find-LATEXsh "lualatex -record 2021-1-C3-intro.tex" :end))
% (defun C () (interactive) (find-LATEXsh "lualatex 2021-1-C3-intro.tex" "Success!!!"))
% (defun D () (interactive) (find-pdf-page      "~/LATEX/2021-1-C3-intro.pdf"))
% (defun d () (interactive) (find-pdftools-page "~/LATEX/2021-1-C3-intro.pdf"))
% (defun e () (interactive) (find-LATEX "2021-1-C3-intro.tex"))
% (defun o () (interactive) (find-LATEX "2021-1-C3-intro.tex"))
% (defun u () (interactive) (find-latex-upload-links "2021-1-C3-intro"))
% (defun v () (interactive) (find-2a '(e) '(d)))
% (defun d0 () (interactive) (find-ebuffer "2021-1-C3-intro.pdf"))
% (defun cv () (interactive) (C) (ee-kill-this-buffer) (v) (g))
%          (code-eec-LATEX "2021-1-C3-intro")
% (find-pdf-page   "~/LATEX/2021-1-C3-intro.pdf")
% (find-sh0 "cp -v  ~/LATEX/2021-1-C3-intro.pdf /tmp/")
% (find-sh0 "cp -v  ~/LATEX/2021-1-C3-intro.pdf /tmp/pen/")
%     (find-xournalpp "/tmp/2021-1-C3-intro.pdf")
%   file:///home/edrx/LATEX/2021-1-C3-intro.pdf
%               file:///tmp/2021-1-C3-intro.pdf
%           file:///tmp/pen/2021-1-C3-intro.pdf
% http://angg.twu.net/LATEX/2021-1-C3-intro.pdf
% (find-LATEX "2019.mk")
% (find-CN-aula-links "2021-1-C3-intro" "3" "c3m211intro" "c3i")
%
% Video novo:
% (find-ssr-links "c3m211intro" "2021-1-C3-intro" "xXOARCnYt_0")
% (code-video     "c3m211introvideo" "$S/http/angg.twu.net/eev-videos/2021-1-C3-intro.mp4")
% (find-c3m211introvideo "0:00")

% Video do semestre passado:
% https://studio.youtube.com/video/3yWLubqHsic/edit

% «.defs»		(to "defs")
% «.title»		(to "title")
% «.introducao»		(to "introducao")
% «.exercicio-1»	(to "exercicio-1")
% «.exercicio-2»	(to "exercicio-2")
% «.exercicio-3»	(to "exercicio-3")
% «.exercicio-4»	(to "exercicio-4")
%
% «.djvuize»		(to "djvuize")

\documentclass[oneside,12pt]{article}
\usepackage[colorlinks,citecolor=DarkRed,urlcolor=DarkRed]{hyperref} % (find-es "tex" "hyperref")
\usepackage{amsmath}
\usepackage{amsfonts}
\usepackage{amssymb}
\usepackage{pict2e}
\usepackage[x11names,svgnames]{xcolor} % (find-es "tex" "xcolor")
\usepackage{colorweb}                  % (find-es "tex" "colorweb")
%\usepackage{tikz}
%
% (find-dn6 "preamble6.lua" "preamble0")
%\usepackage{proof}   % For derivation trees ("%:" lines)
%\input diagxy        % For 2D diagrams ("%D" lines)
%\xyoption{curve}     % For the ".curve=" feature in 2D diagrams
%
\usepackage{edrx15}               % (find-LATEX "edrx15.sty")
\input edrxaccents.tex            % (find-LATEX "edrxaccents.tex")
\input edrxchars.tex              % (find-LATEX "edrxchars.tex")
\input edrxheadfoot.tex           % (find-LATEX "edrxheadfoot.tex")
\input edrxgac2.tex               % (find-LATEX "edrxgac2.tex")
%
%\usepackage[backend=biber,
%   style=alphabetic]{biblatex}            % (find-es "tex" "biber")
%\addbibresource{catsem-slides.bib}        % (find-LATEX "catsem-slides.bib")
%
% (find-es "tex" "geometry")
\usepackage[a6paper, landscape,
            top=1.5cm, bottom=.25cm, left=1cm, right=1cm, includefoot
           ]{geometry}
%
\begin{document}

%\catcode`\^^J=10
%\directlua{dofile "dednat6load.lua"}  % (find-LATEX "dednat6load.lua")

% %L dofile "edrxtikz.lua"  -- (find-LATEX "edrxtikz.lua")
% %L dofile "edrxpict.lua"  -- (find-LATEX "edrxpict.lua")
% \pu

% «defs»  (to ".defs")
% (find-LATEX "edrx15.sty" "colors-2019")
\long\def\ColorRed   #1{{\color{Red1}#1}}
\long\def\ColorViolet#1{{\color{MagentaVioletLight}#1}}
\long\def\ColorViolet#1{{\color{Violet!50!black}#1}}
\long\def\ColorGreen #1{{\color{SpringDarkHard}#1}}
\long\def\ColorGreen #1{{\color{SpringGreenDark}#1}}
\long\def\ColorGreen #1{{\color{SpringGreen4}#1}}
\long\def\ColorGray  #1{{\color{GrayLight}#1}}
\long\def\ColorGray  #1{{\color{black!30!white}#1}}
\long\def\ColorBrown #1{{\color{Brown}#1}}
\long\def\ColorBrown #1{{\color{brown}#1}}
\long\def\ColorOrange#1{{\color{orange}#1}}

\long\def\ColorShort #1{{\color{SpringGreen4}#1}}
\long\def\ColorLong  #1{{\color{Red1}#1}}

\def\frown{\ensuremath{{=}{(}}}
\def\True {\mathbf{V}}
\def\False{\mathbf{F}}
\def\D    {\displaystyle}

\def\drafturl{http://angg.twu.net/LATEX/2021-1-C3.pdf}
\def\drafturl{http://angg.twu.net/2021.1-C3.html}
\def\draftfooter{\tiny \href{\drafturl}{\jobname{}} \ColorBrown{\shorttoday{} \hours}}



%  _____ _ _   _                               
% |_   _(_) |_| | ___   _ __   __ _  __ _  ___ 
%   | | | | __| |/ _ \ | '_ \ / _` |/ _` |/ _ \
%   | | | | |_| |  __/ | |_) | (_| | (_| |  __/
%   |_| |_|\__|_|\___| | .__/ \__,_|\__, |\___|
%                      |_|          |___/      
%
% «title»  (to ".title")
% (c3m211introp 1 "title")
% (c3m211intro    "title")

\thispagestyle{empty}

\begin{center}

\vspace*{1.2cm}

{\bf \Large Cálculo 3 - 2021.1}

\bsk

Aula 1: introdução ao curso

\bsk

Eduardo Ochs - RCN/PURO/UFF

\url{http://angg.twu.net/2021.1-C3.html}

\end{center}

\newpage


% «introducao»  (to ".introducao")
% (c3m201introp 1 "title")
% (c3m201introa   "title")
% (c3m201introp 5 "vetores-como-setas")
% (c3m201introa   "vetores-como-setas")

{\bf Introdução ao curso}

Cálculo 3 é principalmente sobre:

\begin{enumerate}

\item funções de $\R$ em $\R^2$ -- que o Bortolossi costuma chamar de
  \ColorRed{curvas parametrizadas}, mas nós vamos chamar de
  \ColorRed{trajetórias}, e

\item funções de $\R^2$ em $\R$, que vão gerar \ColorRed{superfícies}.

\end{enumerate}

Depois que nós aprendermos o suficiente sobre (1) e (2) nós vamos
poder lidar com coisas um pouco mais gerais, como funções $F:A→\R^n$,
onde $A⊆\R^n$ é um \ColorRed{conjunto aberto}.

\newpage

{\bf Nossos primeiros objetivos vão ser:}

\begin{enumerate}

\item Aprender a representar graficamente algumas trajetórias, usando
  a idéia de \ColorRed{traço} do Bortolossi (cap.6, p.188), mas
  escrevendo algumas informações a mais, como ``$t=0$'' e ``$t=1$'' em
  alguns pontos,

\item Calcular e representar graficamente \ColorRed{vetores tangentes}
  a trajetórias (``\ColorRed{vetores velocidade}''),

\item Entender \ColorRed{vetores secantes} (cap.6, p.199),

\item Entender \ColorRed{aproximações de primeira ordem} pra
  trajetórias, que dão \ColorRed{retas parametrizadas}, e depois
  \ColorRed{aproximações de segunda ordem}, que vão dar
  \ColorRed{parábolas parametrizadas}.

\end{enumerate}

...mas hoje nós vamos fazer uma revisão de algumas idéias de GA.

\newpage

% «convencao-de-GA-e-de-AL»  (to ".convencao-de-GA-e-de-AL")
% (c3m202introp 4 "convencao-de-GA-e-de-AL")
% (c3m202intro    "convencao-de-GA-e-de-AL")
% (c3m201introp 4 "convencao-de-GA-e-de-AL")
% (c3m201intro    "convencao-de-GA-e-de-AL")

Você já deve ter visto estas duas convenções diferentes para
representar pontos e vetores... am \ColorRed{Álgebra Linear} tanto
pontos quanto vetores em $\R^2$ são representados como matrizes-coluna
de altura 2:
%
$$\pmat{2\\3} + \pmat{40\\50} = \pmat{42\\53}$$
%
e em \ColorRed{Geometria Analítica} pontos e vetores são escritos de
forma diferente -- vetores têm uma seta em cima -- e representados
graficamente de formas diferentes...
%
$$(2,3) + \VEC{40,50} = (42,53)$$
%


\newpage

% «vetores-como-setas»  (to ".vetores-como-setas")
% (c3m211introp 5 "vetores-como-setas")
% (c3m211introa   "vetores-como-setas")

{\bf Vetores como setas}

Um \ColorRed{ponto} $(a,b)$ é interpretado graficamente como um ponto
$(a,b)$ de $\R^2$, e um \ColorRed{vetor} $\VEC{c,d}$ é interpetado
como um \ColorRed{deslocamento}, e desenhado como uma \ColorRed{seta}.

Se o vetor $\VEC{c,d}$ aparece sozinho a representação gráfica dele é
\ColorRed{qualquer} seta que anda $c$ unidades pra direita e $d$
unidades pra cima. Às vezes a gente pensa que $\VEC{c,d}$ é o conjunto
de {\sl todas} as setas assim -- o conjunto de todas as setas
``equipolentes'' a esta; veja a p.9 do livro do CEDERJ.

\newpage

% «convencao-temporaria»  (to ".convencao-temporaria")
% (c3m211introp 6 "convencao-temporaria")
% (c3m211introa   "convencao-temporaria")

{\bf Uma convenção (temporária)}

O \ColorRed{resultado} da expressão $(a,b)+\VEC{c,d}$ é o ponto
$(a+c,b+d)$,

mas a representação gráfica dele vai ser:

1) o ponto $(a,b)$,

2) uma seta indo de $(a,b)$ para $(a+c,b+d)$,

3) o ponto $(a+c,b+d)$,

4) anotações dos lados dos pontos $(a,b)$ e $(a+c,b+d)$ dizendo os
``nomes'' destes pontos e uma anotação do lado da seta $\VEC{c,d}$
dizendo o seu ``nome'' --- como nos dois exemplos abaixo (oops! Falta
fazer os desenhos!):

\msk

(pôr o desenho aqui)

\msk

Nesta aula vai ser obrigatório pôr todos os nomes, mas nas outras não.

\newpage

A representação gráfica de

\def\und#1#2{\underbrace{#1}_{\textstyle#2}}
\def\Red#1{\ColorRed{#1}}

$$((1,1) + \VEC{2,0}) + \VEC{1,2} = (1,1) + (\VEC{2,0} + \VEC{1,2})$$

Vai ser um triângulo feito de três pontos e três setas -- os que estão
em vermelho aqui:

$$\und{(\und{\Red{(1,1)} + \Red{\VEC{2,0}}}{\Red{(3,1)}}) + \Red{\VEC{1,2}}}{\Red{(4,3)}} =
  \und{\Red{(1,1)} + (\und{\VEC{2,0} + \VEC{1,2}}{\Red{\VEC{3,2}}})}{\Red{(4,3)}}
$$

O objetivo do próximo exercício é você relembrar como representar
graficamente certas expressões com pontos e vetores usando quase só o
olhômetro, quase sem fazer contas.

\fbox{Veja o vídeo!}

\newpage

{\bf Desenhando parábolas (quase) no olhômetro}

Digamos que conhecemos $A$, $\vv$, e $\ww$. Então a trajetória
%
$$P(t) = A + t\vv + t^2\ww$$
%
é uma parábola -- e queremos aprender a desenhar os 5 pontos mais
fáceis dela, que são $P(0)$, $P(1)$, $P(-1)$, $P(2)$, $P(-2)$, usando
o máximo de olhômetro e o mínimo possível de contas...

\msk


\fbox{Veja o vídeo!}

\newpage

% «exercicio-1»  (to ".exercicio-1")
% (c3m211introp 9 "exercicio-1")
% (c3m211introa   "exercicio-1")

{\bf Exercício: desenhando parábolas (quase) no olhômetro}

1) Sejam $A=(3,1)$, $\vv = \VEC{1,0}$, $\ww = \VEC{0,1}$.

Represente graficamente \ColorRed{num gráfico só}:

a) $A$

b) $(A+\vv)+\ww$

c) $(A+\ww)+\vv$

d) $(A+2\vv)+4\ww$

e) $(A+4\ww)+2\vv$

f) $(A-\vv)+\ww$

g) $(A+\ww)-\vv$

h) $(A-2\vv)+4\ww$

i) $(A+4\ww)-2\vv$


\newpage

% «exercicio-2»  (to ".exercicio-2")
% (c3m211introp 10 "exercicio-2")
% (c3m211introa    "exercicio-2")

{\bf Exercício: desenhando parábolas (quase) no olhômetro (2)}

2) Sejam $A=(1,1)$, $\vv = \VEC{1,-1}$, $\ww = \VEC{1,1}$.

Represente graficamente \ColorRed{num gráfico só}:

a) $A$

b) $(A+\vv)+\ww$

c) $(A+\ww)+\vv$

d) $(A+2\vv)+4\ww$

e) $(A+4\ww)+2\vv$

f) $(A-\vv)+\ww$

g) $(A+\ww)-\vv$

h) $(A-2\vv)+4\ww$

i) $(A+4\ww)-2\vv$

\newpage

% «exercicio-3»  (to ".exercicio-3")
% (c3m211introp 11 "exercicio-3")
% (c3m211introa    "exercicio-3")

{\bf Exercício: desenhando parábolas (quase) no olhômetro (3)}

3) Sejam $A=(1,1)$, $\vv = \VEC{1,-1}$, $\ww = \VEC{-1,1}$.

Represente graficamente \ColorRed{num gráfico só}:

a) $A$

b) $(A+\vv)+\ww$

c) $(A+\ww)+\vv$

d) $(A+2\vv)+4\ww$

e) $(A+4\ww)+2\vv$

f) $(A-\vv)+\ww$

g) $(A+\ww)-\vv$

h) $(A-2\vv)+4\ww$

i) $(A+4\ww)-2\vv$


\newpage

% «exercicio-4»  (to ".exercicio-4")
% (c3m211introp 12 "exercicio-4")
% (c3m211introa    "exercicio-4")

{\bf Exercício: desenhando parábolas (quase) no olhômetro (4)}

4) Sejam $A=(2,6)$, $\vv = \VEC{1,1}$, $\ww = \VEC{2,-1}$.

Represente graficamente \ColorRed{num gráfico só}:

a) $A$

b) $(A+\vv)+\ww$

c) $(A+\ww)+\vv$

d) $(A+2\vv)+4\ww$

e) $(A+4\ww)+2\vv$

f) $(A-\vv)+\ww$

g) $(A+\ww)-\vv$

h) $(A-2\vv)+4\ww$

i) $(A+4\ww)-2\vv$

\msk

Obs: você vai precisar de um gráfico que contenha os pontos

(0,0) e (12,8).



%\printbibliography

\GenericWarning{Success:}{Success!!!}  % Used by `M-x cv'

\end{document}

%  ____  _             _         
% |  _ \(_)_   ___   _(_)_______ 
% | | | | \ \ / / | | | |_  / _ \
% | |_| | |\ V /| |_| | |/ /  __/
% |____// | \_/  \__,_|_/___\___|
%     |__/                       
%
% «djvuize»  (to ".djvuize")
% (find-LATEXgrep "grep --color -nH --null -e djvuize 2020-1*.tex")

 (eepitch-shell)
 (eepitch-kill)
 (eepitch-shell)
# (find-fline "~/2021.1-C3/")
# (find-fline "~/LATEX/2021-1-C3/")
# (find-fline "~/bin/djvuize")

cd /tmp/
for i in *.jpg; do echo f $(basename $i .jpg); done

f () { rm -fv $1.png $1.pdf; djvuize $1.pdf }
f () { rm -fv $1.png $1.pdf; djvuize WHITEBOARDOPTS="-m 1.0" $1.pdf; xpdf $1.pdf }
f () { rm -fv $1.png $1.pdf; djvuize WHITEBOARDOPTS="-m 0.5" $1.pdf; xpdf $1.pdf }
f () { rm -fv $1.png $1.pdf; djvuize WHITEBOARDOPTS="-m 0.25" $1.pdf; xpdf $1.pdf }
f () { cp -fv $1.png $1.pdf       ~/2021.1-C3/
       cp -fv        $1.pdf ~/LATEX/2021-1-C3/
       cat <<%%%
% (find-latexscan-links "C3" "$1")
%%%
}

f 20201213_area_em_funcao_de_theta
f 20201213_area_em_funcao_de_x
f 20201213_area_fatias_pizza



%  __  __       _        
% |  \/  | __ _| | _____ 
% | |\/| |/ _` | |/ / _ \
% | |  | | (_| |   <  __/
% |_|  |_|\__,_|_|\_\___|
%                        
% <make>

 (eepitch-shell)
 (eepitch-kill)
 (eepitch-shell)
# (find-LATEXfile "2019planar-has-1.mk")
make -f 2019.mk STEM=2021-1-C3-intro veryclean
make -f 2019.mk STEM=2021-1-C3-intro pdf

% Local Variables:
% coding: utf-8-unix
% ee-tla: "c3m211intro"
% End:
