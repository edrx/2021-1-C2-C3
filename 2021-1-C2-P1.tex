% (find-LATEX "2021-1-C2-P1.tex")
% (defun c () (interactive) (find-LATEXsh "lualatex -record 2021-1-C2-P1.tex" :end))
% (defun C () (interactive) (find-LATEXsh "lualatex 2021-1-C2-P1.tex" "Success!!!"))
% (defun D () (interactive) (find-pdf-page      "~/LATEX/2021-1-C2-P1.pdf"))
% (defun d () (interactive) (find-pdftools-page "~/LATEX/2021-1-C2-P1.pdf"))
% (defun e () (interactive) (find-LATEX "2021-1-C2-P1.tex"))
% (defun o () (interactive) (find-LATEX "2020-2-C2-P1.tex"))
% (defun u () (interactive) (find-latex-upload-links "2021-1-C2-P1"))
% (defun v () (interactive) (find-2a '(e) '(d)))
% (defun d0 () (interactive) (find-ebuffer "2021-1-C2-P1.pdf"))
% (defun cv () (interactive) (C) (ee-kill-this-buffer) (v) (g))
%          (code-eec-LATEX "2021-1-C2-P1")
% (find-pdf-page   "~/LATEX/2021-1-C2-P1.pdf")
% (find-sh0 "cp -v  ~/LATEX/2021-1-C2-P1.pdf /tmp/")
% (find-sh0 "cp -v  ~/LATEX/2021-1-C2-P1.pdf /tmp/pen/")
%     (find-xournalpp "/tmp/2021-1-C2-P1.pdf")
%   file:///home/edrx/LATEX/2021-1-C2-P1.pdf
%               file:///tmp/2021-1-C2-P1.pdf
%           file:///tmp/pen/2021-1-C2-P1.pdf
% http://angg.twu.net/LATEX/2021-1-C2-P1.pdf
% (find-LATEX "2019.mk")
% (find-CN-aula-links "2021-1-C2-P1" "2" "c2m211p1" "c2p1")
%
% Video (not yet):
% (find-ssr-links "c2m211p1" "2021-1-C2-P1" "{naoexiste}")
% (code-video     "c2m211p1video" "$S/http/angg.twu.net/eev-videos/2021-1-C2-P1.mp4")
% (find-c2m211p1video "0:00")

% «.defs»			(to "defs")
% «.subst-defs»			(to "subst-defs")
% «.defs-T-and-B»		(to "defs-T-and-B")
% «.title»			(to "title")
% «.regras-e-dicas»		(to "regras-e-dicas")
% «.aviso-2»			(to "aviso-2")
% «.questao-1»			(to "questao-1")
% «.questao-2»			(to "questao-2")
% «.questao-3»			(to "questao-3")
% «.questao-3-cont»		(to "questao-3-cont")
% «.questao-1-gab»		(to "questao-1-gab")
% «.questao-2-gab»		(to "questao-2-gab")
% «.questao-3-gab»		(to "questao-3-gab")
% «.apendice»			(to "apendice")
% «.gabarito-2-2020.2»		(to "gabarito-2-2020.2")
% «.quatro-substituicoes»	(to "quatro-substituicoes")
% «.exercicio»			(to "exercicio")
%
% «.djvuize»			(to "djvuize")

\documentclass[oneside,12pt]{article}
\usepackage[colorlinks,citecolor=DarkRed,urlcolor=DarkRed]{hyperref} % (find-es "tex" "hyperref")
\usepackage{amsmath}
\usepackage{amsfonts}
\usepackage{amssymb}
\usepackage{pict2e}
\usepackage[x11names,svgnames]{xcolor} % (find-es "tex" "xcolor")
\usepackage{colorweb}                  % (find-es "tex" "colorweb")
%\usepackage{tikz}
%
% (find-dn6 "preamble6.lua" "preamble0")
%\usepackage{proof}   % For derivation trees ("%:" lines)
%\input diagxy        % For 2D diagrams ("%D" lines)
%\xyoption{curve}     % For the ".curve=" feature in 2D diagrams
%
\usepackage{edrx21}               % (find-LATEX "edrx21.sty")
\input edrxaccents.tex            % (find-LATEX "edrxaccents.tex")
\input edrx21chars.tex            % (find-LATEX "edrx21chars.tex")
\input edrxheadfoot.tex           % (find-LATEX "edrxheadfoot.tex")
\input edrxgac2.tex               % (find-LATEX "edrxgac2.tex")
%
%\usepackage[backend=biber,
%   style=alphabetic]{biblatex}            % (find-es "tex" "biber")
%\addbibresource{catsem-slides.bib}        % (find-LATEX "catsem-slides.bib")
%
% (find-es "tex" "geometry")
\usepackage[a6paper, landscape,
            top=1.5cm, bottom=.25cm, left=1cm, right=1cm, includefoot
           ]{geometry}
%
\begin{document}

\catcode`\^^J=10
\directlua{dofile "dednat6load.lua"}  % (find-LATEX "dednat6load.lua")

%L dofile "edrxtikz.lua"  -- (find-LATEX "edrxtikz.lua")
%L dofile "edrxpict.lua"  -- (find-LATEX "edrxpict.lua")
\pu

% «defs»  (to ".defs")
% (find-LATEX "edrx15.sty" "colors-2019")
%\long\def\ColorRed   #1{{\color{Red1}#1}}
%\long\def\ColorViolet#1{{\color{MagentaVioletLight}#1}}
%\long\def\ColorViolet#1{{\color{Violet!50!black}#1}}
%\long\def\ColorGreen #1{{\color{SpringDarkHard}#1}}
%\long\def\ColorGreen #1{{\color{SpringGreenDark}#1}}
%\long\def\ColorGreen #1{{\color{SpringGreen4}#1}}
%\long\def\ColorGray  #1{{\color{GrayLight}#1}}
%\long\def\ColorGray  #1{{\color{black!30!white}#1}}
%\long\def\ColorBrown #1{{\color{Brown}#1}}
%\long\def\ColorBrown #1{{\color{brown}#1}}
%\long\def\ColorOrange#1{{\color{orange}#1}}
%
%\long\def\ColorShort #1{{\color{SpringGreen4}#1}}
%\long\def\ColorLong  #1{{\color{Red1}#1}}
%
%\def\frown{\ensuremath{{=}{(}}}
%\def\True {\mathbf{V}}
%\def\False{\mathbf{F}}
%\def\D    {\displaystyle}

\def\drafturl{http://angg.twu.net/LATEX/2021-1-C2.pdf}
\def\drafturl{http://angg.twu.net/2021.1-C2.html}
\def\draftfooter{\tiny \href{\drafturl}{\jobname{}} \ColorBrown{\shorttoday{} \hours}}


% «subst-defs»  (to ".subst-defs")
% (find-LATEX "2020-1-C2-TFC2-2.tex" "subst-defs")

\def\pfo#1{\ensuremath{\mathsf{[#1]}}}
\def\veq{\rotatebox{90}{$=$}}
\def\Rd{\ColorRed}
\def\D{\displaystyle}

% Difference with mathstrut
\def\Difms #1#2#3{\left. \mathstrut #3 \right|_{s=#1}^{s=#2}}
\def\Difmu #1#2#3{\left. \mathstrut #3 \right|_{u=#1}^{u=#2}}
\def\Difmx #1#2#3{\left. \mathstrut #3 \right|_{x=#1}^{x=#2}}
\def\Difmth#1#2#3{\left. \mathstrut #3 \right|_{θ=#1}^{θ=#2}}

\def\iequationbox#1#2{
    \left(
    \begin{array}{rcl}
    \D{ #1 } &=& \D{ #2 } \\
    \end{array}
    \right)
  }
\def\isubstbox#1#2#3#4#5{{
    \def\veq{\rotatebox{90}{$=$}}
    \def\ph{\phantom}
    \left(
    \begin{array}{rcl}
    \D{ #1 } &=& \D{ #2 } \\
    {\veq#3} \\
    \D{ #4 } &=& \D{ #5 } \\
    \end{array}
    \right)
  }}
\def\isubstboxT#1#2#3#4#5#6{{
    \def\veq{\rotatebox{90}{$=$}}
    \def\ph{\phantom}
    \left(
    \begin{array}{rcl}
    \multicolumn{3}{l}{\text{#6}} \\%[5pt]
    \D{ #1 } &=& \D{ #2 } \\
    {\veq#3} \\
    \D{ #4 } &=& \D{ #5 } \\
    \end{array}
    \right)
  }}
\def\isubstboxTT#1#2#3#4#5#6#7{{
    \def\veq{\rotatebox{90}{$=$}}
    \def\ph{\phantom}
    \left(
    \begin{array}{rcl}
    \multicolumn{3}{l}{\text{#6}} \\%[5pt]
    \D{ #1 } &=& \D{ #2 } \\
    {\veq#3} \\
    \D{ #4 } &=& \D{ #5 } \\
    \multicolumn{3}{l}{\text{#7}} \\%[5pt]
    \end{array}
    \right)
  }}

% Definição das fórmulas para integração por substituição.
% Algumas são pmatrizes 3x3 usando isubstbox.

\def\TFCtwo{
  \iequationbox {\Intx{a}{b}{F'(x)}}
                {\Difmx{a}{b}{F(x)}}
}
\def\TFCtwoI{
  \iequationbox {\intx{F'(x)}}
                {F(x)}
}

\def\Sone{
  \isubstbox
    {\Difmx{a}{b}{f(g(x))}}  {\Intx{a}{b}{f'(g(x))g'(x)}}
    {\ph{mmm}}
    {\Difmu{g(a)}{g(b)}{f(u)}} {\Intu{g(a)}{g(b)}{f'(u)}}
}
\def\SoneI{
  \isubstbox
    {f(g(x))} {\intx{f'(g(x))g'(x)}}
    {\ph{m}}
    {f(u)}    {\intu{f'(u)}}
}

\def\Stwo{
  \isubstboxT
    {\Difmx{a}{b}{F(g(x))}}   {\Intx{a}{b}{f(g(x))g'(x)}}
    {\ph{mmm}}
    {\Difmu{g(a)}{g(b)}{F(u)}}  {\Intu{g(a)}{g(b)}{f(u)}}
    {Se $F'(u)=f(u)$ então:}
}
\def\StwoI{
  \isubstboxT
    {F(g(x))}  {\intx{f(g(x))g'(x)}}
    {\ph{m}}
    {F(u)}     {\intu{f(u)}}
    {Se $F'(u)=f(u)$ então:}
}
\def\StwoI{
  \isubstboxTT
    {F(g(x))}  {\intx{f(g(x))g'(x)}}
    {\ph{m}}
    {F(u)}     {\intu{f(u)}}
    {Se $F'(u)=f(u)$ então:}
    {Obs: $u=g(x)$.}
}

\def\Sthree{
  \iequationbox {\Intx{a}{b}{f(g(x))g'(x)}}
                {\Intu{g(a)}{g(b)}{f(u)}}
}
\def\SthreeI{
  \iequationbox {\intx{f(g(x))g'(x)}}
                {\intu{f(u)}
                 \qquad [u=g(x)]
                }
  % [u=g(x)]
}

\def\Sthree{
  \pmat{
    \D \Intx{a}{b}{f(g(x))g'(x)} \\
    \veq \\
    \D \Intu{g(a)}{g(b)}{f(u)}
  }}

\def\SthreeI{
  \pmat{
    \D \intx{f(g(x))g'(x)} \\
       \veq \\
    \D \intu{f(u)} \\
    \text{Obs: $u=g(x)$.} \\
  }}



\def\Subst#1{\bmat{#1}}




% «defs-T-and-B»  (to ".defs-T-and-B")
% (c3m202p1p 6 "questao-2")
% (c3m202p1a   "questao-2")
\long\def\ColorOrange#1{{\color{orange!90!black}#1}}
\def\T(Total: #1 pts){{\bf(Total: #1)}}
\def\T(Total: #1 pts){{\bf(Total: #1 pts)}}
\def\T(Total: #1 pts){\ColorRed{\bf(Total: #1 pts)}}
\def\B       (#1 pts){\ColorOrange{\bf(#1 pts)}}



%  _____ _ _   _                               
% |_   _(_) |_| | ___   _ __   __ _  __ _  ___ 
%   | | | | __| |/ _ \ | '_ \ / _` |/ _` |/ _ \
%   | | | | |_| |  __/ | |_) | (_| | (_| |  __/
%   |_| |_|\__|_|\___| | .__/ \__,_|\__, |\___|
%                      |_|          |___/      
%
% «title»  (to ".title")
% (c2m211p1p 1 "title")
% (c2m211p1a   "title")

\thispagestyle{empty}

\begin{center}

\vspace*{1.2cm}

{\bf \Large Cálculo 2 - 2021.1}

\bsk

P1 (primeira prova)

\bsk

Eduardo Ochs - RCN/PURO/UFF

\url{http://angg.twu.net/2021.1-C2.html}

\end{center}

\newpage


% «regras-e-dicas»  (to ".regras-e-dicas")
% (c2m211p1p 2 "regras-e-dicas")
% (c2m211p1a   "regras-e-dicas")
% (c2m202mt1p 2 "regras")
% (c2m202mt1a   "regras")
% (c2m202mt2p 2 "regras")
% (c2m202mt1a   "regras")

As regras e dicas são as mesmas dos mini-testes:

\ssk

\url{http://angg.twu.net/LATEX/2020-2-C2-MT1.pdf}

\url{http://angg.twu.net/LATEX/2020-2-C2-MT2.pdf}

\ssk

exceto que a prova vai ser disponibilizada às 22:00 do dia

2/setembro/2021 e deve ser entregue até as 10:00 do dia

4/setembro/2021.

\newpage

{\bf Aviso}

Se você comparar as integrais desta prova com as de uma prova

``normal'' de Cálculo 2 você vai ver que as daqui são bastante

simples... isso é porque eu vou dar uma importância \ColorRed{ENORME}

a detalhes de sintaxe. Por exemplo, compare:
%
$$\intx {e^{\sen x}} = g(x)$$

com:
%
$$\begin{array}{c}
  \text{Queremos encontrar uma função $g(x)$ que obedeça:} \\[10pt]
  \D \intx {e^{\sen x}} = g(x)
  \end{array}
$$

\newpage

% «aviso-2»  (to ".aviso-2")
% (c2m211p1p 4 "aviso-2")
% (c2m211p1a   "aviso-2")

{\bf Aviso (2)}

\ssk

Se você só escrever ``$\intx {e^{\sen x}} = g(x)$''

a interpretação \ColorRed{default} disso pra um

``leitor que não seja muito seu amigo'' ---

obs: releia isto aqui:

\ssk

{\footnotesize

% (c2m211somas24p 12)
%    http://angg.twu.net/LATEX/2021-1-C2-somas-2-4.pdf#page=12
\url{http://angg.twu.net/LATEX/2021-1-C2-somas-2-4.pdf#page=12}

}

\ssk

vai ser: ``\ColorRed{para toda} função $g(x)$ temos $\intx {e^{\sen x}} = g(x)$''...

\ssk

Então escreva com muito cuidado as suas respostas!!!

\bsk
\bsk

{\bf Obs:} eu costumo pôr nas regras das provas que durante

a duração das provas eu não respondo perguntas... mas

eu vou abrir uma exceção pras ``perguntas sobre sintaxe''.



\newpage

% «questao-1»  (to ".questao-1")
% (c2m211p1p 5 "questao-1")
% (c2m211p1a   "questao-1")
% (c2m211isp 28 "exercicio-5")
% (c2m211isa    "exercicio-5")
% (c2m202p1p 9 "gabarito-2")
% (c2m202p1a   "gabarito-2")

{\bf Questão 1}

\T(Total: 1.0 pts)

\msk

O exercício 5 do último PDF pedia pra vocês refazerem

vocês mesmos uma questão da P1 do semestre passado que

tinha gabarito no final da prova. No gabarito eu pus uma

solução pra ela que usava as caixinhas de anotações...

\msk

Traduza este passo da solução dela pra notação com chaves:

$$\intv {\left(
         \cos
         \left( 2 + \sqrt{v}
         \right)
         \right)
         /
         \left( 2 \sqrt{v}
         \right)
         }
   = \intw {\cos{\left (2 + w \right )}
           }
$$

% Obs: $w = \sqrt{v}$.

\bsk
\bsk

Links:

{\footnotesize

%    http://angg.twu.net/LATEX/2021-1-C2-int-subst.pdf#page=28
\url{http://angg.twu.net/LATEX/2021-1-C2-int-subst.pdf#page=28}

%    http://angg.twu.net/LATEX/2020-2-C2-P1.pdf#page=9
\url{http://angg.twu.net/LATEX/2020-2-C2-P1.pdf#page=9}

}


\newpage

% «questao-2»  (to ".questao-2")
% (c2m211p1p 6 "questao-2")
% (c2m211p1a   "questao-2")
% (c2m211isp 6 "exemplo-contas")
% (c2m211isa   "exemplo-contas")
% (c2m211tfcsp 16 "integral-indefinida")
% (c2m211tfcsa    "integral-indefinida")

{\bf Questão 2}

\T(Total: 5.0 pts)

\msk

Resolva estas duas integrais usando mudanças de variáveis

e teste as suas respostas. Nas mudanças de variáveis use

ou as caixinhas de anotações ou as anotações sob chaves.

\msk

a) \B(2.5 pts) $$ \intx {e^{x^5} x^4} $$

b) \B(2.5 pts) $$ \intx {\sqrt{2 + \sen x} · \cos x} $$




Dicas:

{\footnotesize

% (c2m211isp 6)
%    http://angg.twu.net/LATEX/2021-1-C2-int-subst.pdf#page=6
\url{http://angg.twu.net/LATEX/2021-1-C2-int-subst.pdf#page=6}

% (c2m211tfcsp 16)
%    http://angg.twu.net/LATEX/2021-1-C2-os-dois-TFCs.pdf#page=16
\url{http://angg.twu.net/LATEX/2021-1-C2-os-dois-TFCs.pdf#page=16}

}







\newpage

% «questao-3»  (to ".questao-3")
% (c2m211p1p 7 "questao-3")
% (c2m211p1a   "questao-3")

{\bf Questão 3}

\T(Total: 4.0 pts)

\msk

No curso nós vimos como usar a $\pfo{S2I}$, que é:

% (c2m202isp 3 "def-S2-S2I")
% (c2m202isa   "def-S2-S2I")

$$ \pfo{S2I} \;\;=\;\; \StwoI $$

\bsk

para convencer os incrédulos de que certas

mudanças de variáveis são válidas...



\newpage

% «questao-3-cont»  (to ".questao-3-cont")
% (c2m211p1p 8 "questao-3-cont")
% (c2m211p1a   "questao-3-cont")
% (c2m211substp 9 "igual-depois-de-subst")
% (c2m211substa   "igual-depois-de-subst")
% (c2m211isp 17 "encontre-a-subst")
% (c2m211isa    "encontre-a-subst")
% (c2m211isp 19 "exercicio-2-cont")
% (c2m211isa    "exercicio-2-cont")

{\bf Questão 3 (cont.)}

a) \B(0.5 pts) Use a mudança de variável $u=x^5$ para converter
%
$$ \intx {\tan(x^5) · x^4} $$

em uma integral mais simples.

\bsk

b) \B(3.5 pts) Encontre a substituição que transforma a \pfo{S2I}

numa demonstração da igualdade que você encontrou no item (a).

\bsk
\bsk

Dicas pra (b):

\ssk

{\footnotesize

% (c2m211substp 9)
%    http://angg.twu.net/LATEX/2021-1-C2-subst.pdf#page=9
\url{http://angg.twu.net/LATEX/2021-1-C2-subst.pdf#page=9}

% (c2m211isp 17)
%    http://angg.twu.net/LATEX/2021-1-C2-int-subst.pdf#page=17
\url{http://angg.twu.net/LATEX/2021-1-C2-int-subst.pdf#page=17}

% (c2m211isp 19)
%    http://angg.twu.net/LATEX/2021-1-C2-int-subst.pdf#page=19
\url{http://angg.twu.net/LATEX/2021-1-C2-int-subst.pdf#page=19}

}

\ssk


% (c2m211isp 17 "encontre-a-subst")
% (c2m211isa    "encontre-a-subst")
% (c2m211isp 19 "exercicio-2-cont")
% (c2m211isa    "exercicio-2-cont")


\newpage

\thispagestyle{empty}

\begin{center}

\vspace*{2.0cm}

{\bf \Large Gabarito}

\end{center}

\newpage

% «questao-1-gab»  (to ".questao-1-gab")
% (c2m211p1p 10 "questao-1-gab")
% (c2m211p1a    "questao-1-gab")

{\bf Questão 1: gabarito}

\def\und#1#2{\underbrace{#1}_{#2}}

$$\int   \cos( 2 + \und{\sqrt{v}}{w} )
         ·
         \und{\und{\left( 2 \sqrt{v} \right)^{-1}}
                  {\frac{dw}{dv}}
              \, dv}
             {dw}
   = \intw {\cos{\left (2 + w \right )}
           }
$$

\newpage

% «questao-2-gab»  (to ".questao-2-gab")
% (c2m211p1p 11 "questao-2-gab")
% (c2m211p1a    "questao-2-gab")

{\bf Questão 2: gabarito do item a}

\def\eqa{\overset{\ColorRed{?}}{=}}

$$\begin{array}{l}
    \intx {e^{x^5} x^4} \\
    = \;\; \intu {e^u · \frac{1}{5}} \\
    = \;\; \frac{1}{5} \intu {e^u} \\
    = \;\; \frac{1}{5} e^u \\
    = \;\; \frac{1}{5} e^{x^5} \\
    \end{array}
  \quad
  \bmat{ u = x^5 \\
         \frac{du}{dx} = 5 x^4 \\
         du = 5 x^4 \, dx \\
         \frac{1}{5} du = x^4 \, dx \\
       }
  \qquad
  \begin{array}{rcl}
  \intx {e^{x^5} x^4} &\eqa& \ph{m,}         \frac{1}{5} e^{x^5} \\
         e^{x^5} x^4  \ph{mi} &\eqa& \ddx \, \frac{1}{5} e^{x^5} \\
                                 &=&  \frac{1}{5} e^{x^5} 5 x^4 \\
                                 &=&   e^{x^5} x^4 \\
  \end{array}
$$

\newpage

{\bf Questão 2: gabarito do item b}

$$\begin{array}{l}
    \intx {\sqrt{2 + \sen x} · \cos x} \\
    = \;\; \intu {\sqrt{u}} \\
    = \;\; \intu {u^{\frac{1}{2}}} \\
    = \;\; \frac{2}{3} u^{\frac{3}{2}} \\
    = \;\; \frac{2}{3} (2+\sen x)^{\frac{3}{2}} \\
    \end{array}
  \quad
  \bmat{ u = 2 + \sen x \\
         \frac{du}{dx} = \cos x \\
         du = \cos x \, dx \\
       }
$$

\bsk

$$\begin{array}{l}
    \ddx \left( \frac{2}{3} (2+\sen x)^{\frac{3}{2}} \right) \\
    = \;\; \frac{2}{3} · \ddx (2+\sen x)^{\frac{3}{2}}  \\[2.5pt]
    = \;\; \frac{2}{3} · \frac{3}{2} (2+\sen x)^{\frac{1}{2}} · \ddx (2 + \sen x) \\[2.5pt]
    = \;\; \sqrt{2 + \sen x} · \cos x
    \end{array}
$$


\newpage

% «questao-3-gab»  (to ".questao-3-gab")
% (c2m211p1p 12 "questao-3-gab")
% (c2m211p1a    "questao-3-gab")

{\bf Questão 3: gabarito}

a) \quad
$\intx {\tan(x^5)·x^4} \;\;=\;\; \intu{\tan(u)·\frac{1}{5}}
  \qquad
  \bmat{ u = x^5 \\
         \frac{du}{dx} = 5 x^4 \\
         du = 5 x^4 \, dx \\
         \frac{1}{5} du = x^4 \, dx \\
       }
$

\def\StwoIsetargs #1{\StwoIsetargss #1}
\def\StwoIsetargss#1#2#3#4#5#6{
  \sa{x}{#1} \sa{u}{#2} \sa{gx}{#3} \sa{g'x}{#4}
  \sa{nw}{F(\ga{gx})}   \sa{ne}{#5}
  \sa{sw}{F(\ga{u})}    \sa{se}{#6}
  }
\def\StwoItmp{
  \isubstboxTT
    {\ga{nw}}  {\int \ga{ne} \, d\ga{x}}
    {\ph{m}}
    {\ga{sw}}  {\int \ga{se} \, d\ga{u}}
    {Se $F'(\ga{u})=\ga{se}$ então:}
    {Obs:  $\ga{u} =\ga{gx}$.}
}
\def\StwoIsubsts{
  \bmat{x:=\ga{x} \\
        u:=\ga{u} \\
        f(\ga{u}):=\ga{se} \\
        g(\ga{x}):=\ga{gx} \\
       g'(\ga{x}):=\ga{g'x} \\
       }
}

\bsk

b) \quad
$\scalebox{0.65}{$
    \StwoIsetargs{ {x} {u} {x^5} {5x^4}
                    {\tan(u)·\frac{1}{5}·5x^4}
                    {\tan(u)·\frac{1}{5}} }
    \StwoI \StwoIsubsts \;\;=\;\; \StwoItmp
 $}
$

\newpage


% «apendice»  (to ".apendice")
% (c2m211p1p 14 "apendice")
% (c2m211p1a    "apendice")

\thispagestyle{empty}

\begin{center}

\vspace*{2.0cm}

{\bf \Large Apêndice}

\end{center}

\newpage

% «gabarito-2-2020.2»  (to ".gabarito-2-2020.2")
% (c2m211p1p 15 "gabarito-2-2020.2")
% (c2m211p1a   "gabarito-2-2020.2")
% (c2m202p1p 9 "gabarito-2")
% (c2m202p1a   "gabarito-2")

A primeira parte do gabarito da questão 2 de 2020.2 era isto aqui...

A tradução destas quatro mudanças de variável pra casos particu-

lares do \pfo{S2I} está no próximo slide.

\bsk

$\scalebox{1.0}{$
 \begin{array}{l}
   \begin{array}{l}
   \intx {\frac{3 \cos{\left (2 + \sqrt{3 x + 4} \right )}}
         {2 \sqrt{3 x + 4}}
         } \\
   = \intu {\frac{\cos{\left (2 + \sqrt{u + 4} \right )}}
           {2 \sqrt{u + 4}}
           } \\
   = \intv {\frac{\cos{\left (2 + \sqrt{v} \right )}}
           {2 \sqrt{v}}
           } \\
   = \intw {\cos{\left (2 + w \right )}
           } \\
   = \inty {\cos y}
           \\
   = \sen y \\
   = \sen \left( 2+w \right) \\
   = \sen \left( 2+\sqrt{v} \right) \\
   = \sen \left( 2+\sqrt{u+4} \right) \\
   = \sen \left( 2+\sqrt{3x+4} \right) \\
   \end{array}
   %
   \begin{array}{c}
     \bsm{u = 3x \\ \frac{du}{dx} = 3 \\ du = 3\,dx \\ dx = \frac13 du}
     \\[15pt]
     \bsm{v = u+4 \\ du=dv }
     \\[5pt]
     \bsm{w = \sqrt{v} \\ \frac{dw}{dv} = \frac12 v^{-1/2} = \frac{1}{2\sqrt{v}} \\}
     \\[5pt]
     \bsm{y = 2+w \\ dy=dw }
     \\[60pt]
   \end{array}
   %
  \end{array}
  $}
$

\newpage

% «quatro-substituicoes»  (to ".quatro-substituicoes")
% (c2m211p1p 16 "quatro-substituicoes")
% (c2m211p1a    "quatro-substituicoes")

$\scalebox{0.5}{$
  \begin{array}{l}
    \StwoIsetargs{ {x} {u} {3x} {3}
                    {\frac{\cos(2+\sqrt{3x+4})}{2\sqrt{3x+4}}·3}
                    {\frac{\cos(2+\sqrt{ u+4})}{2\sqrt{ u+4}}  } }
    \StwoI \StwoIsubsts \;\;=\;\; \StwoItmp
    %
    \\[55pt]
    %
    \StwoIsetargs{ {u} {v} {u+4} {1}
                    {\frac{\cos(2+\sqrt{u+4})}{2\sqrt{u+4}}·1}
                    {\frac{\cos(2+\sqrt{ v })}{2\sqrt{ v }}} }
    \StwoI \StwoIsubsts \;\;=\;\; \StwoItmp
    %
    \\[55pt]
    %
    \StwoIsetargs{ {v} {w} {\sqrt{v}} {(2\sqrt{v})^{-1}}
                    {\cos(2+\sqrt{v}) ·(2\sqrt{v})^{-1}}
                    {\cos(2+w)} }
    \StwoI \StwoIsubsts \;\;=\;\; \StwoItmp
    %
    \\[55pt]
    %
    \StwoIsetargs{ {w} {y} {2+w} {1}
                    {\cos(2+w)·1}
                    {\cos(y)} }
    \StwoI \StwoIsubsts \;\;=\;\; \StwoItmp
    \\
  \end{array}
  $}
$

\newpage

% «exercicio»  (to ".exercicio")
% (c2m211p1p 17 "exercicio")
% (c2m211p1a    "exercicio")

{\bf Exercício (pra quem quiser treinar o `$[:=]$')}

Refaça as quatro substituições do slide anterior:

copie para uma folha de papel o \pfo{S2I} e calcule os

resultados das quatro substituições abaixo. Lembre dos

dois truques pra não precisar fazer nada de cabeça:

\ssk

{\scriptsize

% (c2m211substp 9 "igual-depois-de-subst")
% (c2m211substa   "igual-depois-de-subst")
%    http://angg.twu.net/LATEX/2021-1-C2-subst.pdf#page=9
\url{http://angg.twu.net/LATEX/2021-1-C2-subst.pdf#page=9}

% (c2m211substp 28 "depoimento-pessoal")
% (c2m211substa    "depoimento-pessoal")
%    http://angg.twu.net/LATEX/2021-1-C2-subst.pdf#page=28
\url{http://angg.twu.net/LATEX/2021-1-C2-subst.pdf#page=28}

}

\msk

$\scalebox{0.75}{$
  \begin{array}{l}
    \StwoIsetargs{ {x} {u} {3x} {3}
                    {\frac{\cos(2+\sqrt{3x+4})}{2\sqrt{3x+4}}·3}
                    {\frac{\cos(2+\sqrt{ u+4})}{2\sqrt{ u+4}}  } }
    a) \;\; \pfo{S2I} \StwoIsubsts
    %
    \\[40pt]
    %
    \StwoIsetargs{ {u} {v} {u+4} {1}
                    {\frac{\cos(2+\sqrt{u+4})}{2\sqrt{u+4}}·1}
                    {\frac{\cos(2+\sqrt{ v })}{2\sqrt{ v }}} }
    b) \;\; \pfo{S2I} \StwoIsubsts
  \end{array}
  $}
$
%
\qquad
%
$\scalebox{0.75}{$
  \begin{array}{l}
    \StwoIsetargs{ {v} {w} {\sqrt{v}} {(2\sqrt{v})^{-1}}
                    {\cos(2+\sqrt{v}) ·(2\sqrt{v})^{-1}}
                    {\cos(2+w)} }
    c) \;\; \pfo{S2I} \StwoIsubsts
    %
    \\[40pt]
    %
    \StwoIsetargs{ {w} {y} {2+w} {1}
                    {\cos(2+w)·1}
                    {\cos(y)} }
    d) \;\; \pfo{S2I} \StwoIsubsts
    \\
  \end{array}
  $}
$





%\printbibliography

\GenericWarning{Success:}{Success!!!}  % Used by `M-x cv'

\end{document}

%  ____  _             _         
% |  _ \(_)_   ___   _(_)_______ 
% | | | | \ \ / / | | | |_  / _ \
% | |_| | |\ V /| |_| | |/ /  __/
% |____// | \_/  \__,_|_/___\___|
%     |__/                       
%
% «djvuize»  (to ".djvuize")
% (find-LATEXgrep "grep --color -nH --null -e djvuize 2020-1*.tex")

 (eepitch-shell)
 (eepitch-kill)
 (eepitch-shell)
# (find-fline "~/2021.1-C2/")
# (find-fline "~/LATEX/2021-1-C2/")
# (find-fline "~/bin/djvuize")

cd /tmp/
for i in *.jpg; do echo f $(basename $i .jpg); done

f () { rm -v $1.pdf;  textcleaner -f 50 -o  5 $1.jpg $1.png; djvuize $1.pdf; xpdf $1.pdf }
f () { rm -v $1.pdf;  textcleaner -f 50 -o 10 $1.jpg $1.png; djvuize $1.pdf; xpdf $1.pdf }
f () { rm -v $1.pdf;  textcleaner -f 50 -o 20 $1.jpg $1.png; djvuize $1.pdf; xpdf $1.pdf }

f () { rm -fv $1.png $1.pdf; djvuize $1.pdf }
f () { rm -fv $1.png $1.pdf; djvuize WHITEBOARDOPTS="-m 1.0 -f 15" $1.pdf; xpdf $1.pdf }
f () { rm -fv $1.png $1.pdf; djvuize WHITEBOARDOPTS="-m 1.0 -f 30" $1.pdf; xpdf $1.pdf }
f () { rm -fv $1.png $1.pdf; djvuize WHITEBOARDOPTS="-m 1.0 -f 45" $1.pdf; xpdf $1.pdf }
f () { rm -fv $1.png $1.pdf; djvuize WHITEBOARDOPTS="-m 0.5" $1.pdf; xpdf $1.pdf }
f () { rm -fv $1.png $1.pdf; djvuize WHITEBOARDOPTS="-m 0.25" $1.pdf; xpdf $1.pdf }
f () { cp -fv $1.png $1.pdf       ~/2021.1-C2/
       cp -fv        $1.pdf ~/LATEX/2021-1-C2/
       cat <<%%%
% (find-latexscan-links "C2" "$1")
%%%
}

f 20201213_area_em_funcao_de_theta
f 20201213_area_em_funcao_de_x
f 20201213_area_fatias_pizza



%  __  __       _        
% |  \/  | __ _| | _____ 
% | |\/| |/ _` | |/ / _ \
% | |  | | (_| |   <  __/
% |_|  |_|\__,_|_|\_\___|
%                        
% <make>

 (eepitch-shell)
 (eepitch-kill)
 (eepitch-shell)
# (find-LATEXfile "2019planar-has-1.mk")
make -f 2019.mk STEM=2021-1-C2-P1 veryclean
make -f 2019.mk STEM=2021-1-C2-P1 pdf

% Local Variables:
% coding: utf-8-unix
% ee-tla: "c2p1"
% ee-tla: "c2m211p1"
% End:
