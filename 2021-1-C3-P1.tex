% (find-LATEX "2021-1-C3-P1.tex")
% (defun c () (interactive) (find-LATEXsh "lualatex -record 2021-1-C3-P1.tex" :end))
% (defun C () (interactive) (find-LATEXsh "lualatex 2021-1-C3-P1.tex" "Success!!!"))
% (defun D () (interactive) (find-pdf-page      "~/LATEX/2021-1-C3-P1.pdf"))
% (defun d () (interactive) (find-pdftools-page "~/LATEX/2021-1-C3-P1.pdf"))
% (defun e () (interactive) (find-LATEX "2021-1-C3-P1.tex"))
% (defun q () (interactive) (find-LATEX "2021-1-C3-gradiente.tex"))
% (defun o () (interactive) (find-LATEX "2020-2-C3-P1.tex"))
% (defun u () (interactive) (find-latex-upload-links "2021-1-C3-P1"))
% (defun v () (interactive) (find-2a '(e) '(d)))
% (defun d0 () (interactive) (find-ebuffer "2021-1-C3-P1.pdf"))
% (defun cv () (interactive) (C) (ee-kill-this-buffer) (v) (g))
%          (code-eec-LATEX "2021-1-C3-P1")
% (find-pdf-page   "~/LATEX/2021-1-C3-P1.pdf")
% (find-sh0 "cp -v  ~/LATEX/2021-1-C3-P1.pdf /tmp/")
% (find-sh0 "cp -v  ~/LATEX/2021-1-C3-P1.pdf /tmp/pen/")
%     (find-xournalpp "/tmp/2021-1-C3-P1.pdf")
%   file:///home/edrx/LATEX/2021-1-C3-P1.pdf
%               file:///tmp/2021-1-C3-P1.pdf
%           file:///tmp/pen/2021-1-C3-P1.pdf
% http://angg.twu.net/LATEX/2021-1-C3-P1.pdf
% (find-LATEX "2019.mk")
% (find-CN-aula-links "2021-1-C3-P1" "3" "c3m211p1" "c3p1")
%
% Video (not yet):
% (find-ssr-links "c3m211p1" "2021-1-C3-P1")
% (code-video     "c3m211p1video" "$S/http/angg.twu.net/eev-videos/2021-1-C3-P1.mp4")
% (find-c3m211p1video "0:00")

% (defun d () (interactive) (find-pdftools-page "~/LATEX/2021-1-C3-gradiente.pdf"))

% «.defs»		(to "defs")
% «.defs-T-and-B»	(to "defs-T-and-B")
% «.title»		(to "title")
% «.regras-e-dicas»	(to "regras-e-dicas")
% «.questao-1»		(to "questao-1")
% «.questao-2»		(to "questao-2")
% «.questao-3»		(to "questao-3")
% «.questao-4»		(to "questao-4")
% «.questao-1-gab»	(to "questao-1-gab")
% «.questao-1-gab-cont»	(to "questao-1-gab-cont")
% «.questao-2-gab»	(to "questao-2-gab")
%
% «.djvuize»		(to "djvuize")

\documentclass[oneside,12pt]{article}
\usepackage[colorlinks,citecolor=DarkRed,urlcolor=DarkRed]{hyperref} % (find-es "tex" "hyperref")
\usepackage{amsmath}
\usepackage{amsfonts}
\usepackage{amssymb}
\usepackage{pict2e}
\usepackage[x11names,svgnames]{xcolor} % (find-es "tex" "xcolor")
\usepackage{colorweb}                  % (find-es "tex" "colorweb")
%\usepackage{tikz}
%
% (find-dn6 "preamble6.lua" "preamble0")
%\usepackage{proof}   % For derivation trees ("%:" lines)
%\input diagxy        % For 2D diagrams ("%D" lines)
%\xyoption{curve}     % For the ".curve=" feature in 2D diagrams
%
\usepackage{edrx21}               % (find-LATEX "edrx21.sty")
\input edrxaccents.tex            % (find-LATEX "edrxaccents.tex")
\input edrx21chars.tex            % (find-LATEX "edrx21chars.tex")
\input edrxheadfoot.tex           % (find-LATEX "edrxheadfoot.tex")
\input edrxgac2.tex               % (find-LATEX "edrxgac2.tex")
%
%\usepackage[backend=biber,
%   style=alphabetic]{biblatex}            % (find-es "tex" "biber")
%\addbibresource{catsem-slides.bib}        % (find-LATEX "catsem-slides.bib")
%
% (find-es "tex" "geometry")
\usepackage[a6paper, landscape,
            top=1.5cm, bottom=.25cm, left=1cm, right=1cm, includefoot
           ]{geometry}
%
\begin{document}

\catcode`\^^J=10
\directlua{dofile "dednat6load.lua"}  % (find-LATEX "dednat6load.lua")

%L dofile "edrxtikz.lua"  -- (find-LATEX "edrxtikz.lua")
%L dofile "edrxpict.lua"  -- (find-LATEX "edrxpict.lua")
\pu

% «defs»  (to ".defs")
% (find-LATEX "edrx15.sty" "colors-2019")
%\long\def\ColorRed   #1{{\color{Red1}#1}}
%\long\def\ColorViolet#1{{\color{MagentaVioletLight}#1}}
%\long\def\ColorViolet#1{{\color{Violet!50!black}#1}}
%\long\def\ColorGreen #1{{\color{SpringDarkHard}#1}}
%\long\def\ColorGreen #1{{\color{SpringGreenDark}#1}}
%\long\def\ColorGreen #1{{\color{SpringGreen4}#1}}
%\long\def\ColorGray  #1{{\color{GrayLight}#1}}
%\long\def\ColorGray  #1{{\color{black!30!white}#1}}
%\long\def\ColorBrown #1{{\color{Brown}#1}}
%\long\def\ColorBrown #1{{\color{brown}#1}}
%\long\def\ColorOrange#1{{\color{orange}#1}}
%
%\long\def\ColorShort #1{{\color{SpringGreen4}#1}}
%\long\def\ColorLong  #1{{\color{Red1}#1}}
%
%\def\frown{\ensuremath{{=}{(}}}
%\def\True {\mathbf{V}}
%\def\False{\mathbf{F}}
%\def\D    {\displaystyle}

\def\drafturl{http://angg.twu.net/LATEX/2021-1-C3.pdf}
\def\drafturl{http://angg.twu.net/2021.1-C3.html}
\def\draftfooter{\tiny \href{\drafturl}{\jobname{}} \ColorBrown{\shorttoday{} \hours}}

% «defs-T-and-B»  (to ".defs-T-and-B")
% (c3m202p1p 6 "questao-2")
% (c3m202p1a   "questao-2")
\long\def\ColorOrange#1{{\color{orange!90!black}#1}}
\def\T(Total: #1 pts){{\bf(Total: #1)}}
\def\T(Total: #1 pts){{\bf(Total: #1 pts)}}
\def\T(Total: #1 pts){\ColorRed{\bf(Total: #1 pts)}}
\def\B       (#1 pts){\ColorOrange{\bf(#1 pts)}}




%  _____ _ _   _                               
% |_   _(_) |_| | ___   _ __   __ _  __ _  ___ 
%   | | | | __| |/ _ \ | '_ \ / _` |/ _` |/ _ \
%   | | | | |_| |  __/ | |_) | (_| | (_| |  __/
%   |_| |_|\__|_|\___| | .__/ \__,_|\__, |\___|
%                      |_|          |___/      
%
% «title»  (to ".title")
% (c3m211p1p 1 "title")
% (c3m211p1a   "title")

\thispagestyle{empty}

\begin{center}

\vspace*{1.2cm}

{\bf \Large Cálculo 3 - 2021.1}

\bsk

P1 (primeira prova)

\bsk

Eduardo Ochs - RCN/PURO/UFF

\url{http://angg.twu.net/2021.1-C3.html}

\end{center}

\newpage

% «regras-e-dicas»  (to ".regras-e-dicas")

{\bf Regras e dicas}

As regras e dicas são as mesmas dos mini-testes:

\ssk

\url{http://angg.twu.net/LATEX/2020-2-C3-MT1.pdf}

\url{http://angg.twu.net/LATEX/2020-2-C3-MT2.pdf}

\ssk

exceto que a prova vai ser disponibilizada às 20:00 do dia

3/setembro/2021 e deve ser entregue até as 20:00 do dia

4/setembro/2021.



\bsk
\bsk

Todas as questões são baseadas em exercícios deste PDF:

\ssk

{\footnotesize

% (c3mgradp 3 "exercicio-1")
% (c3mgrada   "exercicio-1")
%    http://angg.twu.net/LATEX/2021-1-C3-gradiente.pdf
\url{http://angg.twu.net/LATEX/2021-1-C3-gradiente.pdf}

}

\ssk

Vou me referir a ele como ``[G]'' (de ``Gradiente'').

\newpage

% «questao-1»  (to ".questao-1")
% (c3m211p1p 3 "questao-1")
% (c3m211p1a   "questao-1")
% (c3mgradp 3 "exercicio-1")
% (c3mgrada   "exercicio-1")

{\bf Questão 1.}

\T(Total: 2.5 pts)

\ssk

a) \B(0.5 pts) Faça o exercício (1c) do [G].

b) \B(1.0 pts) Faça o exercício (1g) do [G].

c) \B(1.0 pts) Faça o exercício (1h) do [G].


\newpage

% «questao-2»  (to ".questao-2")
% (c3m211p1p 4 "questao-2")
% (c3m211p1a   "questao-2")
% (c3mgradp 5 "exercicio-2")
% (c3mgrada   "exercicio-2")

{\bf Questão 2.}

\T(Total: 4.5 pts)

\ssk

Faça os itens abaixo do exercício 2 do [G],

mas considerando que $(x_0,y_0) = (-1,1)$.

\msk

a) \B(1.0 pts) Faça o item (2a) do [G].

b) \B(0.5 pts) Faça o item (2b) do [G].

c) \B(0.5 pts) Faça o item (2c) do [G].

d) \B(0.5 pts) Faça o item (2d) do [G].

\msk

e) \B(1.0 pts) Faça o item (2e) do [G], mas mostrando

como representar graficamente $h'_\text{cima}(x_0)$ e explicando

porque o seu resultado ``faz sentido''.

\msk

f) \B(0.5 pts) Faça o item (2f) do [G].

g) \B(0.5 pts) Faça o item (2g) do [G].


\newpage

% «questao-3»  (to ".questao-3")
% (c3m211p1p 5 "questao-3")
% (c3m211p1a   "questao-3")

{\bf Questão 3.}

\T(Total: 2.0 pts)

\ssk

Faça os itens abaixo do exercício 3 do [G],

mas considerando que $(x_0,y_0) = (-1,1)$.

\msk

a) \B(1.0 pts) Faça o item (3a) do [G].

b) \B(0.5 pts) Faça o item (3b) do [G].

c) \B(0.5 pts) Faça o item (3c) do [G].


\newpage

% «questao-4»  (to ".questao-4")
% (c3m211p1p 6 "questao-4")
% (c3m211p1a   "questao-4")

{\bf Questão 4.}

\T(Total: 1.0 pts)

\ssk

Generalize o seu item (3c) --- refaça ele considerando

que o ponto $(x_0,y_0)$ é um ponto qualquer de $\R^2$,

não necessariamente o ponto $(-1,1)$.



\newpage

% «questao-1-gab»  (to ".questao-1-gab")
% (c3m211p1p 7 "questao-1-gab")
% (c3m211p1a   "questao-1-gab")

{\bf Questão 1: gabarito}


% (find-LATEX "material-para-GA.tex" "cells")

\def\cellhc#1{\hbox to 0pt{\hss\cellfont${#1}$\hss}}
\def\cellvm#1{\setbox0#1\lower \celllower \box0}
\def\cellvb#1{\setbox0#1\lower \ht0       \box0}
\def\cellvm#1{\setbox0#1\lower 0.4\ht0    \box0}
\def\cellc   #1{\cellvm{\cellhc{#1}}}

%L fcells = function (f)
%L     local bigstr = ""
%L     for y=-3,3 do
%L       for x=-3,3 do
%L         bigstr = bigstr..f(x, y)
%L       end
%L     end
%L     return (bigstr)
%L   end
%L
%L fxyy = function (x, y)
%L     local z = (x-y)*y
%L     return format("\\put(%d,%d){\\cellc{%s}}%%\n", x,y, z)
%L  end
%L fgrad = function (x,y)
%L     local Gx = y/10
%L     local Gy = (x - 2*y)/10
%L     p0 = v(x, y)
%L     p1 = p0 + v(Gx, Gy)
%L     return pformat("\\Line%s%s%%\n", p0, p1)
%L   end

\pu

\def\fwith#1{%
  \vcenter{\hbox{%
    \beginpicture(-4,-4)(4,4)
    \pictgrid%
    #1%
    \pictaxes%
    \end{picture}%
  }}}

\unitlength=20pt

\def\bpt#1{{%
  \unitlength=1pt
  \begin{picture}(0,0)
    \put(0,-8){\scalebox{0.5}{\ColorRed{(#1)}}}
  \end{picture}}}

\bpt{0.5}%
1a) $(x-y)y =   
     \scalebox{1}{$
       \fwith{\expr{fcells(fxyy)}}
     $}
    $

\bpt{1.0}%
1b) (Vou desenhar as curvas de nível à mão)

\newpage

% «questao-1-gab-cont»  (to ".questao-1-gab-cont")
% (c3m211p1p 8 "questao-1-gab-cont")
% (c3m211p1a    "questao-1-gab-cont")

{\bf Questão 1: gabarito (cont.)}

(Imagine que os tracinhos são setinhas)

\bpt{1.0}%
1c) $∇G = \VEC{\frac{y}{10}, \frac{x-2y}{10}} =
     \scalebox{1.10}{$
       \fwith{\expr{fcells(fgrad)}}
     $}
    $


\newpage

% «questao-2-gab»  (to ".questao-2-gab")
% (c3m211p1p 9 "questao-2-gab")
% (c3m211p1a   "questao-2-gab")

{\bf Questão 2: gabarito}

\bpt{1.0}%
2a) $h_\text{cima}(x)  = (x + \sqrt{x^2 + 8})/2$, \;\;
    $h_\text{baixo}(x) = (x - \sqrt{x^2 + 8})/2$

\msk

\bpt{0.5}%
2b) (Vou desenhar o gráfico à mão)

\msk

\bpt{0.5}%
2c) $h_\text{cima}(x_0) = (-1 + \sqrt{(-1)^2 + 8})/2 = 1 = y_0$

\msk

\bpt{0.5}%
2d) $h'_\text{cima}(x) = \frac{x}{2\sqrt{x^2+8}} + \frac{1}{2} $

\msk
\bpt{1.0}%
2e) $h'_\text{cima}(x_0) = \frac{-1}{2\sqrt{(-1)^2+8}} + \frac{1}{2} = \frac13$

\ph{2e)} (falta o desenho)

\msk
\bpt{0.5}%
2f) $\vv = \VEC{1, \frac13}$; desenhar

\msk
\bpt{0.5}%
2g) $∇F = \VEC{y, x-2y}$; $∇F(x_0,y_0) = \VEC{1, -1-2·1} = \VEC{1,-3}$;

\ph{2g)} $\vv·∇F(x_0,y_0) = \VEC{1, \frac13}·\VEC{1,-3} = 1-1 = 0$


\newpage

{\bf Questão 3: gabarito}

\def\ddz{\frac{d}{dz}}

\bpt{1.0}%
a) $\ddz z = 0$, $\ddz z = z_x + z_y y_x$, $y_x = -\frac{z_x}{z_y}$;

\ph{a)} se $z = (x-y)y = xy - y^2$ então $z_x = y$, $z_y = x-2y$,

\ph{a)} $y_x = -\frac{y}{x-2y}$.

\msk

\bpt{0.5}%
b) Em $(x,y)=(-1,1)$ temos $y_x = -\frac{1}{-1-2·1} = \frac13$.

\msk

\bpt{0.5}%
c) Em $(x,y)=(0,1)$ temos

\ph{c)} $\VEC{1,y_x}·∇F = \VEC{1, \frac13}·\VEC{1,-3} = 1-1 = 0$.


\newpage

{\bf Questão 4: gabarito}

\ColorRed{(1.0 pts)}

\msk

$\ddz z = 0$, $\ddz z = z_x + z_y y_x$, $y_x = -\frac{z_x}{z_y}$;

se $z = (x-y)y = xy - y^2$ então $z_x = y$, $z_y = x-2y$,

$y_x = -\frac{y}{x-2y}$.

\msk

Temos $∇F = \VEC{y, x-2y}$ (num ponto $(x,y)$ qualquer).

$\VEC{1,y_x}·∇F = \VEC{1, -\frac{y}{x-2y}}· \VEC{y, x-2y} = y + (-\frac{y}{x-2y}·(x-2y)) = 0$.




%\printbibliography

\GenericWarning{Success:}{Success!!!}  % Used by `M-x cv'

\end{document}

%  ____  _             _         
% |  _ \(_)_   ___   _(_)_______ 
% | | | | \ \ / / | | | |_  / _ \
% | |_| | |\ V /| |_| | |/ /  __/
% |____// | \_/  \__,_|_/___\___|
%     |__/                       
%
% «djvuize»  (to ".djvuize")
% (find-LATEXgrep "grep --color -nH --null -e djvuize 2020-1*.tex")

 (eepitch-shell)
 (eepitch-kill)
 (eepitch-shell)
# (find-fline "~/2021.1-C3/")
# (find-fline "~/LATEX/2021-1-C3/")
# (find-fline "~/bin/djvuize")

cd /tmp/
for i in *.jpg; do echo f $(basename $i .jpg); done

f () { rm -v $1.pdf;  textcleaner -f 50 -o  5 $1.jpg $1.png; djvuize $1.pdf; xpdf $1.pdf }
f () { rm -v $1.pdf;  textcleaner -f 50 -o 10 $1.jpg $1.png; djvuize $1.pdf; xpdf $1.pdf }
f () { rm -v $1.pdf;  textcleaner -f 50 -o 20 $1.jpg $1.png; djvuize $1.pdf; xpdf $1.pdf }

f () { rm -fv $1.png $1.pdf; djvuize $1.pdf }
f () { rm -fv $1.png $1.pdf; djvuize WHITEBOARDOPTS="-m 1.0 -f 15" $1.pdf; xpdf $1.pdf }
f () { rm -fv $1.png $1.pdf; djvuize WHITEBOARDOPTS="-m 1.0 -f 30" $1.pdf; xpdf $1.pdf }
f () { rm -fv $1.png $1.pdf; djvuize WHITEBOARDOPTS="-m 1.0 -f 45" $1.pdf; xpdf $1.pdf }
f () { rm -fv $1.png $1.pdf; djvuize WHITEBOARDOPTS="-m 0.5" $1.pdf; xpdf $1.pdf }
f () { rm -fv $1.png $1.pdf; djvuize WHITEBOARDOPTS="-m 0.25" $1.pdf; xpdf $1.pdf }
f () { cp -fv $1.png $1.pdf       ~/2021.1-C3/
       cp -fv        $1.pdf ~/LATEX/2021-1-C3/
       cat <<%%%
% (find-latexscan-links "C3" "$1")
%%%
}

f 20201213_area_em_funcao_de_theta
f 20201213_area_em_funcao_de_x
f 20201213_area_fatias_pizza



%  __  __       _        
% |  \/  | __ _| | _____ 
% | |\/| |/ _` | |/ / _ \
% | |  | | (_| |   <  __/
% |_|  |_|\__,_|_|\_\___|
%                        
% <make>

 (eepitch-shell)
 (eepitch-kill)
 (eepitch-shell)
# (find-LATEXfile "2019planar-has-1.mk")
make -f 2019.mk STEM=2021-1-C3-P1 veryclean
make -f 2019.mk STEM=2021-1-C3-P1 pdf

% Local Variables:
% coding: utf-8-unix
% ee-tla: "c3p1"
% ee-tla: "c3m211p1"
% End:
