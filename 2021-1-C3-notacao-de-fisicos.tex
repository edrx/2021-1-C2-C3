% (find-LATEX "2021-1-C3-notacao-de-fisicos.tex")
% (defun c () (interactive) (find-LATEXsh "lualatex -record 2021-1-C3-notacao-de-fisicos.tex" :end))
% (defun C () (interactive) (find-LATEXsh "lualatex 2021-1-C3-notacao-de-fisicos.tex" "Success!!!"))
% (defun D () (interactive) (find-pdf-page      "~/LATEX/2021-1-C3-notacao-de-fisicos.pdf"))
% (defun d () (interactive) (find-pdftools-page "~/LATEX/2021-1-C3-notacao-de-fisicos.pdf"))
% (defun e () (interactive) (find-LATEX "2021-1-C3-notacao-de-fisicos.tex"))
% (defun o () (interactive) (find-LATEX "2021-1-C3-notacao-de-fisicos.tex"))
% (defun u () (interactive) (find-latex-upload-links "2021-1-C3-notacao-de-fisicos"))
% (defun v () (interactive) (find-2a '(e) '(d)))
% (defun d0 () (interactive) (find-ebuffer "2021-1-C3-notacao-de-fisicos.pdf"))
% (defun cv () (interactive) (C) (ee-kill-this-buffer) (v) (g))
%          (code-eec-LATEX "2021-1-C3-notacao-de-fisicos")
% (find-pdf-page   "~/LATEX/2021-1-C3-notacao-de-fisicos.pdf")
% (find-sh0 "cp -v  ~/LATEX/2021-1-C3-notacao-de-fisicos.pdf /tmp/")
% (find-sh0 "cp -v  ~/LATEX/2021-1-C3-notacao-de-fisicos.pdf /tmp/pen/")
%     (find-xournalpp "/tmp/2021-1-C3-notacao-de-fisicos.pdf")
%   file:///home/edrx/LATEX/2021-1-C3-notacao-de-fisicos.pdf
%               file:///tmp/2021-1-C3-notacao-de-fisicos.pdf
%           file:///tmp/pen/2021-1-C3-notacao-de-fisicos.pdf
% http://angg.twu.net/LATEX/2021-1-C3-notacao-de-fisicos.pdf
% (find-LATEX "2019.mk")
% (find-CN-aula-links "2021-1-C3-notacao-de-fisicos" "3" "c3m211nf" "c3nf")
%
% Video:
% (find-ssr-links "c3m211nf" "2021-1-C3-notacao-de-fisicos")
% (code-video     "c3m211nfvideo" "$S/http/angg.twu.net/eev-videos/2021-1-C3-notacao-de-fisicos.mp4")
% (find-c3m211nfvideo "0:00")

% «.defs»		(to "defs")
% «.title»		(to "title")
% «.introducao»		(to "introducao")
% «.links»		(to "links")
% «.primeiro-exemplo»	(to "primeiro-exemplo")
% «.segundo-exemplo»	(to "segundo-exemplo")
%
% «.djvuize»	(to "djvuize")

\documentclass[oneside,12pt]{article}
\usepackage[colorlinks,citecolor=DarkRed,urlcolor=DarkRed]{hyperref} % (find-es "tex" "hyperref")
\usepackage{amsmath}
\usepackage{amsfonts}
\usepackage{amssymb}
\usepackage{pict2e}
\usepackage[x11names,svgnames]{xcolor} % (find-es "tex" "xcolor")
\usepackage{colorweb}                  % (find-es "tex" "colorweb")
%\usepackage{tikz}
%
% (find-dn6 "preamble6.lua" "preamble0")
%\usepackage{proof}   % For derivation trees ("%:" lines)
%\input diagxy        % For 2D diagrams ("%D" lines)
%\xyoption{curve}     % For the ".curve=" feature in 2D diagrams
%
\usepackage{edrx15}               % (find-LATEX "edrx15.sty")
\input edrxaccents.tex            % (find-LATEX "edrxaccents.tex")
\input edrxchars.tex              % (find-LATEX "edrxchars.tex")
\input edrxheadfoot.tex           % (find-LATEX "edrxheadfoot.tex")
\input edrxgac2.tex               % (find-LATEX "edrxgac2.tex")
%
%\usepackage[backend=biber,
%   style=alphabetic]{biblatex}            % (find-es "tex" "biber")
%\addbibresource{catsem-slides.bib}        % (find-LATEX "catsem-slides.bib")
%
% (find-es "tex" "geometry")
\usepackage[a6paper, landscape,
            top=1.5cm, bottom=.25cm, left=1cm, right=1cm, includefoot
           ]{geometry}
%
\begin{document}

%\catcode`\^^J=10
%\directlua{dofile "dednat6load.lua"}  % (find-LATEX "dednat6load.lua")

% %L dofile "edrxtikz.lua"  -- (find-LATEX "edrxtikz.lua")
% %L dofile "edrxpict.lua"  -- (find-LATEX "edrxpict.lua")
% \pu

% «defs»  (to ".defs")
% (find-LATEX "edrx15.sty" "colors-2019")
\long\def\ColorRed   #1{{\color{Red1}#1}}
\long\def\ColorViolet#1{{\color{MagentaVioletLight}#1}}
\long\def\ColorViolet#1{{\color{Violet!50!black}#1}}
\long\def\ColorGreen #1{{\color{SpringDarkHard}#1}}
\long\def\ColorGreen #1{{\color{SpringGreenDark}#1}}
\long\def\ColorGreen #1{{\color{SpringGreen4}#1}}
\long\def\ColorGray  #1{{\color{GrayLight}#1}}
\long\def\ColorGray  #1{{\color{black!30!white}#1}}
\long\def\ColorBrown #1{{\color{Brown}#1}}
\long\def\ColorBrown #1{{\color{brown}#1}}
\long\def\ColorOrange#1{{\color{orange}#1}}

\long\def\ColorShort #1{{\color{SpringGreen4}#1}}
\long\def\ColorLong  #1{{\color{Red1}#1}}

\def\frown{\ensuremath{{=}{(}}}
\def\True {\mathbf{V}}
\def\False{\mathbf{F}}
\def\D    {\displaystyle}

\def\drafturl{http://angg.twu.net/LATEX/2021-1-C3.pdf}
\def\drafturl{http://angg.twu.net/2021.1-C3.html}
\def\draftfooter{\tiny \href{\drafturl}{\jobname{}} \ColorBrown{\shorttoday{} \hours}}



%  _____ _ _   _                               
% |_   _(_) |_| | ___   _ __   __ _  __ _  ___ 
%   | | | | __| |/ _ \ | '_ \ / _` |/ _` |/ _ \
%   | | | | |_| |  __/ | |_) | (_| | (_| |  __/
%   |_| |_|\__|_|\___| | .__/ \__,_|\__, |\___|
%                      |_|          |___/      
%
% «title»  (to ".title")
% (c3m211nfp 1 "title")
% (c3m211nfa   "title")

\thispagestyle{empty}

\begin{center}

\vspace*{1.2cm}

{\bf \Large Cálculo 3 - 2021.1}

\bsk

Aula ??: Notação de físicos

\bsk

Eduardo Ochs - RCN/PURO/UFF

\url{http://angg.twu.net/2021.1-C3.html}

\end{center}

\newpage

% «introducao»  (to ".introducao")
% (c3m211nfp 2 "introducao")
% (c3m211nfa   "introducao")

{\bf Introdução}

% (find-bortolossi5page (+ -162 164) "5.2. Definições e exemplos")
% (find-bortolossi5page (+ -162 165)   "Fig. 5.2: Interpretação geométrica")
% (find-bortolossi5page (+ -162 167)   "Exemplo 5.1: Cobb-Douglas")
% (find-bortolossi5page (+ -162 170)   "derivada parcial")
% (find-bortolossi5page (+ -162 171)   "a notação D_1 f é a mais clara")
% (find-bortolossi5page (+ -162 172)   "omitir os pontos onde as parciais são calculadas")

Na página 170--172 do cap.5 o Bortolossi fala de algumas convenções
sobre variáveis que ele vai usar o mínimo possível, porque elas às
vezes são difíceis de interpretar e às vezes são ambíguas...

Isso é um assunto bem maior e mais complicado do que parece. Quando eu
fiz graduação em algumas matérias essas convenções -- que eu vou
chamar de ``notação de físicos'' -- eram totalmente
\ColorRed{proibidas}, mas em outras elas eram tratadas como algo
\ColorRed{óbvio} que \ColorRed{todo mundo sabia usar}.

A gente vai aprender alguns dos princípios por trás da ``notação de
físicos'' e vamos como usar essa ``notação de físicos'' como uma
\ColorRed{abreviação} pra uma notação muito menos ambígua que
matemáticos ``estritos'' aceitam.


\newpage

% «links»  (to ".links")
% (c3m211nfp 3 "links")
% (c3m211nfa   "links")

{\bf Links (pra matemáticos que estiverem lendo isso aqui)}

\msk

Eu aprendi ``notação de físicos'' estudando EDPs pelo livro do

Fritz John e estudando Cálculo de Variações:

\ssk

{\footnotesize

\url{https://www.springer.com/gp/book/9781461599661} Fritz John

% \url{https://math.stackexchange.com/questions/44828/introductory-text-for-calculus-of-variations}

% \url{https://mathoverflow.net/questions/129122/good-book-on-calculus-of-variations}

\url{http://www-users.math.umn.edu/~olver/ln_/cv.pdf} Olver

}

\ssk

e um pouco pelos livros do Física do Moysés Nussenzveig.

\bsk

O ``The Language of Mathematics'' do Mohan Ganesalingam 

em coisas muito boas sobre variáveis.
\href{https://www.amazon.com/Language-Mathematics-Linguistic-Philosophical-Investigation/dp/364237011X}{Amazon},
\href{https://www.maa.org/press/maa-reviews/the-language-of-mathematics}{MAA
  Reviews}.


\bsk

Andrej Bauer: ``\href{http://math.andrej.com/2021/06/24/the-dawn-of-formalized-mathematics/}{The
  dawn of formalized mathematics}''.

% http://math.andrej.com/asset/data/the-dawn-of-formalized-mathematics.pdf#page=15
A partir do
\href{http://math.andrej.com/asset/data/the-dawn-of-formalized-mathematics.pdf\#page=15}{slide
  15}.


\newpage

{\bf Mais links (pra matemáticos)}

\bsk

{\footnotesize

\url{https://en.wikipedia.org/wiki/Physical_quantity}

\url{https://en.wikipedia.org/wiki/Dependent_and_independent_variables}

\url{https://en.wikipedia.org/wiki/Variable_(mathematics)}

\url{https://en.wikipedia.org/wiki/Variable_(computer_science)}

\msk

% (find-books "__analysis/__analysis.el" "redish-gupta")
%
Redish/Gupta: ``\href{https://arxiv.org/pdf/1002.0472.pdf}{Making
  Meaning with Math in Physics: A Semantic Analysis}''

\msk

Ellermeijer/Heck: ``Differences between the use of mathematical entities

in mathematics and physics and the consequences for an integrated learning

environment.''
\href{https://staff.fnwi.uva.nl/a.j.p.heck/research/art/girep2001.pdf\#page=6}{Page
  6}.

}


\newpage

% «primeiro-exemplo»  (to ".primeiro-exemplo")
% (c3m211nfp 5 "primeiro-exemplo")
% (c3m211nfa   "primeiro-exemplo")

{\bf Um primeiro exemplo}

Digamos que $y=\sqrt{x}$.

Podemos considerar que $x$ e $y$ ``variam juntos'',

``obedecendo certas restrições''.

Se considerarmos o conjunto dos pontos $(x,y)$ que

obedecem essas restrições obtemos isto aqui:

\bsk
\bsk

Que define duas funções...

Em ``$y=f(x)$'' temos $f(x)=\sqrt{x}$, e

em ``$x=g(y)$'' temos $g(y)=y^2$.



\newpage

% «segundo-exemplo»  (to ".segundo-exemplo")
% (c3m211nfp 6 "segundo-exemplo")
% (c3m211nfa   "segundo-exemplo")

{\bf Um segundo exemplo}

Digamos que o conjunto dos pontos $(x,y)$

``que obedecem as restrições'' é esse aqui:
%
$$\setofst{(x,y)∈\R^2}{x^2+y^2=5}$$

e que $(x_0,y_0) = (3,4)$.

\bsk
\bsk

Os físicos consideram que ``é óbvio'' que (em geral!) variáveis

``variam continuamente'', então se $x_1=x_0+ε$ e $y_1=y_0+δ$

e $ε$ é muito pequeno então $δ$ é muito pequeno também.

(Veja o vídeo!...)


\newpage

{\bf O contexto importa muito}


% z = x + y
% y = x


% O meu modo preferido de formalizar a notação de físicos é esse aqui.



%\printbibliography

\GenericWarning{Success:}{Success!!!}  % Used by `M-x cv'

\end{document}

%  ____  _             _         
% |  _ \(_)_   ___   _(_)_______ 
% | | | | \ \ / / | | | |_  / _ \
% | |_| | |\ V /| |_| | |/ /  __/
% |____// | \_/  \__,_|_/___\___|
%     |__/                       
%
% «djvuize»  (to ".djvuize")
% (find-LATEXgrep "grep --color -nH --null -e djvuize 2020-1*.tex")

 (eepitch-shell)
 (eepitch-kill)
 (eepitch-shell)
# (find-fline "~/2021.1-C3/")
# (find-fline "~/LATEX/2021-1-C3/")
# (find-fline "~/bin/djvuize")

cd /tmp/
for i in *.jpg; do echo f $(basename $i .jpg); done

f () { rm -fv $1.png $1.pdf; djvuize $1.pdf }
f () { rm -fv $1.png $1.pdf; djvuize WHITEBOARDOPTS="-m 1.0" $1.pdf; xpdf $1.pdf }
f () { rm -fv $1.png $1.pdf; djvuize WHITEBOARDOPTS="-m 0.5" $1.pdf; xpdf $1.pdf }
f () { rm -fv $1.png $1.pdf; djvuize WHITEBOARDOPTS="-m 0.25" $1.pdf; xpdf $1.pdf }
f () { cp -fv $1.png $1.pdf       ~/2021.1-C3/
       cp -fv        $1.pdf ~/LATEX/2021-1-C3/
       cat <<%%%
% (find-latexscan-links "C3" "$1")
%%%
}

f 20201213_area_em_funcao_de_theta
f 20201213_area_em_funcao_de_x
f 20201213_area_fatias_pizza



%  __  __       _        
% |  \/  | __ _| | _____ 
% | |\/| |/ _` | |/ / _ \
% | |  | | (_| |   <  __/
% |_|  |_|\__,_|_|\_\___|
%                        
% <make>

 (eepitch-shell)
 (eepitch-kill)
 (eepitch-shell)
# (find-LATEXfile "2019planar-has-1.mk")
make -f 2019.mk STEM=2021-1-C3-notacao-de-fisicos veryclean
make -f 2019.mk STEM=2021-1-C3-notacao-de-fisicos pdf

% Local Variables:
% coding: utf-8-unix
% ee-tla: "c3nf"
% ee-tla: "c3m211nf"
% End:
