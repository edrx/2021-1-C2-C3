% (find-LATEX "2021-1-C2-funcoes-conjs-infinitos.tex")
% (defun c () (interactive) (find-LATEXsh "lualatex -record 2021-1-C2-funcoes-conjs-infinitos.tex" :end))
% (defun C () (interactive) (find-LATEXsh "lualatex 2021-1-C2-funcoes-conjs-infinitos.tex" "Success!!!"))
% (defun D () (interactive) (find-pdf-page      "~/LATEX/2021-1-C2-funcoes-conjs-infinitos.pdf"))
% (defun d () (interactive) (find-pdftools-page "~/LATEX/2021-1-C2-funcoes-conjs-infinitos.pdf"))
% (defun e () (interactive) (find-LATEX "2021-1-C2-funcoes-conjs-infinitos.tex"))
% (defun o () (interactive) (find-LATEX "2021-1-C2-funcoes-conjs-infinitos.tex"))
% (defun u () (interactive) (find-latex-upload-links "2021-1-C2-funcoes-conjs-infinitos"))
% (defun v () (interactive) (find-2a '(e) '(d)))
% (defun d0 () (interactive) (find-ebuffer "2021-1-C2-funcoes-conjs-infinitos.pdf"))
% (defun cv () (interactive) (C) (ee-kill-this-buffer) (v) (g))
%          (code-eec-LATEX "2021-1-C2-funcoes-conjs-infinitos")
% (find-pdf-page   "~/LATEX/2021-1-C2-funcoes-conjs-infinitos.pdf")
% (find-sh0 "cp -v  ~/LATEX/2021-1-C2-funcoes-conjs-infinitos.pdf /tmp/")
% (find-sh0 "cp -v  ~/LATEX/2021-1-C2-funcoes-conjs-infinitos.pdf /tmp/pen/")
%     (find-xournalpp "/tmp/2021-1-C2-funcoes-conjs-infinitos.pdf")
%   file:///home/edrx/LATEX/2021-1-C2-funcoes-conjs-infinitos.pdf
%               file:///tmp/2021-1-C2-funcoes-conjs-infinitos.pdf
%           file:///tmp/pen/2021-1-C2-funcoes-conjs-infinitos.pdf
% http://angg.twu.net/LATEX/2021-1-C2-funcoes-conjs-infinitos.pdf
% (find-LATEX "2019.mk")
% (find-CN-aula-links "2021-1-C2-funcoes-conjs-infinitos" "2" "c2m211fci" "c2fci")
%
% Video:
% (find-ssr-links "c2m211fci" "2021-1-C2-funcoes-conjs-infinitos")
% (code-video     "c2m211fcivideo" "$S/http/angg.twu.net/eev-videos/2021-1-C2-funcoes-conjs-infinitos.mp4")
% (find-c2m211fcivideo "0:00")

% «.defs»	(to "defs")
% «.title»	(to "title")
%
% «.djvuize»	(to "djvuize")

\documentclass[oneside,12pt]{article}
\usepackage[colorlinks,citecolor=DarkRed,urlcolor=DarkRed]{hyperref} % (find-es "tex" "hyperref")
\usepackage{amsmath}
\usepackage{amsfonts}
\usepackage{amssymb}
\usepackage{pict2e}
\usepackage[x11names,svgnames]{xcolor} % (find-es "tex" "xcolor")
\usepackage{colorweb}                  % (find-es "tex" "colorweb")
%\usepackage{tikz}
%
% (find-dn6 "preamble6.lua" "preamble0")
%\usepackage{proof}   % For derivation trees ("%:" lines)
%\input diagxy        % For 2D diagrams ("%D" lines)
%\xyoption{curve}     % For the ".curve=" feature in 2D diagrams
%
\usepackage{edrx15}               % (find-LATEX "edrx15.sty")
\input edrxaccents.tex            % (find-LATEX "edrxaccents.tex")
\input edrxchars.tex              % (find-LATEX "edrxchars.tex")
\input edrxheadfoot.tex           % (find-LATEX "edrxheadfoot.tex")
\input edrxgac2.tex               % (find-LATEX "edrxgac2.tex")
%
%\usepackage[backend=biber,
%   style=alphabetic]{biblatex}            % (find-es "tex" "biber")
%\addbibresource{catsem-slides.bib}        % (find-LATEX "catsem-slides.bib")
%
% (find-es "tex" "geometry")
\usepackage[a6paper, landscape,
            top=1.5cm, bottom=.25cm, left=1cm, right=1cm, includefoot
           ]{geometry}
%
\begin{document}

%\catcode`\^^J=10
%\directlua{dofile "dednat6load.lua"}  % (find-LATEX "dednat6load.lua")

% %L dofile "edrxtikz.lua"  -- (find-LATEX "edrxtikz.lua")
% %L dofile "edrxpict.lua"  -- (find-LATEX "edrxpict.lua")
% \pu

% «defs»  (to ".defs")
% (find-LATEX "edrx15.sty" "colors-2019")
\long\def\ColorRed   #1{{\color{Red1}#1}}
\long\def\ColorViolet#1{{\color{MagentaVioletLight}#1}}
\long\def\ColorViolet#1{{\color{Violet!50!black}#1}}
\long\def\ColorGreen #1{{\color{SpringDarkHard}#1}}
\long\def\ColorGreen #1{{\color{SpringGreenDark}#1}}
\long\def\ColorGreen #1{{\color{SpringGreen4}#1}}
\long\def\ColorGray  #1{{\color{GrayLight}#1}}
\long\def\ColorGray  #1{{\color{black!30!white}#1}}
\long\def\ColorBrown #1{{\color{Brown}#1}}
\long\def\ColorBrown #1{{\color{brown}#1}}
\long\def\ColorOrange#1{{\color{orange}#1}}

\long\def\ColorShort #1{{\color{SpringGreen4}#1}}
\long\def\ColorLong  #1{{\color{Red1}#1}}

\def\frown{\ensuremath{{=}{(}}}
\def\True {\mathbf{V}}
\def\False{\mathbf{F}}
\def\D    {\displaystyle}

\def\drafturl{http://angg.twu.net/LATEX/2021-1-C2.pdf}
\def\drafturl{http://angg.twu.net/2021.1-C2.html}
\def\draftfooter{\tiny \href{\drafturl}{\jobname{}} \ColorBrown{\shorttoday{} \hours}}



%  _____ _ _   _                               
% |_   _(_) |_| | ___   _ __   __ _  __ _  ___ 
%   | | | | __| |/ _ \ | '_ \ / _` |/ _` |/ _ \
%   | | | | |_| |  __/ | |_) | (_| | (_| |  __/
%   |_| |_|\__|_|\___| | .__/ \__,_|\__, |\___|
%                      |_|          |___/      
%
% «title»  (to ".title")
% (c2m211fcip 1 "title")
% (c2m211fci    "title")

\thispagestyle{empty}

\begin{center}

\vspace*{1.2cm}

{\bf \Large Cálculo 2 - 2021.1}

\bsk

Material extra: funções são conjuntos infinitos de pontos

(e contêm uma quantidade infinita de informação)

\bsk

Eduardo Ochs - RCN/PURO/UFF

\url{http://angg.twu.net/2021.1-C2.html}

\end{center}

\newpage

% (find-telegram-save-log-links "2021" "1" "C2" "E1")
% (code-pdf-page "20211C2E1" "~/2021.1-C2/C2-E1-RCN-PURO-2021.1.pdf")
% (code-pdf-text "20211C2E1" "~/2021.1-C2/C2-E1-RCN-PURO-2021.1.pdf")
% (find-20211C2E1page)
% (find-20211C2E1text)
% (find-20211C2E1page 36 "ainda não consegui entender esse\" jeito esperto\"")
% (find-20211C2E1text 36 "ainda não consegui entender esse\" jeito esperto\"")
% (find-20211C2E1page 34 "18 June 2021")
% (find-20211C2E1text 34 "18 June 2021")

% (fooi-re "[ A-Za-z]* + [0-9][0-9]:[0-9][0-9]" "")
% (fooi-re " +[0-9][0-9]:[0-9][0-9]\n" "\n")
% (fooi-re "Isabelle Mendes" "" "Eduardo Ochs" "")
% (fooi-re "`a" "à")
% (fooi-re "cao" "ção")
% (fooi-re "ao" "ão")
% 
% 
%      Isabelle, deixa eu dar mais umas dicas aqui, vamos ver se ajudam...
% 
%      
% IM
%      Tá bom
% 
% 
%      
% EO
%      Nesse video aqui - http://angg.twu.net/eev-videos/
%      2020.2-C2-somas-1.mp4 - no trecho a partir do 4:00, eu desenhei
%      uma curva y=f(x) à mão livre...
% 
%      Lembra que lá no inicio de Calculo 1 a gente vê uma
%      definição de função que diz que uma função do conjunto A pro
%      conjunto B e' um subconjunto de AxB tal que pra todo a em A existe
%      exatamente um par (a,b) em AxB associado a ele...
% 
%      Pera, deixa eu por um link pra essa definição num livro antes
%      de continuar, porque isso aqui e' uma duvida que muita, muita,
%      muita gente tem
% 
%      Aqui: http://angg.twu.net/2020.2-C2/martins_martins__cap_1.pdf#page=4
% 
%      Com essa definição uma função de R em R e' um conjunto
%      infinito de pontos - o grafico dela.
% 
%      So' que o curso de Calculo 1 quase so' usa funcoes dadas
%      desse jeito aqui - deixa eu dar um exemplo: f(x) = sen(10x + 4) * 2
% 
%      Nesse caso a função e' dada por um jeito de calcular o
%      resultado dela pra cada x - e e' um jeito que a gente consegue
%      executar em qualquer calculadora. Se eu te peco pra calcular f(2.34)
%      voce sabe usar a calculadora pra obter o resultado.
% 
% 
%      
% IM
%      In reply to this message
%      Sim
% 
% 
%      
% EO
%      Em Calculo 2 a gente vai usar bastante funcoes que são
%      "definidas" pelos graficos delas. Deixa eu pegar um exemplo que eu
%      pus nos slides do semestre passado e ainda não adaptei pros desse
%      semeste, um instante...
% 
% 
% 
% 
% 
% 
%      Agora me da' mais um instante pra eu fazer uma versão disso
%      com umas anotacoes em cima...
% 
% 
% 
% 
% 
% 
%      Nessa figura ai' eu estou _definindo_ a f(x) pelo grafico dela. Na
%      verdade so' estou definindo como a f(x) se comporta pra x entre 0 e
%      10, mas vamos esquecer esse detalhe...
% 
%      Eu marquei as coordenadas de alguns pontos. Voce consegue
%      ver que o ponto (1,2) pertence ao grafico da f, o ponto (2,3) tambem,
%      mas o ponto (3,3) não?
% 
%      Mas o ponto (3,-3) pertence ao grafico da f...
% 
%      Bolinha preta quer dizer "este ponto pertence ao grafico da f" e
%      bolinha oca quer dizer "este ponto não pertence ao grafico da f".
% 
% 
%      
% IM
%      In reply to this message
%      Sim
% 
% 
%      
% EO
%      Joia!
%      
% IM
%      Até aí tudo tô entendendo
% 
% 
%      
% EO
%      Então, como o ponto (3,-3) pertence ao grafico da f isso quer dizer
%      que f(3) = -3...
% 
% 
%      
% IM
%      Sim
% 
% 
%      
% EO
%      E isso a gente conseguiu ver so' pelo grafico. Se a gente for definir
%      como calcular o valor de f(x) isso vai dar um trabalhão, porque a
%      gente vai ter que usar definicoes por casos, como essa daqui...
% 
% 
% 
% 
% 
% 
%      Da' pra fazer um programa que calcule o valor de f para cada
%      x, mas esse programa vai ter que usar "if". Muitas calculadoras nem
%      permitem algo parecido com isso.
% 
% 
%      
% IM
%      Tudo bem prof ,só não mistura programação kkkkk
% 
% 
%      
% EO
%      In reply to this message
%      Mas vamos voltar pra essa função aqui, que o grafico dela e' feito de
%      um monte de segmentos horizontais. Olhando pro grafico dela voce
%      consegue me dizer qual e' o valor de f(5)?
% 
%      Ok, combinado =)
% 
% 
%      
% IM
%      In reply to this message
%      -1
% 
% 
%      
% EO
%      Isso!!!
%      E voce consegue me dizer qual e' o valor de f(1.5)?
% 
%      (Eu prefiro escrever 1.5 ao inves de 1,5)
% 
% 
%      
% IM
%      In reply to this message
%      2?
% 
% 
%      
% EO
%      SIIIIIIIMMMMMM
% 
%      Então se a gente tem o grafico de uma função f e esse grafico
%      esta' muito bem feito a gente consegue calcular o valor de f(x) pra
%      qualquer x so' olhando pro grafico, certo?...
% 
% 
%      
% IM
%      Certo !
% 
% 
%      
% EO
%      Pra calcular f(1.5) voce comecou no ponto x=1.5, "subiu pro grafico"
%      e depois "foi pro eixo vertical"...
% 
%      Então, o "jeito esperto" e' baseado nisso, mas e' um pouquinho
%      mais abstrato.
% 
%      Pera, preciso resolver uma coisa aqui, volto em 5 minutos!
% 
%      Pronto! Voltando: voce concorda que se a gente subir o ponto
%      x=3 pro grafico da f a gente vai chegar no ponto (3,f(x))?
% 
%      E que se a gente subir o ponto 4.32 pro grafico da f a gente vai
%      chegar no ponto (4.32, f(4.32))?
% 
% 
%      
% IM
%      Sim
% 
% 
%      
% EO
%      Isso vale sempre... se eu comecar no ponto x=a no eixo horizontal e
%      subir ele pro grafico da f eu chego no ponto (a,f(a)).
% 
%      E se eu pegar esse ponto (a,f(a)) e "projetar ele no eixo y"
%      deslocando ele na horizontal eu vou chegar em y=f(a).
%      
% IM
%      O problema é montar os retângulos
% 
% 
%      
% EO
%      Pera, os retangulos vem depois =)
% 
% 
%      
% IM
%      Isso que eu não estou sabendo fazer certo
% 
%      In reply to this message
%      Ok !
% 
% 
%      
% EO
%      Esse truque - de escrever (4.32, f(4.32)) ao inves de (4, -2) e' que vai
%      nos permitir fazer as coisas de um jeito um pouco mais abstrato e
%      mais geral, e vai nos permitir fazer menos contas.
% 
% 
%      
% IM
%      In reply to this message
%      Entendi !
% 
% 
%      
% EO
%      E vai nos permitir usar desenhos tortos!!!!!!! =)
% 
%      In reply to this message
%      Voce pode assistir de novo isso aqui a partir do 4:00 e ver se agora
%      faz sentido?
% 
%      Agora voce aprendeu uns truques novos...
% 
% 
%      
% IM
%      In reply to this message
%      Eu assisti
% 
% 
%      
% EO
%      Então agora tenta fazer o exercicio 2 de novo e manda foto
% 
% 
%      
% IM
%      Vou fazer
%      
% EO
%      ???
% 
% 
%      
% IM
% 
% 
% 
% 
%      No seu o senhor parou na mesma altura do f(0.5) né ?
% 
% 
%      
% EO
%      Sim!
% 
% 
%      
% IM
%      Mas se é para esticar até a curva ?
% 
% 
%      
% EO
%      Bom, agora voce conseguiu desenhar f(0.5) no eixo y... e voce pode
%      usar isso pra desenhar o topo do retangulo...
% 
% 
%      
% IM
%      Pq o f(1.5) tem que parar na altura do f(0.5) é isso que não entra na
%      minha cabeça kkkk??
% 
% 
%      
% EO
%      Aaah, não, e' pra ignorar o f(1.5) nesse exercicio! Voce so' quer
%      desenhar um retangulo, e voce sabe que a parede esquerda dele fica
%      em x=0.5, a parede direita fica em x=1.5, e o topo dele fica em
%      y=f(0.5)...
% 
%      O enunciado do problema diz que "a base dele vai de x=0.5
%      ate' x=1.5", e voce sabe que a base dele fica em y=0. Falta
%      encontrar a altura to topo dele.
% 
% 
%      
% IM
%      Aaaaaaah entendi
% 
%      Desculpa prof mas agora entendi mesmo kkkk
% 
% 
%      
% EO
%      Faz e manda foto!!!! =)
% 
% 
%      
% IM
% 
% 
% 
% 
%      
% EO
%      Isso!!!!!!! ??????
% 
% 
%      
% IM
%      O 3 assim
% 
%      Prof muito obrigada pela atenção , valeu mesmo !! Boa noite !!
% 
% 
%      
% EO
%      A foto ta' com resolução muito baixa... voce pode mandar uma foto
%      so' da parte importante do papel?
% 
% 
%      
% IM
% 
% 
% 
% 
%      
% EO
%      Confere esse depois... todos os retangulos que voce desenhou estão
%      com altura f(2)...
%      
% 



%\printbibliography

\GenericWarning{Success:}{Success!!!}  % Used by `M-x cv'

\end{document}

%  ____  _             _         
% |  _ \(_)_   ___   _(_)_______ 
% | | | | \ \ / / | | | |_  / _ \
% | |_| | |\ V /| |_| | |/ /  __/
% |____// | \_/  \__,_|_/___\___|
%     |__/                       
%
% «djvuize»  (to ".djvuize")
% (find-LATEXgrep "grep --color -nH --null -e djvuize 2020-1*.tex")

 (eepitch-shell)
 (eepitch-kill)
 (eepitch-shell)
# (find-fline "~/2021.1-C2/")
# (find-fline "~/LATEX/2021-1-C2/")
# (find-fline "~/bin/djvuize")

cd /tmp/
for i in *.jpg; do echo f $(basename $i .jpg); done

f () { rm -fv $1.png $1.pdf; djvuize $1.pdf }
f () { rm -fv $1.png $1.pdf; djvuize WHITEBOARDOPTS="-m 1.0" $1.pdf; xpdf $1.pdf }
f () { rm -fv $1.png $1.pdf; djvuize WHITEBOARDOPTS="-m 0.5" $1.pdf; xpdf $1.pdf }
f () { rm -fv $1.png $1.pdf; djvuize WHITEBOARDOPTS="-m 0.25" $1.pdf; xpdf $1.pdf }
f () { cp -fv $1.png $1.pdf       ~/2021.1-C2/
       cp -fv        $1.pdf ~/LATEX/2021-1-C2/
       cat <<%%%
% (find-latexscan-links "C2" "$1")
%%%
}

f 20201213_area_em_funcao_de_theta
f 20201213_area_em_funcao_de_x
f 20201213_area_fatias_pizza



%  __  __       _        
% |  \/  | __ _| | _____ 
% | |\/| |/ _` | |/ / _ \
% | |  | | (_| |   <  __/
% |_|  |_|\__,_|_|\_\___|
%                        
% <make>

 (eepitch-shell)
 (eepitch-kill)
 (eepitch-shell)
# (find-LATEXfile "2019planar-has-1.mk")
make -f 2019.mk STEM=2021-1-C2-funcoes-conjs-infinitos veryclean
make -f 2019.mk STEM=2021-1-C2-funcoes-conjs-infinitos pdf

% Local Variables:
% coding: utf-8-unix
% ee-tla: "c2fci"
% ee-tla: "c2m211fci"
% End:
