% (find-LATEX "2021-1-C3-P2.tex")
% (defun c () (interactive) (find-LATEXsh "lualatex -record 2021-1-C3-P2.tex" :end))
% (defun C () (interactive) (find-LATEXsh "lualatex 2021-1-C3-P2.tex" "Success!!!"))
% (defun D () (interactive) (find-pdf-page      "~/LATEX/2021-1-C3-P2.pdf"))
% (defun d () (interactive) (find-pdftools-page "~/LATEX/2021-1-C3-P2.pdf"))
% (defun e () (interactive) (find-LATEX "2021-1-C3-P2.tex"))
% (defun o () (interactive) (find-LATEX "2021-1-C3-P2.tex"))
% (defun u () (interactive) (find-latex-upload-links "2021-1-C3-P2"))
% (defun v () (interactive) (find-2a '(e) '(d)))
% (defun d0 () (interactive) (find-ebuffer "2021-1-C3-P2.pdf"))
% (defun cv () (interactive) (C) (ee-kill-this-buffer) (v) (g))
%          (code-eec-LATEX "2021-1-C3-P2")
% (find-pdf-page   "~/LATEX/2021-1-C3-P2.pdf")
% (find-sh0 "cp -v  ~/LATEX/2021-1-C3-P2.pdf /tmp/")
% (find-sh0 "cp -v  ~/LATEX/2021-1-C3-P2.pdf /tmp/pen/")
%     (find-xournalpp "/tmp/2021-1-C3-P2.pdf")
%   file:///home/edrx/LATEX/2021-1-C3-P2.pdf
%               file:///tmp/2021-1-C3-P2.pdf
%           file:///tmp/pen/2021-1-C3-P2.pdf
% http://angg.twu.net/LATEX/2021-1-C3-P2.pdf
% (find-LATEX "2019.mk")
% (find-CN-aula-links "2021-1-C3-P2" "3" "c3m211p2" "c3p2")
%
% Video (not yet):
% (find-ssr-links "c3m211p2" "2021-1-C3-P2")
% (code-video     "c3m211p2video" "$S/http/angg.twu.net/eev-videos/2021-1-C3-P2.mp4")
% (find-c3m211p2video "0:00")

% «.defs»			(to "defs")
% «.defs-T-and-B»		(to "defs-T-and-B")
% «.title»			(to "title")
% «.regras-e-dicas»		(to "regras-e-dicas")
% «.questao-1»			(to "questao-1")
% «.questao-1-cont»		(to "questao-1-cont")
% «.questao-1-abcde»		(to "questao-1-abcde")
% «.questao-1-fg»		(to "questao-1-fg")
% «.questao-1-hij»		(to "questao-1-hij")
% «.questao-1-k»		(to "questao-1-k")
% «.questao-3»			(to "questao-3")
% «.questao-3-cont»		(to "questao-3-cont")
%
% «.gabarito»			(to "gabarito")
% «.gabarito-1abcde»		(to "gabarito-1abcde")
% «.gabarito-1fgh»		(to "gabarito-1fgh")
% «.gabarito-1ij»		(to "gabarito-1ij")
% «.gabarito-2ab»		(to "gabarito-2ab")
% «.gabarito-3»			(to "gabarito-3")
%
% «.djvuize»			(to "djvuize")

\documentclass[oneside,12pt]{article}
\usepackage[colorlinks,citecolor=DarkRed,urlcolor=DarkRed]{hyperref} % (find-es "tex" "hyperref")
\usepackage{amsmath}
\usepackage{amsfonts}
\usepackage{amssymb}
\usepackage{pict2e}
\usepackage[x11names,svgnames]{xcolor} % (find-es "tex" "xcolor")
\usepackage{colorweb}                  % (find-es "tex" "colorweb")
%\usepackage{tikz}
%
% (find-dn6 "preamble6.lua" "preamble0")
%\usepackage{proof}   % For derivation trees ("%:" lines)
%\input diagxy        % For 2D diagrams ("%D" lines)
%\xyoption{curve}     % For the ".curve=" feature in 2D diagrams
%
\usepackage{edrx21}               % (find-LATEX "edrx21.sty")
\input edrxaccents.tex            % (find-LATEX "edrxaccents.tex")
\input edrx21chars.tex            % (find-LATEX "edrx21chars.tex")
\input edrxheadfoot.tex           % (find-LATEX "edrxheadfoot.tex")
\input edrxgac2.tex               % (find-LATEX "edrxgac2.tex")
%
%\usepackage[backend=biber,
%   style=alphabetic]{biblatex}            % (find-es "tex" "biber")
%\addbibresource{catsem-slides.bib}        % (find-LATEX "catsem-slides.bib")
%
% (find-es "tex" "geometry")
\usepackage[a6paper, landscape,
            top=1.5cm, bottom=.25cm, left=1cm, right=1cm, includefoot
           ]{geometry}
%
\begin{document}

\catcode`\^^J=10
\directlua{dofile "dednat6load.lua"}  % (find-LATEX "dednat6load.lua")

%L dofile "edrxtikz.lua"  -- (find-LATEX "edrxtikz.lua")
%L dofile "edrxpict.lua"  -- (find-LATEX "edrxpict.lua")
\pu

% «defs»  (to ".defs")
% (find-LATEX "edrx15.sty" "colors-2019")
%\long\def\ColorRed   #1{{\color{Red1}#1}}
%\long\def\ColorViolet#1{{\color{MagentaVioletLight}#1}}
%\long\def\ColorViolet#1{{\color{Violet!50!black}#1}}
%\long\def\ColorGreen #1{{\color{SpringDarkHard}#1}}
%\long\def\ColorGreen #1{{\color{SpringGreenDark}#1}}
%\long\def\ColorGreen #1{{\color{SpringGreen4}#1}}
%\long\def\ColorGray  #1{{\color{GrayLight}#1}}
%\long\def\ColorGray  #1{{\color{black!30!white}#1}}
%\long\def\ColorBrown #1{{\color{Brown}#1}}
%\long\def\ColorBrown #1{{\color{brown}#1}}
%\long\def\ColorOrange#1{{\color{orange}#1}}
%
%\long\def\ColorShort #1{{\color{SpringGreen4}#1}}
%\long\def\ColorLong  #1{{\color{Red1}#1}}
%
%\def\frown{\ensuremath{{=}{(}}}
%\def\True {\mathbf{V}}
%\def\False{\mathbf{F}}
%\def\D    {\displaystyle}

\def\degs{^\circ}

\def\drafturl{http://angg.twu.net/LATEX/2021-1-C3.pdf}
\def\drafturl{http://angg.twu.net/2021.1-C3.html}
\def\draftfooter{\tiny \href{\drafturl}{\jobname{}} \ColorBrown{\shorttoday{} \hours}}

% «defs-T-and-B»  (to ".defs-T-and-B")
% (c3m202p1p 6 "questao-2")
% (c3m202p1a   "questao-2")
\long\def\ColorOrange#1{{\color{orange!90!black}#1}}
\def\T(Total: #1 pts){{\bf(Total: #1)}}
\def\T(Total: #1 pts){{\bf(Total: #1 pts)}}
\def\T(Total: #1 pts){\ColorRed{\bf(Total: #1 pts)}}
\def\B       (#1 pts){\ColorOrange{\bf(#1 pts)}}



%  _____ _ _   _                               
% |_   _(_) |_| | ___   _ __   __ _  __ _  ___ 
%   | | | | __| |/ _ \ | '_ \ / _` |/ _` |/ _ \
%   | | | | |_| |  __/ | |_) | (_| | (_| |  __/
%   |_| |_|\__|_|\___| | .__/ \__,_|\__, |\___|
%                      |_|          |___/      
%
% «title»  (to ".title")
% (c3m211p2p 1 "title")
% (c3m211p2a   "title")

\thispagestyle{empty}

\begin{center}

\vspace*{1.2cm}

{\bf \Large Cálculo 3 - 2021.1}

\bsk

P2 (segunda prova)

\bsk

Eduardo Ochs - RCN/PURO/UFF

\url{http://angg.twu.net/2021.1-C3.html}

\end{center}

\newpage

% «regras-e-dicas»  (to ".regras-e-dicas")
% (c3m211p2p 2 "regras-e-dicas")
% (c3m211p2a   "regras-e-dicas")

{\bf Regras e dicas}

As regras e dicas são as mesmas dos mini-testes:

\ssk

\url{http://angg.twu.net/LATEX/2020-2-C3-MT1.pdf}

\url{http://angg.twu.net/LATEX/2020-2-C3-MT2.pdf}

\ssk

exceto que a prova vai ser disponibilizada às 23:30 do dia

15/setembro/2021 e deve ser entregue até as 23:30 do dia

17/setembro/2021. \ColorRed{Talvez o Classroom esteja com a data

de entrega errada, como se o prazo fosse só de 24 horas}.

\bsk

Pra fazer essa prova você vai precisar de idéias que a gente

viu durante o curso todo. Se você precisar saber onde estão

as idéias necessárias pra resolver algum item pergunte

\ColorRed{no grupo do Telegram da turma} que eu respondo com um

link pros slides, vídeos, ou livros em que aquela idéia

aparece.




\newpage

% «questao-1»  (to ".questao-1")
% (c3m211p2p 3 "questao-1")
% (c3m211p2a   "questao-1")

{\bf Questão 1.}

\T(Total: 5.5 pts)

\ssk

Você deve ter lido que ``o gradiente de uma função aponta pra

direção de maior crescimento dela''. Nesta questão nós vamos

ver um modo de provar isto --- num caso particular em $\R^2$.

\def\questaoumcasogeral{
  \begin{array}[t]{rcl}
       F &:& \R^2 → \R  \\
     P_0 &∈& \R^2       \\
     \vv &=& ∇F(P_0)    \\
     \vv &=& \VEC{a,b}  \\
     \uu &⊥& \vv, \text{ obedecendo } ||\uu|| = ||\vv|| \\
       A &=& \setofst{Q∈\R^2}{d(P_0,Q) = ||\vv||} \\
       B &=& \{ P_0 + \VEC{\pm a, \pm b} \}
           ∪ \{ P_0 + \VEC{\pm b, \pm a} \} \\
       θ &∈& \R \\
     \ww &=& (\cos θ)\vv + (\senθ)\uu \\
  \end{array}
  }

O caso geral vai ser este aqui:
%
$$\scalebox{0.8}{$
  \questaoumcasogeral
  $}
$$

\newpage

% «questao-1-cont»  (to ".questao-1-cont")
% (c3m211p2p 4 "questao-1-cont")
% (c3m211p2a   "questao-1-cont")

{\bf Questão 1 (cont.)}

O caso particular vai ser o da coluna da direita abaixo:
%
$$\scalebox{0.9}{$
  \questaoumcasogeral
  %
  \hspace*{-1cm}
  %
  \begin{array}[t]{rcl}
    F(x,y) &=& 10 - 2x + y \\
       P_0 &=& (3,2)       \\
       \vv &=& ∇F(P_0)     \\[2pt]
       \uu &=& \VEC{1,2}   \\
  \end{array}
  $}
$$

\ColorRed{Dica:} neste caso particular o conjunto $B$ vai ser um

conjunto de 8 pontos equidistantes de $P_0$, todos com

coordenadas inteiras.

\ssk

\newpage

% «questao-1-abcde»  (to ".questao-1-abcde")
% (c3m211p2p 5 "questao-1-abcde")
% (c3m211p2a   "questao-1-abcde")

{\bf Questão 1 (cont.)}

Nesse caso particular,

\msk

a) \B(0.2 pts) Dê as coordenadas dos 8 pontos de $B$.

\msk

b) \B(0.2 pts) Represente graficamente num gráfico só:

$P_0$, $P_0 + \vv$ e $P_0 + \uu$ (como setas), $A$, $B$.

\msk

c) \B(0.5 pts) Faça um diagrama de numerozinhos pra função

$F$, mas no qual você só vai indicar os valores de $F(x,y)$ nos

pontos em que $(x,y)∈B$ e em $(x,y)=P_0$.

\msk

d) \B(0.1 pts) Seja $P_1$ o ponto de $B$ no qual $F(x,y)$ assume

o maior valor. Diga as coordenadas de $P_1$ e faça um círculo

em torno de $P_1$ na figura que você fez no item (c).

\msk

e) \B(1.0 pts) Acrescente à figura do seu item (c) as curvas de

nível da $F(x,y)$ que passam pelos pontos de $B$.

\newpage

% «questao-1-fg»  (to ".questao-1-fg")
% (c3m211p2p 6 "questao-1-fg")
% (c3m211p2a   "questao-1-fg")

{\bf Questão 1 (cont.)}

Ainda nesse caso particular,

\msk

f) \B(0.5 pts) Verifique que $P_1$ é um ponto da forma

$P_0 + α\vv$ para algum $α∈\R$. Qual é o valor de $α$?

\msk

g) \B(0.5 pts) Sejam $P_2$ e $P_3$ os dois pontos de $B$ em que

temos $F(P_0) = F(P_2) = F(P_3)$. Diga as coordenadas

de $P_2$ e $P_3$ e verifique que tanto $P_2$ quanto $P_3$ são da

forma $P_0 + \bb \uu$. Qual é o valor de $\bb$ associado ao $P_2$?

E qual é o valor de $\bb$ associado ao $P_3$?

\msk

\newpage

% «questao-1-hij»  (to ".questao-1-hij")
% (c3m211p2p 7 "questao-1-hij")
% (c3m211p2a   "questao-1-hij")

{\bf Questão 1 (cont.)}

Ainda nesse caso particular...

\msk

h) \B(0.5 pts) Na ``notação de físicos'' podemos dizer que 

``$\ww$ é função de $θ$'', e podemos escrever isto assim: $\ww = \ww(θ)$.

Represente graficamente num gráfico só, separado dos

gráficos anteriores: $P_0$, o conjunto $A$, e $P_0+\ww(θ)$ para estes

valores de $θ$: $0\degs$, $45\degs$, $90\degs$. Não esqueça de indicar qual é o $θ$

associado a cada seta!

\msk

i) \B(1.0 pts) Encontre uma fórmula para $F(P_0+\ww(θ))$.

Simplifique o resultado dela o máximo que puder.

\msk

j) \B(0.5 pts) Encontre uma fórmula para $\frac{d}{dθ} F(P_0+\ww(θ))$.

Simplifique o resultado dela o máximo que puder.


\newpage

% «questao-1-k»  (to ".questao-1-k")
% (c3m211p2p 8 "questao-1-k")
% (c3m211p2a   "questao-1-k")

{\bf Questão 1 (cont.)}

Ainda nesse caso particular...

\msk

k) \B(0.5 pts) Seja $z = z(θ) = F(P_0+\ww(θ))$, pra abreviar.

Faça o gráfico de $z(θ)$ --- com $θ$ crescendo pra direita e

$z$ crescendo pra cima --- e mostre no gráfico para quais

valores de $θ$ o valor de $z$ é maximo e mínimo.

\msk

\newpage

{\bf Questão 2.}

\T(Total: 3.0 pts)

\ssk

Aqui nós vamos generalizar a questão 1 ---

pra um caso particular bem mais geral que o anterior,

que é o da coluna da direita abaixo:
%
$$\scalebox{0.8}{$
  \questaoumcasogeral
  \hspace{-1.0cm}
  \begin{array}[t]{rcl}
    F(x,y) &=& αx + βy + γ \\
    (α,β)  &≠& (0,0) \\
    \uu    &=& \VEC{b, -a} \\
  \end{array}
  $}
$$

\newpage

{\bf Questão 2 (cont.)}

Seja $z = z(θ) = F(P_0+\ww(θ))$, pra abreviar.

\msk

a) \B(1.0 pts) Encontre uma fórmula para $z(θ)$.

Simplifique o resultado dela o máximo que puder.

\msk

b) \B(1.0 pts) Encontre uma fórmula para $\frac{d}{dθ} z(θ)$.

Simplifique o resultado dela o máximo que puder.

\msk

c) \B(1.0 pts) Faça o gráfico de $z(θ)$ --- com $θ$ crescendo

pra direita e $z$ crescendo pra cima --- e mostre no gráfico

para quais valores de $θ$ o valor de $z$ é maximo e mínimo.






\newpage

% «questao-3»  (to ".questao-3")
% (c3m211p2p 11 "questao-3")
% (c3m211p2a    "questao-3")

{\bf Questão 3.}

\T(Total: 2.0 pts)

\ssk

Na última aula antes da prova nós começamos a

fazer esse ``Exercício 5'' aqui, mas não terminamos...

\ssk

{\scriptsize

% (c3m211afp 9 "exercicio-5")
% (c3m211afa   "exercicio-5")
%    http://angg.twu.net/LATEX/2021-1-C3-abertos-e-fechados.pdf#page=9
\url{http://angg.twu.net/LATEX/2021-1-C3-abertos-e-fechados.pdf#page=9}

}

\ssk

Use as ``traduções'' dos slides 14 e 15 desse PDF sobre

abertos e fechados pra traduzir isto aqui pra uma outra

expressão que seja ``a mais simples possível'':
%
$$[2,4] \text{ não é aberto}$$



\newpage

% «questao-3-cont»  (to ".questao-3-cont")
% (c3m211p2p 12 "questao-3-cont")
% (c3m211p2a    "questao-3-cont")

{\bf Questão 3 (cont.)}

\ssk

Dá pra definir formalmente o que quer dizer esse 

``mais simples possível'' entre aspas, mas a definição

formal é meio horrível. A dica é que no slide 14 desse

PDF, cujo título é ``Algumas traduções'', cada igualdade

da tabela é desta forma:
%
$$\text{expressão ``mais complicada''}
  \;\;=\;\;
  \text{expressão ``mais simples''}
$$

com a expressão ``mais simples'' à direita.


\newpage

% «gabarito»  (to ".gabarito")
% (c3m211p2p 13 "gabarito")
% (c3m211p2a    "gabarito")

\thispagestyle{empty}

\begin{center}

\vspace*{2.0cm}

{\bf \Large Gabarito}

\end{center}


\newpage

% «gabarito-1abcde»  (to ".gabarito-1abcde")
% (c3m211p2p 14 "gabarito-1abcde")
% (c3m211p2a    "gabarito-1abcde")

{\bf Questões 1a até 1e}

(Os desenhos estão muito incompletos)

\msk

\unitlength=10pt

1a)
%
$\vcenter{\hbox{%
    \beginpicture(0,-1)(6,5)
    \pictgrid%
    \pictpiecewise{(3,2)o (5,3)c (5,1)c (4,0)c (2,0)c (1,1)c (1,3)c (2,4)c (4,4)c}%
    \pictaxes%
    \end{picture}%
  }}
$
%
$\begin{array}{c}
 (2,4), (4,4),        \\
 (1,3), \ph{m} (5,3), \\
 (1,1), \ph{m} (5,1), \\
 (2,0), (4,0)         \\
 \end{array}
$

\msk

1b) $\sm{
     \vv &=& \VEC{-2,1} \\
     \uu &=& \VEC{1,2} \\
     }
     \quad
     \sm{
     P_0 &=& (3,2) \\
     P_0 + \vv &=& (1,3) \\
     P_0 + \uu &=& (4,4) \\
     }
    $

\msk

% -- (c3m211p1p 7 "questao-1-gab")
% -- (c3m211p1a   "questao-1-gab")
%
%  (eepitch-lua51)
%  (eepitch-kill)
%  (eepitch-lua51)
%
% F = function (x,y) return 10 - 2*x + y end
% pts = {{3,2},
%  {2,4}, {4,4},        
%  {1,3}, {5,3}, 
%  {1,1}, {5,1}, 
%  {2,0}, {4,0}         
% }
% for _,xy in pairs(pts) do
%   local x,y = xy[1], xy[2]
%   local z = F(x,y)
%   -- print(x, y, z)
%   print(format("\\put(%d,%d){\\cellc{%s}}%%", x,y, z))
% end

% (c3m211p1p 7 "questao-1-gab")
% (c3m211p1a   "questao-1-gab")
%
\def\cellhc#1{\hbox to 0pt{\hss\cellfont${#1}$\hss}}
\def\cellvm#1{\setbox0#1\lower \celllower \box0}
\def\cellvb#1{\setbox0#1\lower \ht0       \box0}
\def\cellvm#1{\setbox0#1\lower 0.4\ht0    \box0}
\def\cellc   #1{\cellvm{\cellhc{#1}}}

1c)
%
$\vcenter{\hbox{%
    \beginpicture(0,-1)(6,5)
    \pictgrid%
      \put(3,2){\cellc{6}}%
      \put(2,4){\cellc{10}}%
      \put(4,4){\cellc{6}}%
      \put(1,3){\cellc{11}}%
      \put(5,3){\cellc{3}}%
      \put(1,1){\cellc{9}}%
      \put(5,1){\cellc{1}}%
      \put(2,0){\cellc{6}}%
      \put(4,0){\cellc{2}}%
    \pictaxes%
    \end{picture}%
  }}
$

\msk
1d) $P_1 = (1,3)$


\newpage

% «gabarito-1fgh»  (to ".gabarito-1fgh")
% (c3m211p2p 15 "gabarito-1fgh")
% (c3m211p2a    "gabarito-1fgh")

{\bf Questões 1f até 1h}

\msk

1f) $P_1 = (1,3) = P_0 + 1 \vv = (3,2) + 1 \VEC{-2,1}$

\msk

1g) $\begin{array}[t]{l}
     P_2 = (2,0) = P_0 + (-1) · \uu = (3,2) + (-1)·\VEC{1, 2} \\
     P_3 = (4,4) = P_0 + 1 · \uu = (3,2) + 1·\VEC{1, 2} \\
     \end{array}
    $

\msk

\def\sqt{\frac{\sqrt2}{2}}

1h) $\scalebox{0.8}{$
     \begin{array}{rcccl}
     \ww(0\degs)  &=&  (\cos 0\degs)\vv +  (\sen 0\degs)\vv &=& \VEC{-2,1} + \VEC{0,0} \\
     \ww(45\degs) &=& (\cos 45\degs)\vv + (\sen 45\degs)\vv &=& \sqt\VEC{-2,1} + \sqt\VEC{1,2} \\
     \ww(90\degs) &=& (\cos 90\degs)\vv + (\sen 90\degs)\vv &=& \VEC{0,0} + \VEC{1,2} \\
     P_0 + \ww(0\degs)  &=& (3,2) + \VEC{-2,1} \ph{mi}\\
     P_0 + \ww(45\degs) &=& (3,2) + \sqt\VEC{-1,3} \\
     P_0 + \ww(90\degs) &=& (3,2) + \VEC{1,2} \ph{miii} \\
     \end{array}
     $}
    $
%
\unitlength=5pt
%
$$\vcenter{\hbox{%
    \beginpicture(0,0)(6,5)
    \pictgrid%
    \pictpiecewise{(3,2)o (1,3)c (2.3,4.1)c (4,4)c}%
    \pictaxes%
    \end{picture}%
  }}
$$

\newpage

% «gabarito-1ij»  (to ".gabarito-1ij")
% (c3m211p2p 16 "gabarito-1ij")
% (c3m211p2a    "gabarito-1ij")

{\bf Questões 1i e 1j}

\msk

1i) $\scalebox{0.8}{$
     \begin{array}{rcl}
     F(P_0 + \ww(θ))
       &=& F((3,2) + (\cos θ)\VEC{-2,1} + (\senθ)\VEC{1,2}) \\
       &=& F((3,2) + \VEC{-2\cos θ + \senθ, \cosθ + 2\senθ}) \\
       &=& F((3 -2\cos θ + \senθ, 2 + \cosθ + 2\senθ)) \\
       &=& 10 -2(3 -2\cos θ  + \senθ) + (2 + \cosθ + 2\senθ) \\
       &=& 10 -6   +4\cos θ -2 \senθ  +  2 + \cosθ + 2\senθ  \\
       &=& 6 + 5\cosθ  \\
     \end{array}
     $}
    $

\msk

\def\ddth{\frac{d}{dθ}}

1j) $\scalebox{0.8}{$
     \begin{array}{rcl}
     \ddth F(P_0 + \ww(θ))
       &=& \ddth (6 + 5\cosθ) \\
       &=& - 5\senθ \\
     \end{array}
     $}
    $


\newpage

% «gabarito-2ab»  (to ".gabarito-2ab")
% (c3m211p2p 17 "gabarito-2ab")
% (c3m211p2a    "gabarito-2ab")

{\bf Questões 2a e 2b}

\msk

2a) Temos $\vv = ∇F(P_0) = \VEC{α,β}$ e $\vv = \VEC{a,b}$,

\ph{2a) }Então $α=a$ e $β=b$. Aí:
%
$$\scalebox{0.8}{$
    \begin{array}{rcl}
      z(θ) &=& F(P_0 + \ww(θ)) \\
       &=& F((x_0,y_0) + (\cos θ)\VEC{a,b} + (\senθ)\VEC{b,-a}) \\
       &=& F((x_0,y_0) + \VEC{(\cos θ)a + (\senθ)b, (\cos θ)b - (\senθ)a}) \\
       &=& α(x_0 + (\cos θ)a + (\senθ)b) + β(y_0 + (\cos θ)b - (\senθ)a) + γ \\
       &=& α(x_0 + (\cos θ)α + (\senθ)β) + β(y_0 + (\cos θ)β - (\senθ)α) + γ \\
       &=& αx_0 + α^2(\cos θ) + αβ(\senθ) + βy_0 + β^2(\cos θ) - αβ(\senθ) + γ \\
       &=& αx_0 + α^2(\cos θ)             + βy_0 + β^2(\cos θ)             + γ \\
       &=& (αx_0 + βy_0 + γ) + (α^2 + β^2)(\cos θ)                             \\
       &=& F(P_0)            + (α^2 + β^2)(\cos θ)                             \\
   \end{array}
  $}
$$

2b)
%
 $\scalebox{0.8}{$
    \begin{array}{rcl}
      \ddth z(θ) &=& \ddth((αx_0 + βy_0 + γ) + (α^2 + β^2)(\cos θ)) \\
                 &=& (α^2 + β^2)(- \sen θ) \\
   \end{array}
  $}
 $

\newpage

% «gabarito-3»  (to ".gabarito-3")
% (c3m211p2p 10 "gabarito-3")
% (c3m211p2a    "gabarito-3")
% (c3m211afp 9 "exercicio-5")
% (c3m211afa   "exercicio-5")

{\bf Gabarito da 3}

\ssk

\def\BA{\mathsf{B}}
\def\BF{\overline{\mathsf{B}}}

Lembre que $A = [2,4]$. Temos:
%
\def\iff{\Leftrightarrow}
%
$$\begin{array}{c}
  \begin{array}{lrcl}
                 & A & \multicolumn{2}{l}{\text{não é aberto}} \\ 
   \iff \ph{m}   & A & \not⊂ & \Int(A) \\
   \iff          & A & \not⊂ & \setofst{P∈A}{∃ε>0.\, \BA_ε(P)⊆A} \\
   \iff          & [2,4] & \not⊂ & \setofst{P∈[2,4]}{∃ε>0.\, \BA_ε(P)⊆[2,4]} \\
  \end{array}
  \\
  [25pt]
  \begin{array}{ll}
   \iff & ∃x∈[2,4]. \; x \not∈ \setofst{P∈[2,4]}{∃ε>0.\, \BA_ε(P)⊆[2,4]} \\
   \iff & ∃x∈[2,4]. ¬ (x ∈ \setofst{P∈[2,4]}{∃ε>0.\, \BA_ε(P)⊆[2,4]}) \\
   \iff & ∃x∈[2,4]. ¬ (x ∈ [2,4] ∧ ∃ε>0.\, \BA_ε(x)⊆[2,4]) \\
   \iff & ∃x∈[2,4]. (¬ x ∈ [2,4]) ∨ (¬ \; ∃ε>0.\, \BA_ε(x)⊆[2,4]) \\
   \iff & ∃x∈[2,4]. (¬ x ∈ [2,4]) ∨ (∀ε>0. ¬ (\BA_ε(x)⊆[2,4])) \\
   \iff & ∃x∈[2,4]. (¬ x ∈ [2,4]) ∨ (∀ε>0. ¬ (∀y∈\BA_ε(x).\; y∈[2,4])) \\
   \iff & ∃x∈[2,4]. (¬ x ∈ [2,4]) ∨ (∀ε>0. \; ∃y∈\BA_ε(x).\; ¬ y∈[2,4])) \\
   \iff & \ldots \\
  \end{array}
  \end{array}
$$





%\printbibliography

\GenericWarning{Success:}{Success!!!}  % Used by `M-x cv'

\end{document}

%  ____  _             _         
% |  _ \(_)_   ___   _(_)_______ 
% | | | | \ \ / / | | | |_  / _ \
% | |_| | |\ V /| |_| | |/ /  __/
% |____// | \_/  \__,_|_/___\___|
%     |__/                       
%
% «djvuize»  (to ".djvuize")
% (find-LATEXgrep "grep --color -nH --null -e djvuize 2020-1*.tex")

 (eepitch-shell)
 (eepitch-kill)
 (eepitch-shell)
# (find-fline "~/2021.1-C3/")
# (find-fline "~/LATEX/2021-1-C3/")
# (find-fline "~/bin/djvuize")

cd /tmp/
for i in *.jpg; do echo f $(basename $i .jpg); done

f () { rm -v $1.pdf;  textcleaner -f 50 -o  5 $1.jpg $1.png; djvuize $1.pdf; xpdf $1.pdf }
f () { rm -v $1.pdf;  textcleaner -f 50 -o 10 $1.jpg $1.png; djvuize $1.pdf; xpdf $1.pdf }
f () { rm -v $1.pdf;  textcleaner -f 50 -o 20 $1.jpg $1.png; djvuize $1.pdf; xpdf $1.pdf }

f () { rm -fv $1.png $1.pdf; djvuize $1.pdf }
f () { rm -fv $1.png $1.pdf; djvuize WHITEBOARDOPTS="-m 1.0 -f 15" $1.pdf; xpdf $1.pdf }
f () { rm -fv $1.png $1.pdf; djvuize WHITEBOARDOPTS="-m 1.0 -f 30" $1.pdf; xpdf $1.pdf }
f () { rm -fv $1.png $1.pdf; djvuize WHITEBOARDOPTS="-m 1.0 -f 45" $1.pdf; xpdf $1.pdf }
f () { rm -fv $1.png $1.pdf; djvuize WHITEBOARDOPTS="-m 0.5" $1.pdf; xpdf $1.pdf }
f () { rm -fv $1.png $1.pdf; djvuize WHITEBOARDOPTS="-m 0.25" $1.pdf; xpdf $1.pdf }
f () { cp -fv $1.png $1.pdf       ~/2021.1-C3/
       cp -fv        $1.pdf ~/LATEX/2021-1-C3/
       cat <<%%%
% (find-latexscan-links "C3" "$1")
%%%
}

f 20201213_area_em_funcao_de_theta
f 20201213_area_em_funcao_de_x
f 20201213_area_fatias_pizza



%  __  __       _        
% |  \/  | __ _| | _____ 
% | |\/| |/ _` | |/ / _ \
% | |  | | (_| |   <  __/
% |_|  |_|\__,_|_|\_\___|
%                        
% <make>

 (eepitch-shell)
 (eepitch-kill)
 (eepitch-shell)
# (find-LATEXfile "2019planar-has-1.mk")
make -f 2019.mk STEM=2021-1-C3-P2 veryclean
make -f 2019.mk STEM=2021-1-C3-P2 pdf

% Local Variables:
% coding: utf-8-unix
% ee-tla: "c3p2"
% ee-tla: "c3m211p2"
% End:
